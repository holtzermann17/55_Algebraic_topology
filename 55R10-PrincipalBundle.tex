\documentclass[12pt]{article}
\usepackage{pmmeta}
\pmcanonicalname{PrincipalBundle}
\pmcreated{2013-03-22 13:07:18}
\pmmodified{2013-03-22 13:07:18}
\pmowner{rmilson}{146}
\pmmodifier{rmilson}{146}
\pmtitle{principal bundle}
\pmrecord{8}{33555}
\pmprivacy{1}
\pmauthor{rmilson}{146}
\pmtype{Definition}
\pmcomment{trigger rebuild}
\pmclassification{msc}{55R10}
\pmdefines{principal G-bundle}

\endmetadata

% this is the default PlanetMath preamble.  as your knowledge
% of TeX increases, you will probably want to edit this, but
% it should be fine as is for beginners.

% almost certainly you want these
\usepackage{amssymb}
\usepackage{amsmath}
\usepackage{amsfonts}

% used for TeXing text within eps files
%\usepackage{psfrag}
% need this for including graphics (\includegraphics)
%\usepackage{graphicx}
% for neatly defining theorems and propositions
%\usepackage{amsthm}
% making logically defined graphics
%%%\usepackage{xypic} 

% there are many more packages, add them here as you need them

% define commands here
\begin{document}
Let $E$ be a topological space on which a topological group $G$ acts continuously and freely.  The map $\pi:E\to E/G=B$ is called a \emph{principal bundle} (or \emph{principal $G$-bundle}) if the projection
map $\pi:E\to B$ is a locally trivial bundle.

Any principal bundle with a section $\sigma:B\to E$ is trivial, since the map $\phi:B\times G\to E$ given by $\phi(b,g)=g\cdot\sigma(b)$ is an isomorphism. In particular, any $G$-bundle which is topologically trivial is also isomorphic to $B\times G$ as a $G$-space.  Thus any local trivialization of $\pi:E\to B$ as a topological bundle is an equivariant trivialization.
%%%%%
%%%%%
\end{document}

\documentclass[12pt]{article}
\usepackage{pmmeta}
\pmcanonicalname{BibliographyOfQuantumAlgebraAndQuantumSymmetry}
\pmcreated{2013-03-22 19:21:31}
\pmmodified{2013-03-22 19:21:31}
\pmowner{bci1}{20947}
\pmmodifier{bci1}{20947}
\pmtitle{bibliography of quantum algebra and quantum symmetry}
\pmrecord{13}{42311}
\pmprivacy{1}
\pmauthor{bci1}{20947}
\pmtype{Bibliography}
\pmcomment{trigger rebuild}
\pmclassification{msc}{55Q35}
\pmclassification{msc}{55Q05}
\pmclassification{msc}{20L05}
\pmclassification{msc}{18D05}
\pmclassification{msc}{18-00}

% this is the default PlanetMath preamble. as your knowledge
% of TeX increases, you will probably want to edit this, but
\usepackage{amsmath, amssymb, amsfonts, amsthm, amscd, latexsym}
%%\usepackage{xypic}
\usepackage[mathscr]{eucal}
% define commands here
\theoremstyle{plain}
\newtheorem{lemma}{Lemma}[section]
\newtheorem{proposition}{Proposition}[section]
\newtheorem{theorem}{Theorem}[section]
\newtheorem{corollary}{Corollary}[section]
\theoremstyle{definition}
\newtheorem{definition}{Definition}[section]
\newtheorem{example}{Example}[section]
%\theoremstyle{remark}
\newtheorem{remark}{Remark}[section]
\newtheorem*{notation}{Notation}
\newtheorem*{claim}{Claim}
\renewcommand{\thefootnote}{\ensuremath{\fnsymbol{footnote%%@
}}}
\numberwithin{equation}{section}
\newcommand{\Ad}{{\rm Ad}}
\newcommand{\Aut}{{\rm Aut}}
\newcommand{\Cl}{{\rm Cl}}
\newcommand{\Co}{{\rm Co}}
\newcommand{\DES}{{\rm DES}}
\newcommand{\Diff}{{\rm Diff}}
\newcommand{\Dom}{{\rm Dom}}
\newcommand{\Hol}{{\rm Hol}}
\newcommand{\Mon}{{\rm Mon}}
\newcommand{\Hom}{{\rm Hom}}
\newcommand{\Ker}{{\rm Ker}}
\newcommand{\Ind}{{\rm Ind}}
\newcommand{\IM}{{\rm Im}}
\newcommand{\Is}{{\rm Is}}
\newcommand{\ID}{{\rm id}}
\newcommand{\GL}{{\rm GL}}
\newcommand{\Iso}{{\rm Iso}}
\newcommand{\Sem}{{\rm Sem}}
\newcommand{\St}{{\rm St}}
\newcommand{\Sym}{{\rm Sym}}
\newcommand{\SU}{{\rm SU}}
\newcommand{\Tor}{{\rm Tor}}
\newcommand{\U}{{\rm U}}
\newcommand{\A}{\mathcal A}
\newcommand{\Ce}{\mathcal C}
\newcommand{\D}{\mathcal D}
\newcommand{\E}{\mathcal E}
\newcommand{\F}{\mathcal F}
\newcommand{\G}{\mathcal G}
\newcommand{\Q}{\mathcal Q}
\newcommand{\R}{\mathcal R}
\newcommand{\cS}{\mathcal S}
\newcommand{\cU}{\mathcal U}
\newcommand{\W}{\mathcal W}
\newcommand{\bA}{\mathbb{A}}
\newcommand{\bB}{\mathbb{B}}
\newcommand{\bC}{\mathbb{C}}
\newcommand{\bD}{\mathbb{D}}
\newcommand{\bE}{\mathbb{E}}
\newcommand{\bF}{\mathbb{F}}
\newcommand{\bG}{\mathbb{G}}
\newcommand{\bK}{\mathbb{K}}
\newcommand{\bM}{\mathbb{M}}
\newcommand{\bN}{\mathbb{N}}
\newcommand{\bO}{\mathbb{O}}
\newcommand{\bP}{\mathbb{P}}
\newcommand{\bR}{\mathbb{R}}
\newcommand{\bV}{\mathbb{V}}
\newcommand{\bZ}{\mathbb{Z}}
\newcommand{\bfE}{\mathbf{E}}
\newcommand{\bfX}{\mathbf{X}}
\newcommand{\bfY}{\mathbf{Y}}
\newcommand{\bfZ}{\mathbf{Z}}
\renewcommand{\O}{\Omega}
\renewcommand{\o}{\omega}
\newcommand{\vp}{\varphi}
\newcommand{\vep}{\varepsilon}
\newcommand{\diag}{{\rm diag}}
\newcommand{\grp}{{\mathbb G}}
\newcommand{\dgrp}{{\mathbb D}}
\newcommand{\desp}{{\mathbb D^{\rm{es}}}}
\newcommand{\Geod}{{\rm Geod}}
\newcommand{\geod}{{\rm geod}}
\newcommand{\hgr}{{\mathbb H}}
\newcommand{\mgr}{{\mathbb M}}
\newcommand{\ob}{{\rm Ob}}
\newcommand{\obg}{{\rm Ob(\mathbb G)}}
\newcommand{\obgp}{{\rm Ob(\mathbb G')}}
\newcommand{\obh}{{\rm Ob(\mathbb H)}}
\newcommand{\Osmooth}{{\Omega^{\infty}(X,*)}}
\newcommand{\ghomotop}{{\rho_2^{\square}}}
\newcommand{\gcalp}{{\mathbb G(\mathcal P)}}
\newcommand{\rf}{{R_{\mathcal F}}}
\newcommand{\glob}{{\rm glob}}
\newcommand{\loc}{{\rm loc}}
\newcommand{\TOP}{{\rm TOP}}
\newcommand{\wti}{\widetilde}
\newcommand{\what}{\widehat}
\renewcommand{\a}{\alpha}
\newcommand{\be}{\beta}
\newcommand{\ga}{\gamma}
\newcommand{\Ga}{\Gamma}
\newcommand{\de}{\delta}
\newcommand{\del}{\partial}
\newcommand{\ka}{\kappa}
\newcommand{\si}{\sigma}
\newcommand{\ta}{\tau}
\newcommand{\lra}{{\longrightarrow}}
\newcommand{\ra}{{\rightarrow}}
\newcommand{\rat}{{\rightarrowtail}}
\newcommand{\oset}[1]{\overset {#1}{\ra}}
\newcommand{\osetl}[1]{\overset {#1}{\lra}}
\newcommand{\hr}{{\hookrightarrow}}

\begin{document}
\section{Topical references for Quantum Symmetry and Quantum Operator Algebra}

\begin{thebibliography}{199}


\bibitem{Abragam-Bleaney70}
Abragam, A.; Bleaney, B.  \emph{Electron paramagnetic resonance of transition ions.}; Clarendon Press: Oxford, 1970.

\bibitem{Aguiar-Andrusk2k4}
Aguiar, M.; Andruskiewitsch, N.  Representations of matched pairs of groupoids and applications to weak Hopf algebras. {\it Contemp. Math.} {\bf 2005}, {\em 376}, 127--173.  

%%$\href{http://arxiv.org/abs/math.QA/0402118}{math.QA/0402118}$.

\bibitem{Aguiar2k9a}
Aguiar, M.C.O.; Dobrosavljevic, V.; Abrahams, E.; Kotliar G.  Critical behavior at Mott--Anderson transition: a TMT-DMFT perspective. \emph{Phys. Rev. Lett.} {\bf 2009}, {\em 102}, 156402, 4~pages. 

%%$\href{http://arxiv.org/abs/0811.4612}{arXiv:0811.4612}$.

\bibitem{Aguiar2k9b}
Aguiar, M.C.O.; Dobrosavljevic, V.; Abrahams, E.; Kotliar G.  Scaling behavior of an Anderson impurity close to the Mott-Anderson transition. \textit{Phys. Rev.~B} {\bf 2006}, {\em 73}, 115117, 7~pages. 

%%\href{http://arxiv.org/abs/cond-mat/0509706}{cond-mat/0509706}.

\bibitem{Alfsen-Schultz2k3}
Alfsen, E.M.; Schultz, F.W. {\em Geometry of state spaces of operator algebras.}; Birkh$\"a$user: Boston~-- Basel~-- Berlin, 2003.

\bibitem{Altintas-Arika2k8}
Altintash, A.A.; Arika, M.  The inhomogeneous invariance quantum supergroup of supersymmetry algebra. \textit{Phys. Lett. A} {\bf 2008}, {\em 372}, 5955--5958.

\bibitem{Anderson58}
Anderson, P.W.  Absence of diffusion in certain random lattices. \textit{Phys. Rev.} {\bf 1958}, {\em 109}, 1492--1505.

\bibitem{Anderson77}
Anderson, P.W. Topology of Glasses and Mictomagnets. {\em  Lecture presented at The Cavendish Laboratory}: Cambridge, UK, 1977.

\bibitem{Baez-Huerta2010}
Baez, J. and  Huerta, J.  {\em An Invitation to Higher Gauge Theory}. {\bf 2010}, {\em Preprint, March 23, 2010}: Riverside, CA;pp. 60.  http://www.math.ucr.edu/home/baez/invitation.pdf 
; 

%%\href{http://arxiv.org/abs/hep-th/1003.4485}{arXiv:1003.4485v1} 

\bibitem{Baez-Schr2k8}
Baez, J. and  Schreiber, U. {\em  Higher Gauge Theory II: 2-Connections} (JHEP Preprint), 2008;pp.75. \\
 http://math.ucr.edu/home/baez/2conn.pdf.

\bibitem{Baianu2010}
Baianu, I.C.; Editor. {\em Quantum Algebra and Symmetry: Quantum Algebraic Topology, Quantum Field Theories and Higher Dimensional Algebra}; PediaPress GmbH: Mainz, Second Edition, Vol. \\{\em 1}: {\em Quantum Algebra, Group Representations and Symmetry}; Vol.{\em 2}: {\em Quantum Algebraic Topology: \\ QFTs, SUSY, HDA};Vol.{\em 3}: {\em Quantum Logics and Quantum Biographies}, December 20, 2010;pp. 1,068.

\bibitem{Baianu71}
Baianu, I.C. Categories, functors and automata theory: a novel approach to quantum automata through algebraic-topological quantum computation. In {\em  Proceedings of the 4th Intl. Congress of Logic, Methodology and Philosophy of Science, Bucharest, August~-- September, 1971}; University of Bucharest: Bucharest, 1971;pp. 256--257.

\bibitem{Baianu74}
Baianu, I.C. \emph{Structural studies of erythrocyte and bacterial cytoplasmic membranes by X-ray diffraction and electron microscopy.} PhD Thesis; Queen Elizabeth College: University of London, 1974.

\bibitem{Baianu78}
Baianu, I.C.  X-ray scattering by partially disordered membrane systems. 
\emph{Acta Cryst. A}, 1978, {\bf 34}: 751--753.

\bibitem{Baianu-etal78a}
Baianu, I.C.; Boden, N.; Levine, Y.K.;Lightowlers, D.  Dipolar coupling between groups of three spin-1/2 undergoing hindered reorientation in solids. {\it Solid State Comm.} 1978, {\bf 27}, 474--478.

\bibitem{Baianu80}
Baianu, I.C.  Structural order and partial disorder in biological systems. {\em Bull. Math. Biology} {\bf 1980}, {\em 42}, 464--468.

\bibitem{Baianu-etal78b}
Baianu, I.C.; Boden N.; Levine Y.K.; Ross S.M.  Quasi-quadrupolar NMR spin-Echoes in solids containing dipolar coupled methyl groups.  {\em J. Phys.  C: Solid State Phys.}
{\bf 1978}, {\em 11}, L37--L41.

\bibitem{Baianu-etal81}
Baianu, I.C.; Boden, N.;Lightowlers, D.  NMR spin-echo responses of dipolar-coupled spin-1/2 triads in solids. {\em J. Magnetic Resonance} {\bf 1981},{\em 43}, 101--111.

\bibitem{Baianu-etal78c}
Baianu, I.C.; Boden, N.; Mortimer, M.; Lightowlers, D.  A new approach to the structure of concentrated aqueous electrolyte solutions by pulsed N.M.R. {\em Chem. Phys. Lett.}
{\bf 1978}, {\em 54}, 169--175.

\bibitem{Baianu-etal2k7}
Baianu, I.C.; Glazebrook J.F.; Brown, R.  A non-Abelian, Categorical Ontology of Spacetimes and Quantum Gravity. {\em Axiomathes}  {\bf 2007}, {\em 17}, 353--408.

\bibitem{BGB1}
Baianu, I. C.; Glazebrook, J. F.; Brown, R.  Algebraic Topology Foundations of Supersymmetry and Symmetry Breaking in Quantum Field Theory and Quantum Gravity: A Review.
{\em Sigma} {\bf 2009}, {\em 5}, 70 pages.

\bibitem{Baianu-etal79a}
Baianu, I.C.; Rubinson,  K.A.; Patterson, J. The observation of structural relaxation in a FeNiPB glass by $X$-ray scattering and ferromagnetic resonance. {\em Phys. Status
Solidi A} {\bf 1979}, {\em 53}, K133--K135.

\bibitem{Baianu-etal79b}
Baianu, I.C.; Rubinson,  K.A.;  Patterson, J. Ferromagnetic resonance and spin--wave excitations in metallic glasses. \emph{J. Phys. Chem. Solids} {\bf 1979}, {\em 40}, 940--951.

\bibitem{BSS2k2}
Bais, F. A.;  Schroers, B. J.; and  Slingerland, J. K.  Broken quantum symmetry and confinement phases in planar physics. \emph{Phys. Rev. Lett.} \textbf{2002}, {\em 89}, No. 18 (1--4), 181--201. 

%%\href{http://arxiv.org/abs/hep-th/0205117}{hep-th/0205117}.

\bibitem{Ball2k5}
Ball, R.C.  Fermions without fermion fields. {\em Phys. Rev. Lett.} \textbf{95} (2005), 176407, 4~pages.

%% \href{http://arxiv.org/abs/cond-mat/0409485}{cond-mat/0409485}.

\bibitem{Banica96}
Banica, T. The'orie des repre'sentations du groupe quantique compact libre $O(n)$. {\em  C. R. Acad. Sci. Paris.}, {\bf 1996}, {\em 322}, Serie I, 241--244.

\bibitem{Banica2k}
Banica, T.  Compact Kac algebras and commuting squares. {\em J. Funct. Anal.}  {\bf 2000} ,{\em 176}, no. 1, 80--99.

\bibitem{Barrett2k3}
Barrett, J.W.  Geometrical measurements in three-dimensional quantum gravity.  In {\em Proceedings of the Tenth Oporto Meeting on Geometry, Topology and Physics (2001)}, \textit{Internat. J. Modern Phys. A}  {\bf 2003}, {\em 18}, October, suppl., 97--113.  

%%\href{http://arxiv.org/abs/gr-qc/0203018}{gr-qc/0203018}.

\bibitem{Barrett-Mackaay2k6}
Barrett, J.W.; Mackaay, M.  Categorical representation of categorical groups. \emph{Theory Appl. Categ.}  {\bf 2006},{\em 16}, 529--557.  %%\href{http://arxiv.org/abs/math.CT/0407463}{math.CT/0407463}.

\bibitem{HJB-DC91}
Baues, H.J. ; Conduch{\'e}, D.  On the tensor algebra of a nonabelian group and applications. \emph{$K$-Theory}  {\bf 1991/92}, {\em 5} ), 531--554.

\bibitem{Bellissard1}
Bellissard, J.  K-theory of C*-algebras in solid state physics. \emph{Statistical Mechanics and Field Theory: Mathematical Aspects}
;Dorlas, T. C. et al., Editors; Springer Verlag. \emph{Lect. Notes in Physics}, {\bf 1986}, {\em 257}, 99--156.

\bibitem{Bichon2k8}
Bichon, J.  Algebraic quantum permutation groups. {\em Asian-Eur. J. Math.} {\bf 2008}, {\em 1}, no. 1, 1--13. arXiv:0710.1521

\bibitem{Bichonetal2k6}
Bichon, J.; De Rijdt, A.; Vaes, S.  Ergodic coactions with large multiplicity and monoidal equivalence of quantum groups. {\em Comm. Math. Phys.} {\bf 2006}, {\em 262}, 703--728. 

\bibitem{Blaom2k7}
Blaom, A.D.  Lie algebroids and Cartan's method of equivalence. %%\href{http://arxiv.org/abs/math.DG/0509071}{math.DG/0509071}.

\bibitem{Blok-Wen92}
Blok, B.; Wen X.-G. Many-body systems with non-Abelian statistics. {\em Nuclear Phys. B} {\bf 1992}, {\em 374}, 615--619.

\bibitem{BEK2k}
B$\"o$ckenhauer, J.; Evans, D.; Kawahigashi, Y.  Chiral structure of modular invariants for subfactors. \emph{Comm. Math. Phys.} {\bf 2000}, {\bf 210}, 733--784, 
%%\href{http://arxiv.org/abs/math.OA/9907149}{math.OA/9907149}.

\bibitem{Bohm-etal99}
B$\"o$hm, G.; Nill F.; Szlach\'anyi, K.  Weak Hopf algebras. I.~Integral theory and $C^*$-structure, \emph{J. Algebra}  {\bf 1999},{\em 221}, 385--438. %%\href{http://arxiv.org/abs/math.OA/9805116}{math.OA/9805116}.

\bibitem{Bohm-Gadella89}
Bohm, A.; Gadella, M.  Dirac kets, Gamow vectors and Gelfan'd triples. The rigged Hilbert space formulation of quantum mechanics, {\it  Lecture Notes in Physics}, Vol.~\emph{348}; Springer-Verlag:Berlin, 1989.

\bibitem{Borceux-Janelidze2k1}
Borceux, F.; Janelidze, G.  Galois Theories. {\em Cambridge Studies in Advanced Mathematics}, Vol.{\em 72}; Cambridge University Press: Cambridge, 2001.

\bibitem{Bos2k8}
Bos, R.  Continuous representations of groupoids, \emph{Houston J. Math}, {\em in press}
%%: available at  \url{http://www.math.ru.nl/~rdbos/ContinReps.pdf};

 %%\href{http://arxiv.org/abs/math.RT/0612639}{math.RT/0612639}.

\bibitem{Bos2k7a}
Bos, R.  Geometric quantization of Hamiltonian actions of Lie algebroids and Lie groupoids, \textit{Int. J. Geom. Methods Mod. Phys.}  {\bf 2007}, {\em 4}, 389--436, 

%%\href{http://arxiv.org/abs/math.SG/0604027}{math.SG/0604027}.

\bibitem{Bos2k7b}
Bos, R.  {\em Groupoids in geometric quantization}. PhD thesis: Radboud University Nijmegen, 2007.

\bibitem{Brauner2010}
Brauner, T. Spontaneous Symmetry Breaking and Nambu--Goldstone Bosons in Quantum Many-Body Systems. {\em Symmetry}  {\bf 2010}, {\em 2}, 609-657.

\bibitem{RB1}
Brown, R.  Groupoids and Van Kampen's theorem. {\em Proc. London Math. Soc.} {\bf 1967}, {\em 3} (17), 385-401.

\bibitem{Brown-elements}
Brown, R. {\em Elements of {M}odern {T}opology}; McGraw-Hill Book Co.: New York, 1968.

\bibitem{Br-adams}
Brown, R. Computing homotopy types using crossed $n$-cubes of  groups. In {\em Adams Memorial Symposium on Algebraic Topology, 1,  (Manchester, 1990)}. 
{\em London Math. Soc. Lecture Note Ser.} Volume {\em 175}; Cambridge Univ. Press: Cambridge, 1992;pp. 187--210.

\bibitem{Brown99}
Brown, R. Groupoids and crossed objects in algebraic topology.  \emph{Homology Homotopy Appl.} {\bf 1999}, \textbf{1}, 1--78 (electronic).

%book
\bibitem{Brown2k6}
Brown, R.  {\em Topology and Groupoids}; BookSurge, LLC: Charleston, SC, 2006.

\bibitem{Brown2k8}
Brown, R.  Exact sequences of fibrations of crossed complexes, homotopy classif\/ication of maps, and nonabelian extensions of groups. \textit{J. Homotopy Relational Struct.}
{\bf 2008}, {\em 3}, 331--342.

\bibitem{RB2k9}
Brown, R. Crossed complexes and higher homotopy groupoids as noncommutative tools for higher dimensional local-to-global problems. In {\em Handbook of Algebra}. Vol.{\em 6}, 83--124;
Michiel Hazewinkel, Editor; Elsevier:North-Holland: Amsterdam, 2009.  See also the previous lecture presented by R. Brown in {\em Categorical Structures for Descent, Galois Theory, Hopf algebras and semiabelian categories.}; Fields Institute, September 23-28, 2002. {\em Fields Institute Communications}  {\bf 2004}  {\em 43}, 101-130.

\bibitem{Brown-Gilbert89}
Brown, R.; and Gilbert, N. D. Algebraic models of {$3$}-types and automorphism structures   for crossed modules. \emph{Proc. London Math. Soc. (3)} {\bf 1989},{\em 59}~(1)  51--73.

\bibitem{Brown-etal2k7}
Brown, R.; Glazebrook, J.F.; Baianu, I.C.  A categorical and higher dimensional algebra framework for complex systems and spacetime structures. \emph{Axiomathes} {\bf 2008}, {\em 17}, 409--493.

\bibitem{Brown-etal2k2a}
Brown, R.; Hardie K.; Kamps, H.; Porter T.  The homotopy double groupoid of a Hausdorff space. {\em Theory Appl. Categ.} {\bf 2002}, {\em 10}, 71--93.

\bibitem{Brown-etal2k2}
Brown, R.;  Kamps, H.; Porter T.  A homotopy double groupoid of a Hausdorff space II: a van Kampen theorem.  {\em Theory and Applications of Categories} {\bf (2005)}, {\em 14}, 200-220.

\bibitem{BHi81a}
Brown, R.; and Higgins, P.J.  The algebra of cubes.  {\em J. Pure Appl. Alg.} {\bf 1981}, {\em 21}, 233--“260.

\bibitem{Brown-Higgins87}
Brown, R.; Higgins, P. J.  Tensor products and homotopies for $\omega$-groupoids and crossed complexes. \emph{J. Pure Appl. Algebra},(1987), {\bf 47},1--33.

\bibitem{Brown-etal2k9}
Brown, R.; Higgins, P. J.; and Sivera, R. {\em Nonabelian Algebraic Topology: filtered spaces, crossed complexes, cubical homotopy
groupoids}. \emph{European Math. Soc. Tracts Vol 15}: {\bf (2011)};available at: %%\url{http://www.bangor.ac.uk/r.brown/nonab-a-t.html}.

\bibitem{Brown-Janelidze2k4}
Brown, R.; Janelidze, G. Galois theory and a new homotopy double groupoid of a map of spaces. \textit{Appl. Categ. Structures} {\bf 2004},{\em 12}, 63--80. %%\href{http://arxiv.org/abs/math.AT/0208211}{math.AT/0208211}.

\bibitem{Brown-Janelidze87}
Brown, R.; Janelidze, G.  Van Kampen theorems for categories of covering morphisms in lextensive categories. \textit{J. Pure Appl. Algebra} {\bf 1997}, {\em119} , 255--263.

\bibitem{Brown-etal2k2}
Brown, R.;  Kamps, H.; Porter T.  A homotopy double groupoid of a Hausdorff space II: a van Kampen theorem. {\em Theory and Applications of Categories}, {\bf (2005)} {\em 14}  200-220.

\bibitem{Brown-Loday87a}
Brown, R.; Loday, J.-L. Van Kampen theorems for diagrams of spaces. {\em Topology} {\bf 1987}, {\em 26}, 311--335.

\bibitem{Brown-Loday87b}
Brown, R.; Loday, J.-L. Homotopical excision, and Hurewicz theorems for $n$-cubes of spaces. \emph{Proc. London Math. Soc. (3)} {\bf 1987},{\em 54},176--192..

\bibitem{Brown-Mosa86}
Brown, R.; Mosa G.H.  Double algebroids and crossed modules of algebroids. {\em Preprint}; University of Wales-Bangor, 1986.

\bibitem{Brown-Porter2k3}
Brown, R.; and Porter, T.  The intuitions of higher dimensional algebra for the study   of structured space.  \emph{Revue de Synth\`ese}  {\bf 2003}, {\em 124}, 174--203.

\bibitem{BRa84}
Brown, R.; and Razak~Salleh, A. A van Kampen theorem for unions of nonconnected spaces.{\em Arch. Math. (Basel)} {\bf 1984}, {\em 42}~(1), 85--88.

\bibitem{Brown-Spencer76a}
Brown, R.;  Spencer, C.B.  Double groupoids and crossed modules. \emph{Cahiers Top. G\'eom. Diff\'erentielle}  {\bf 1976}, {\em 17},
343--362.

\bibitem{Brown-Spencer76b}
Brown, R.;  Spencer, C.B. $G$-groupoids, crossed modules and the fundamental groupoid of a topological group. {\em Indag. Math.} {\bf
1976},{\em 38}  296--302.

\bibitem{BRM2k3}
 Buneci, M. R. \emph{Groupoid Representations.} (ro only : orig. title ``{\em Reprezentari de Grupoizi}'');  Ed. Mirton: Timishoara , 2003.

\bibitem{Buneci2k2a}
Buneci, M. R.  The structure of the Haar systems on locally compact groupoids, \textit{Math. Phys. Electron. J.}  {\bf 2002}, {\em 8}, Paper~4, 11~pages.

\bibitem{Buneci2k2b}
Buneci, M.R. Consequences of the Hahn structure theorem for Haar measure, \textit{Math. Rep. (Bucur.)}  {\bf 2002}, {\em 4(54)}, 321--334.

\bibitem{Buneci2k3a}
Buneci, M. R.  Amenable unitary representations of measured groupoids, \textit{Rev. Roumaine Math. Pures Appl.}  {\bf 2003}, {\em 48}, 129--133.

\bibitem{Buneci2k3b}
Buneci, M. R. Invariant means on groupoids, \textit{Rev. Roumaine Math. Pures Appl.}  {\bf 2003}, {\em 48},13--29.

\bibitem{Buneci2k3c}
Buneci, M. R. Groupoid algebras for transitive groupoids, \textit{Math. Rep. (Bucur.)}  {\bf 2003}, {\em 5(55)}, 9--26.

\bibitem{Buneci2k3d}
Buneci, M. R. Haar systems and homomorphism on groupoids., In  {\em Operator Algebras and \\
Mathematical Physics  (Constantza, 2001)}; Editors: J.-M.; Combes, J.; Cuntz, G.; Elliott, Gh.; \\
 Nenciu, H.; Siedentop, R. ; and  S.~Stratila, S; Theta: Bucharest, 2003;pp. 35--49.

\bibitem{Buneci2k4}
Buneci, M. R. Necessary and sufficient conditions for the continuity of a pre-Haar system at a unit with singleton orbit. {\em Quasigroups and Related Systems}  {\bf 2004}, {\em 12}, 29--38.

\bibitem{Buneci2k5a}
Buneci, M. R. Isomorphic groupoid $C^*$-algebras associated with different Haar systems. \textit{New York J. Math.}   {\bf 2005},{\em 11}, 225--245, %%\href{http://arxiv.org/abs/math.OA/0504024}{math.OA/0504024}.

\bibitem{Buneci2k5b}
Buneci, M. R. The equality of the reduced and the full $C^*$-algebras and the amenability of a topological groupoid.  In {\em Recent Advances in Operator Theory, Operator Algebras, and Their Applications}, \textit{Oper. Theory Adv. Appl.}, Vol.{\em 153}, Birkh\"auser: Basel, 2005;p.61--78.

\bibitem{Buneci2k6}
Buneci, M. R.  Groupoid $C^*$-algebras. \textit{Surveys in Math. and its Appl.}   {\bf 2006}, {\em 1}, 71--98.

\bibitem{BuneciPre2k9}  
Buneci, M. R. A Review of Groupoid $C^*$-algebras and Groupoid Representations. Preprint, 2009;\\ 

%%{\em http://www.utgjiu.ro/math/mbuneci/preprint/p0024.pdf}. 

\bibitem{Buneci2010}
Buneci, M. R. Representations of Double Groupoids.  Preprint, 2010;({\em personal communication from M.R. Buneci}).

\bibitem{Butterfield-Isham2k}
Butterfield, J.; Isham, C.J.  A topos perspective on the Kochen--Specker theorem. I.~Quantum states as generalized valuations. \textit{Internat. J. Theoret. Phys.}
{\bf 1998},{\em 37},  2669--2733,

%% \href{http://arxiv.org/abs/quant-ph/9803055}{quant-ph/9803055}.\\

Butterfield, J.; Isham, C.J.  A topos perspective on the Kochen--Specker theorem. II.~Conceptual aspects and classical analogues' \textit{Internat. J. Theoret. Phys.}
{\bf 1999},{\em 38}, 827--859, 

%%\href{http://arxiv.org/abs/quant-ph/9808067}{quant-ph/9808067}.\\
Butterfield, J.; Isham, C.J.  A topos perspective on the Kochen--Specker theorem. III.~Von Neumann algebras as the base category' \textit{Internat. J. Theoret. Phys.}
{\bf 2000},{\em 39}, 1413--1436. 

%%\href{http://arxiv.org/abs/quant-ph/9911020}{quant-ph/9911020}.\\
Butterfield, J.; Isham, C.J., A topos perspective on the Kochen--Specker theorem. IV.~Interval valuations. \textit{Internat. J. Theoret. Phys.} {\bf 2002},{\em 41}, 613--639. 

%%\href{http://arxiv.org/abs/quant-ph/0107123}{quant-ph/0107123}.

\bibitem{Byczuk-Hofstetter-Vollhardt2k8}
Byczuk, K.; Hofstetter, W.; Vollhardt, D.  Mott--Hubbard transition vs.\ Anderson localization of correlated, disordered electrons. \textit{Phys. Rev. Lett.} {\bf 2005},{\em 94}, 401--404. 
%%\href{http://arxiv.org/abs/cond-mat/0403765}{cond-mat/0403765}.

\bibitem{Carter-Saito94}
Carter, J.S.; Saito, M.  Knotted surfaces, braid moves, and beyond. In {\em Knots and Quantum Gravity}; Riverside, CA, 1993; \textit{Oxford Lecture Ser. Math. Appl.}, Vol.{\em 1}; Oxford Univ. Press: New York, 1994;pp. 191--229.

\bibitem{Carter-F-S95}
Carter, J.S.;Flath, D.E.; Saito, M. {\em The Classical and Quantum $6j$-symbols}. {\em Mathematical Notes 43}; Princeton University Press: Princeton, NJ, 1995;pp.164.

\bibitem{Chaician-Demichev96}
Chaician, M.;  Demichev, A. {\em  Introduction to quantum groups}.; World Scientific Publishing Co., Inc.: River Edge, NJ, 1996.

\bibitem{Chari-Pressley94}
Chari, V.; Pressley, A. \emph{A Guide to Quantum Groups}; Cambridge University Press: Cambridge, UK, 1994. ISBN 0-521-55884-0.

\bibitem{Chamon2k5}
Chamon, C.  Quantum glassiness in strongly correlated clean systems: an example of topological overprotection. \emph{Phys. Rev. Lett.} {\bf 2005},{\bf 94}, 
398--402, 4~pages.  

%%\href{http://arxiv.org/abs/cond-mat/0404182}{cond-mat/0404182}.

\bibitem{Chowd}
Chowdury, A. E.; Choudury, A. G.; \emph{Quantum Integrable Systems}, \emph{Chapman and Hall Research Notes in Math.},  {\bf 435}; CRC Press: London-New York, 2004.

\bibitem{Coleman-DeLuccia80}
Coleman, P.; De Luccia, F.  Gravitational effects on and of vacuum decay.  \emph{Phys. Rev. D} {\bf 1980},{\em 21}, 3305--3309.

\bibitem{Connes79}
Connes, A.  Sur la th\'eorie non-commutative de l'int\'egration, in Alg\`ebres d'op\'erateurs. \emph{S\'em., Les Plans-sur-Bex} {\bf 1978}, \emph{Lecture Notes in Math.}, Vol.{\bf 725}; Springer:Berlin, 1979; pp.19--143.

\bibitem{Connes94}
Connes, A.  \emph{ Noncommutative geometry}; Academic Press, Inc.: San Diego, CA, 1994.

\bibitem{Crane-Frenkel94}
Crane, L.; Frenkel, I.B. Four-dimensional topological quantum field theory. Hopf categories, and the canonical bases. Topology and physics. \emph{J. Math. Phys.}
(1994),{\bf 35}, 5136--5154.

%% \href{http://arxiv.org/abs/hep-th/9405183}{hep-th/9405183}.

\bibitem{Crane-Kauffman-Yetter97}
Crane, L.; Kauffman, L.H.;  Yetter, D.N.  State-sum invariants of 4-manifolds. \textit{J. Knot Theory Ramifications}{\bf 1997},{\em 6},177--234. 
%%\href{http://arxiv.org/abs/hep-th/9409167}{hep-th/9409167}.

\bibitem{Cr59}
Crowell, R.~H. On the van {K}ampen theorem. {\em Pacific J. Math.}  {\bf 1959}, {\em 9},43--50.

\bibitem{Day-Street97}
Day, B.J.; Street, R.  Monoidal bicategories and Hopf algebroids. \textit{Adv. Math.}{\bf 1997},{\em 129}, 99--157.

\bibitem{Day}
Day, B; and Strret, R. Quantum categories, star autonomy, and quantum groupoids, Galois theory, Hopf algebras, and semiabelain categories, \emph{Fields Inst. Commun., 43}, {\em Amer. Math. Soc.}; Providence, RI, 2004; pp. 187--225.

\bibitem{Deligne2k2}
Deligne, P.  Cat\'egories tensorielles \textit{Moscow Math. J.}  {\bf 2002}, {\em 2}, 227--248.

\bibitem{Deligne-Morgan99}
Deligne, P.; Morgan, J.W.  Notes on supersymmetry (following Joseph Berenstein).  In  {\em Quantum Fields and Strings: a Course for Mathematicians}; Princeton, NJ, 1996/1997); Editors J.W.~Morgan et al., Vols. {\em 1, 2},{\em Amer. Math. Soc.}: Providence RI, 1999;pp. 41--97.

\bibitem{Dennis-etal2k2}
Dennis, E.; Kitaev, A.; Landahl, A.; Preskill J. Topological quantum memory. \textit{J. Math. Phys.} {\bf 2002},{\em 43}, 4452--4459, %%\href{http://arxiv.org/abs/quant-ph/0110143}{quant-ph/0110143}.

\bibitem{Dixmier69}
Dixmier, J.  Les $C^*$-alg\`ebras et leurs repr\'esentations. Deuxi\`eme \`edition, {\it Cahiers Scientifiques}, Fasc.{\em XXIX}; Gauthier-Villars \'Editeur: Paris, 1969.

\bibitem{Doebner-Hennig89}
Doebner, H.-D.; Hennig, J. D.; Editors. \textit{Quantum groups}. In {\em Proceedings of the 8th International Workshop on Mathematical Physics}; Arnold Sommerfeld Institute: Clausthal; Springer-Verlag Berlin, 1989, \textit{ISBN 3-540-53503-9}.

\bibitem{Drechsler-Tuckey96}
Drechsler, W.;Tuckey, P.A.  On quantum and parallel transport in a Hilbert bundle over spacetime. \textit{Classical  Quantum Gravity} {\bf 1996},{\em 13}, 611--632.  

%%\href{http://arxiv.org/abs/gr-qc/9506039}{gr-qc/9506039}.

\bibitem{Drinfeld87}
Drinfel'd, V.G. Quantum groups. Iin  {\em Proceedings Intl. Cong. of Mathematicians}, Vols. {\em 1, 2}; Berkeley, Calif., 1986; Editor A. Gleason. {\em  Amer. Math. Soc.}: Providence, RI, 1987; pp.798--820.

\bibitem{Drinfeld92}
Drinfel'd, V.G.  Structure of quasi-triangular quasi-Hopf algebras, \textit{Funct. Anal. Appl.} {\bf 1992},{\em 26}, 63--65.

\bibitem{Dupre1}
Dupr\'{e}, M. J.; \emph{The Classification and Structure of $C^*$-algebra bundles}. Mem. Amer. Math. Soc. {\bf 1979} {\em 21}, 222; Amer. Math. Soc.: Providence, RI.

\bibitem{Ellis88}
Ellis, G.J.  Higher dimensional crossed modules of algebras. \emph{J. Pure Appl. Algebra} {\bf 1988},{\em 52}, 277--282.

\bibitem{ESt}
Ellis, G.~J. and Steiner, R. Higher-dimensional crossed modules and the homotopy groups  of $(n+1)$-ads.
\newblock {\em J. Pure Appl. Algebra} {\bf 1987} {\em 46}~(2-3), 117--136.

\bibitem{Etingof-Ostrik2k3}
Etingof, P. I.; Ostrik, V.  Finite tensor categories. Preprint {\bf2003}, 26 pages, arXiv:math/0301027v2 [math.QA]. 

%%\href{http://arxiv.org/abs/math.QA/0301027}{math.QA/0301027}.

\bibitem{Etingof-Varchenko98}
Etingof, P. I.; Varchenko, A.N.  Solutions of the Quantum Dynamical Yang-Baxter Equation and Dynamical Quantum Groups. \emph{Commun. Math. Phys.}
\textbf{1998} \emph{196}, 591--640.

\bibitem{Etingof-Varchenko99}
Etingof, P.I.; Varchenko, A. N.  Exchange dynamical quantum groups.  \emph{Commun. Math. Phys.} \textbf{1999}, \emph{205} (1), 19--52.

\bibitem{Etingof-Schiffmann2k1}
Etingof, P.I.; Schiffmann, O.  Lectures on the dynamical Yang--Baxter equations.  In {\em Quantum Groups and Lie Theory}; Durham, 1999. {\it London Math. Soc. Lecture Note Ser.}, {\em 290}; Cambridge Univ. Press: Cambridge, 2001;pp. 89--129. 
%%\href{http://arxiv.org/abs/math.QA/9908064}{math.QA/9908064}.

\bibitem{Evans}
Evans, D. E.; Kawahigashi, Y.  {\em Quantum symmetries on operator algebras};The Clarendon Press; Oxford University Press: New York, 1998.

\bibitem{Eymard64}
Eymard, P.  L'alg\`ebre de Fourier d'un groupe localement compact. {\em Bull. Soc. Math. France} {\bf 1964} {\em 92},181--236.

\bibitem{martins-picken-cub}
Faria~Martins, J.; and Picken, R.  A cubical set approach to{\em 2}-bundles with connection and  Wilson surfaces. {\bf 2010}, {\em arXiv/CT} {\em 0808.3964v3}, 1--59.

\bibitem{Fauser2k2}
Fauser, B.  A treatise on quantum Clifford algebras.; Konstanz, Habilitationsschrift. \\ 
%%\href{http://arxiv.org/abs/math.QA/0202059}{math.QA/0202059}.

\bibitem{Fauser2k4}
Fauser, B.  Grade free product formulae from Grassman--Hopf gebras.  In  {\em Clifford Algebras: Applications to Mathematics, Physics and Engineering};
Editor R. Ablamowicz. \emph{Prog. Math. Phys.}, {\em 34}; Birkh\"{a}user: Boston, MA, 2004;pp.279--303; 

%%\href{http://arxiv.org/abs/math-ph/0208018}{math-ph/0208018}.

\bibitem{Fell60}
Fell, J.M.G. The dual spaces of  $C^*$-algebras. \emph{Trans. Amer. Math. Soc.} {\bf 1960},{\em 94}, 365--403.

\bibitem{Fernandez-Castro96}
Fernandez, F.M.; Castro, E.A. Lie  algebraic methods in quantum chemistry and physics. {\it Mathematical Chemistry Series}, CRC Press, Boca Raton, FL, 1996.

\bibitem{Feynman48}
Feynman, R.P.  Space-time approach to non-relativistic quantum mechanics.  \emph{Rev. Modern Phys.} (1948), {\bf 20}, 367--387.

\bibitem{Fradkin-H-S78}
Fradkin, E.; Huberman, S.; Shenker, S. Gauge symmetries in random magnetic systems. {\em Phys. Rev. B} {\bf1978}, {\em 18}, 4783-4794. 
 http://prh.aps.org/abstract/PRB/v18/i9/p4879-1 .


\bibitem{Freedman-etal2k3}
Freedman, M.H; Kitaev, A.; Larsen, M.J.; Wang Z. Topological quantum computation.  \emph{Bull. Amer. Math. Soc. (N.S.)} {\bf 2003},{\bf 40}, 31--38.
%%\href{http://arxiv.org/abs/quant-ph/0101025}{quant-ph/0101025}.

\bibitem{Frohlich61}
Fr{\"o}hlich, A.  Non-Abelian homological algebra. I.~Derived functors and satellites, \emph{Proc. London Math. Soc.~(3)}{\bf 1961},{\em 11}, 239--275.

\bibitem{Gelfand-Naimark43}
Gel'fand, I.;  Neumark, M. On the Imbedding of Normed Rings into the Ring of Operators in Hilbert Space. {\it Rec. Math. [Mat. Sbornik] N.S.} {\bf 1943},{\em  12 (54)},197--213.
Reprinted in {\em $C^*$-algebras} {\bf1943--1993}, \emph{Contemp. Math.} {\bf 1994},{\em 167}; American Mathematical Society: Providence RI.)

\bibitem{Georgescu2k6}
Georgescu, G.  $N$-valued logics and \L ukasiewicz--Moisil algebras. \textit{Axiomathes}{\bf 2006},{\em 16}, 123--136.

\bibitem{Georgescu-Popescu68}
Georgescu, G.; Popescu, D.  On algebraic categories, \textit{Rev. Roumaine Math. Pures Appl.}{\bf 1968},{\em 13}, 337--342.

\bibitem{Gilmore2k5}
Gilmore, R. \emph{Lie groups, Lie algebras and some of their applications.}; Dover Publs. Inc.: Mineola~-- New York, 2005.

\bibitem{Girelli-P-P2k8}
Girelli, F.; Pfeiffer, H.; Popescu, E. M. Topological higher gauge theory: from $BF$ to $BFCG$ theory. (English summary) J. Math. Phys. {\bf 2008} {\bf 49} (2008), no. 3,, 17 pp. 032503.

\bibitem{Goldblatt84}
Goldblatt, R.  \emph{Topoi: The Categorial Analysis of Logic}; Dover Books on Mathematics; North-Holland,1984;pp.568

\bibitem{Grabowski-Marmo2k1}
Grabowski, J.; Marmo, G. Generalized Lie bialgebroids and Jacobi structures. \emph{J. Geom Phys.} {\bf 2001}, {\bf 40}, 176--199.

\bibitem{Grandis-Mauri2k3}
Grandis, M.; Mauri, L. Cubical sets and their site, \emph{Theory Appl. Categ.} {\bf 2003},{\em 11}, no.~8, 185--211.

\bibitem{Grisaru-Pendleton77}
Grisaru, M.T.; Pendleton H.N.  Some properties of scattering amplitudes in supersymmetric theories. \emph{Nuclear Phys. B} {\bf 1977},{\em1246}, 81--92.

\bibitem{Alex57}
Grothendieck, A. Sur quelque point d-alg$\`{e}$bre homologique. \emph{T$\^o$hoku Math. J. (2)}{\bf 1957},{\em 9}, 119--121.

\bibitem{Grothendieck61}
Grothendieck, A. Technique de descente et th$\'e$or$\`$emes d'existence en g$\'e$om$\'e$trie alg$\'e$brique. II. {\em  S$\'e$minaire Bourbaki} {\bf 1959-1960},{\em 12},exp.~195; Secr\'etariat Math\'ematique; Paris, 1961.

\bibitem{Grothendieck62}
Grothendieck, A. Technique de construction en g$\'e$om$\'e$trie alg$\'e$brique IV. Formalisme g$\'e$n$\'e$ral des foncteurs repr$\'e$sentables. {\em  S$\'$eminaire H. Cartan}{\bf 1960-61},{\em 13}, exp.~11; Secr$\'{e}$tariat Math$\'e$matique: Paris, 1962.

\bibitem{Grothendieck71}
Grothendieck, A.  {\em Rev$\^{e}$tements $\'E$tales et Groupe Fondamental (SGA1)}, chapter VI: Cat$\'e$gories fibr$\'e$es et descente, \emph{Lecture Notes in Math.},  Vol.~{\em 224}; Springer-Verlag:  Berlin, 1971.

\bibitem{Grothendieck-Dieudone60}
Grothendieck, A.; Dieudonn\'{e}, J. {\em El$\'{e}$ments de geometrie alg$\`{e}$brique}.  \emph{Publ. Inst. des Hautes $\'E$tudes de Science}{\bf 1960},{\em 4}, 5--365.  Grothendieck,  A.; Dieudonn$\'{e}$, J.  $\'E$tude cohomologique des faisceaux coherents. {\it  Publ. Inst. des Hautes \'Etudes de Science} {\bf 1961},{\em 11}, 5--167.

\bibitem{Gu-Wen2k6}
Gu, Z.-G.; Wen, X.-G.  A lattice bosonic model as a quantum theory of gravity.  %%\href{http://arxiv.org/abs/gr-qc/0606100}{gr-qc/0606100}.

\bibitem{Hahn78a}
Hahn, P.  Haar measure for measure groupoids.  \textit{Trans. Amer. Math. Soc.}{\bf 1978},{\em 242}, 1--33.

\bibitem{Hahn78b}
Hahn, P.  The regular representations of measure groupoids. {\em Trans. Amer. Math. Soc.} {\bf 1978},{\em 242}, 34--72.

\bibitem{Harrison2k5}
Harrison, B.K.  The differential form method for finding symmetries. \emph{SIGMA} {\bf 2005}, {\bf 1}, 001, 12~pages. %%\href{http://arxiv.org/abs/math-ph/0510068}{math-ph/0510068}.

\bibitem{Hazewinkel2k6}
 Hazewinkel, A. (Editor). {\em Handbook of algebra.} Vol.~{\em 4}; Elsevier: St. Louis, Chatswood~-- Singapore, 2006.

\bibitem{Heynman-Lifschitz58}
Heynman, R.; Lifschitz S. Lie groups and Lie algebras.;  Nelson Press: New York~-- London, 1958.

\bibitem{Heunen-etal2k8}
Heunen, C.; Landsman, N. P.; and Spitters, B. A topos for algebraic quantum theory. {\bf 2008}; 12 pages. \emph{arXiv:0709.4364v2 [quant--ph]}.

\bibitem{Hindeleh-Hosemann88}
Hindeleh, A.M.; Hosemann, R.  Paracrystals representing the physical state of matter.  \emph{Solid State Phys.}(1988), {\bf 21}, 4155--4170.

\bibitem{Hosemann-Bagchi62}
Hosemann, R.; and Bagchi, R.N. \emph{Direct analysis of diffraction by matter.}; North-Holland Publs.: Amsterdam~-- New York, 1962.

\bibitem{Hosemann-etal81}
Hosemann, R.; Vogel W.;  Weick, D.; Balta-Calleja, F.J.  Novel aspects of the real paracrystal.  \emph{Acta Cryst.~A} {\bf 1981}, {\bf 376}, 85--91.

\bibitem{ThVetal69}
Ionescu, Th.V.; Parvan, R.; and Baianu, I.  Les oscillations ioniques dans les cathodes creuses dans un champ magnetique. \emph{C.R.Acad.Sci.Paris}. {\bf 1969}, {\bf 270}, 1321--1324; (\emph{paper communicated by Nobel laureate Louis Ne\'el}).

\bibitem{Butterfield-Isham2k4}
Isham, C.J.; Butterfield, J.  Some possible roles for topos theory in quantum theory and quantum gravity. {\em Found. Phys.} {\bf 2000},{\em 30}, 1707--1735. %%\href{http://arxiv.org/abs/gr-qc/9910005}{gr-qc/9910005}.

\bibitem{Janelidze91}
Janelidze, G.  Precategories and Galois theory. In \emph{Category theory}; Como, 1990. {\em Springer Lecture Notes in Math.}, Vol.~\emph{1488}; Springer: Berlin, 1991;pp. 157--173.

\bibitem{Janelidze90}
Janelidze, G. Pure Galois theory in categories. \emph{J. Algebra}(1990), {\bf 132}, 270--286.

\bibitem{Jimbo85}
Jimbo, M.  A $q$-difference analogue of $U_g$ and the Yang--Baxter equations. \emph{Lett. Math. Phys.} (1985), {\bf 10},63--69. 

\bibitem{Jones83}
Jones, V. F. R. Index for Subfactors. \emph{Invetiones Math.} {\bf 1989} {\bf  72},1--25. Reprinted in: \emph{New Developments in the Theory of Knots.}; World Scientific Publishing: Singapore,1989.

\bibitem{Andre-Street93}
Joyal, A.; Street R.  Braided tensor categories. \emph{Adv. Math.} (1995), \emph{102}, 20--78.

\bibitem{Kac77}
Kac, V.  Lie superalgebras. \emph{Adv. Math.} {\bf 1977}, {\bf 26}, 8--96.

\bibitem{Kamps-Porter99}
Kamps, K. H.; Porter, T.  A homotopy 2--groupoid from a fibration. \emph{Homotopy, Homology and Applications.} (1999), {\bf 1}, 79--93.

\bibitem{Kapitsa78}
Kapitsa,  P. L.  Plasma and the controlled thermonuclear reaction. \emph{The Nobel Prize lecture in Physics 1978}, in \emph{Nobel Lectures, Physics 1971--1980}; Editor S.~Lundqvist; World Scientific Publishing Co.: Singapore, 1992.

\bibitem{Kauffman89}
Kauffman, L. Spin Networks and the Jones Polynomial. \emph{Twistor Newsletter, No.29}; Mathematics Institute: Oxford, November 8th, 1989.

\bibitem{Kauffman91}
Kauffman, L.  $SL(2)_q$--Spin Networks. \emph{Twistor Newsletter, No.32 }; Mathematics Institute: Oxford, March 12th, 1989.

\bibitem{Khoroshkin-Tolstoy91}
Khoroshkin, S.M.; Tolstoy, V.N. Universal $R$-matrix for quantum supergroups.  In {\em Group Theoretical Methods in Physics}; Moscow, 1990; {\em Lecture Notes in Phys.}, Vol.~382; Springer: Berlin, 1991;pp. 229--232.

\bibitem{Kirilov-R89}
Kirilov, A.N.; and Reshetikhin, N. Yu. Representations of the Algebra $U_q(sl(2))$, $q$-Orthogonal Polynomials and Invariants of Links. Reprinted in: {\em New Developments in the Theory of Knots.}; Kohno, Editor; World Scientific Publishing, 1989.

\bibitem{KlimykSch97}
Klimyk, A. U.; and Schmadgen, K. {\em  Quantum Groups and Their Representations.}; Springer--Verlag: Berlin, 1997.

\bibitem{Korepin}
Korepin, V. E.  Completely integrable models in quasicrystals. {\em Commun. Math. Phys.}  {\bf 1987}, {\em 110}, 157--171.

\bibitem{Kustermans-Vaes2k}
Kustermans, J.; Vaes, S. The operator algebra approach to quantum groups. {\em Proc. Natl. Acad. Sci. USA} {\bf 2000}, {\em 97}, 547--552.

\bibitem{Lambe-Redford97}
Lambe, L.A.; Radford, D.E.  Introduction to the quantum Yang--Baxter equation and quantum groups: an algebraic approach. \emph{Mathematics and its Applications}, Vol.~{\em 423}; Kluwer Academic Publishers: Dordrecht, 1997.

\bibitem{Lance95}
Lance, E.C.  Hilbert $C^*$-modules. A toolkit for operator algebraists.  {\em London Mathematical Society Lecture Note Series}, Vol.~{\em 210}; Cambridge University Press: Cambridge, 1995.

\bibitem{Landsman98}
Landsman, N.P.  Mathematical topics between classical and quantum mechanics. {\em Springer Monographs in Mathematics}; Springer-Verlag: New York, 1998.

\bibitem{Landsman2k}
Landsman, N.P.  Compact quantum groupoids, in  Quantum Theory and Symmetries (Goslar,  July 18--22, 1999), Editors H.-D.~Doebner et al., World Sci. Publ., River Edge, NJ, 2000, 421--431, 
%%\href{http://arxiv.org/abs/math-ph/9912006}{math-ph/9912006}.

\bibitem{Landsman-Ramazan2k1}
Landsman, N.P.; Ramazan, B. Quantization of Poisson algebras associated to Lie algebroids, in {\em Proc. Conf. on Groupoids in Physics, Analysis and Geometry: Boulder CO, 1999}; Editors: J.~Kaminker et al., \emph{Contemp. Math.}{\bf 2001},{\em 282}, 159--192.  
%%\href{http://arxiv.org/abs/math-ph/0001005}{math-ph/0001005}.

\bibitem{Lawrence95}
Lawrence, R.L. Algebra and Triangle Relations. In: {\em Topological and Geometric Methods in Field Theory}; Editors: J. Michelsson and O.Pekonen; World Scientific Publishing,1992;pp.429-447. {\em J.Pure Appl. Alg.} {\bf 1995}, {\em 100}, 245-251.

\bibitem{Lee-etal2k4}
Lee, P.A.; Nagaosa, N.; Wen, X.-G.  Doping a Mott insulator: physics of high temperature superconductivity.{\bf 2004}. 
%%\href{http://arxiv.org/abs/cond-mat/0410445}{cond-mat/0410445}.

\bibitem{Levin-Olshanetsky08}
Levin, A.; Olshanetsky, M.  Hamiltonian Algebroids and deformations of complex structures on Riemann curves. {\bf 2008}, hep--th/0301078v1.

\bibitem{Levin-Wen2k3}
Levin, M.; Wen, X.-G.  Fermions, strings, and gauge fields in lattice spin models.  {\em Phys. Rev. B} {\bf 2003}, {\em 67}, 245316, 4~pages. %%\href{http://arxiv.org/abs/cond-mat/0302460}{cond-mat/0302460}.

\bibitem{Levin-Wen2k6a}
Levin, M.; Wen, X.-G.  Detecting topological order in a ground state wave function. {\em Phys. Rev. Lett.} {\bf 2006}, {\em 96}, 110405, 4~pages. %%\href{http://arxiv.org/abs/cond-mat/0510613}{cond-mat/0510613}.

\bibitem{Levin-Wen2k5}
Levin M.; Wen, X.-G. Colloquium: photons and electrons as emergent phenomena. {\em Rev. Modern Phys.} {\bf 2005}, {\em 77}, 871--879. %%\href{http://arxiv.org/abs/cond-mat/0407140}{cond-mat/0407140}.

\bibitem{Levin-Wen2k6b}
Levin, M.; Wen, X.-G. Quantum ether: photons and electrons from a rotor model. {\em Phys. Rev. B} {\bf 2006}, {\em 73}, 035122, 10~pages. %%\href{http://arxiv.org/abs/hep-th/0507118}{hep-th/0507118}.

\bibitem{Loday82}
Loday, J.L. Spaces with finitely many homotopy groups. {\em J. Pure Appl. Algebra} {\bf 1982},{\em 24}, 179--201.

\bibitem{Lu96}
Lu, J.-H.  Hopf algebroids and quantum groupoids. {\em Internat. J. Math.} {\bf 1996}, {\em 7}, 47--70. 
%%\href{http://arxiv.org/abs/q-alg/9505024}{q-alg/9505024}.

\bibitem{Mack-Schomerus92}
Mack, G.; Schomerus, V.  Quasi Hopf quantum symmetry in quantum theory. {\em Nuclear Phys. B} {\bf 1992}, {\em 370}, 185--230.

\bibitem{Mackaay-Picken}
Mackaay, M.; Picken, P.  State-sum models, gerbes and holonomy.  In {\em Proceedings of the XII Fall Workshop on Geometry and Physics}, {\em Publ. R. Soc. Mat. Esp.},  Vol.~{\em 7}: Madrid, 2004;pp. 217--221.

\bibitem{Mackaay99}
Mackaay, M.  Spherical 2-categories and 4-manifold invariants. \emph{Adv. Math.} {\bf 1999},{\em 143}, 288--348. 
%%\href{http://arxiv.org/abs/math.QA/9805030}{math.QA/9805030}.

\bibitem{Mackenzie2k5}
Mackenzie, K. C. H.  General theory of Lie groupoids and Lie algebroids. {\em London Mathematical Society Lecture Note Series}, Vol.~{\em 213}; Cambridge University Press: Cambridge, 2005.

\bibitem{Mackey92}
Mackey, G.W.  The scope and history of commutative and noncommutative harmonic analysis. {\em History of Mathematics}, Vol.{\em 5}; American Mathematical Society: Providence, RI; London Mathematical Society: London, 1992.

\bibitem{MacLane65}
Mac Lane, S. Categorical algebra. \emph{Bull. Amer. Math. Soc.} (1965), 40--106.

\bibitem{MacLane-Moerdijk92}
Mac Lane, S.; Moerdijk, I.  \emph{Sheaves in geometry and logic. A first introduction to topos theory.}; Springer Verlag: New York, 1994.

\bibitem{Majid95}
Majid,S. \emph{Foundations of Quantum Group Theory.};Cambridge Univ. Press: Cambridge UK, 1995.

\bibitem{Maltsiniotis92}
Maltsiniotis, G. Groupo\"{\i}des quantiques. {\em C. R. Acad. Sci. Paris S\'er. I Math.} {\bf 1927}, {\em 314}, 249--252.

\bibitem{Manoj}
Manojlovi\'{c}, N.; Sambleten, H.  Schlesinger transformations and quantum R-matrices. \emph{Commun. Math. Phys.} {\bf 2002}, {\em 230}, 517--537.

\bibitem{Miranda2k3}
Miranda, E. Introduction to bosonization. {\em Brazil. J. Phys.} {\bf 2003}, {\em 33}, 1--33. (http://www.ifi.unicamp.br/~emiranda/papers/bos2.pdf).

\bibitem{Mitchell65}
Mitchell, B. {\em The theory of categories.} {\em Pure and Applied Mathematics}, Vol.{\em  17}; Academic Press: New York and London, 1965.

\bibitem{Mitchell72}
Mitchell, B. Rings with several objects. {\em Adv. Math.} {\bf 1972}, {\em 8}, 1--161.

\bibitem{Mitchell85}
Mitchell, B. Separable algebroids. {\em Mem. Amer. Math. Soc.}, {\bf 1985}, {\em 57}, no.333.

\bibitem{Moerdijk-Svensson93}
Moerdijk, I.; Svensson, J.A. Algebraic classification of equivariant 2-types. {\em J. Pure Appl. Algebra} {\bf 1993}, {\em 89}, 187--216.

\bibitem{Mosa86}
Mosa, G.H. \emph{Higher dimensional algebroids and Crossed complexes.} PhD Thesis; University of Wales: Bangor, 1986.

\bibitem{Mott77}
Mott, N.F. \emph{Electrons in glass}, Nobel Lecture, 11 pages, December 8, 1977.

\bibitem{Mott78}
Mott, N.F.  Electrons in solids with partial disorder. \emph{Lecture presented at The Cavendish Laboratory}: Cambridge, UK, 1978.

\bibitem{Mott--Davis78}
Mott, N.F.; Davis, E.A. {Electronic processes in non-crystalline materials}; Oxford University Press: Oxford, 1971; 2nd ed., 1978.

\bibitem{Mott--etal75}
Mott, N.F.; et al. The Anderson transition. \emph{Proc. Royal Soc. London Ser. A} (1975), {\bf 345}, 169--178.

\bibitem{Moultaka-etal2k5}
Moultaka, G.; Rausch, M. de Traubenberg; T\u{a}nas\u{a}, A.  Cubic supersymmetry and abelian gauge invariance.
 {\em Internat. J. Modern Phys. A} {\bf 2005}, {\em 20}, 5779--5806. %%\href{http://arxiv.org/abs/hep-th/0411198}{hep-th/0411198}.

\bibitem{Mrcun2k2}
Mr\v{c}un, J.  On spectral representation of coalgebras and Hopf algebroids. %%\href{http://arxiv.org/abs/math.QA/0208199}{math.QA/0208199}.

%%Mr\v{c}un, J.  Sheaf coalgebras and duality. {Topology Appl.} {\bf 2007}, {\em 154}, 2795--2812.
%%Mr\v{c}un J., On duality between etale groupoids and Hopf algebroids. {\em J. Pure Appl. Algebra} {\bf 2007}, {\em 210}, 267--282.

\bibitem{Mrcun07}
Mr\v{c}un, J.  On the duality between \'{e}tale groupoids and Hopf algebroids. \emph{J. Pure Appl. Algebra} (2007), {\bf 210}, 267--282.

\bibitem{MRW8717}
Muhly, P.; Renault, J.; Williams, D. Equivalence and isomorphism for groupoid $C^*$-algebras. {\em J. Operator Theory}, {\bf 1987}, {\em 6}, 3--22.

\bibitem{Meusberger}
Meusberger, C.  Quantum double and $\kappa$-Poincar\'{e} symmetries in $(2+1)$-gravity and Chern-Simons theory. {\em Can. J. Phys.} \textbf{2009}, {\em 87}, 245--250.

\bibitem{Nayak-etal2k7}
Nayak, C.; Simon, S.H.; Stern, A.; Freedman, M.; Das Sarma, S.  Non-Abelian anyons and topological quantum computation. {\it Rev. Modern Phys.} {\bf 2008}, {\em 80},1083, 77~pages. 
%%\href{http://arxiv.org/abs/0707.1889}{arXiv:0707.1889}.

\bibitem{Nikshych-Vainerman2k}
Nikshych, D.A.; Vainerman, L.  A characterization of depth 2 subfactors of ${\rm II}_1$ factors. {\em J. Funct. Anal.} {\bf 2000}, {\em 171}, 278--307. %%\href{http://arxiv.org/abs/math.QA/9810028}{math.QA/9810028}.

\bibitem{Nishimura96}
Nishimura, H. Logical quantization of topos theory. {\em Internat. J. Theoret. Phys.} {\bf 1996}, {\em 35}, 2555--2596.

\bibitem{Neuchl97}
Neuchl, M. {\em Representation theory of Hopf categories}. PhD Thesis: University of M\"unich, 1997.

\bibitem{Noether1918}
Noether, E. Invariante Variationsprobleme. {\em Nachr. D. Konig. Gesellsch. D. Wiss. Zu Gottingen, Math-phys. Klasse} {\bf 1918}, 235--“257.

\bibitem{Norrie90}
Norrie, K. Actions and automorphisms of crossed modules. {\em Bull. Soc. Math. France} {\bf 1990}, {\em 118}~(2), 129--146.

\bibitem{Ocneanu88}
Ocneanu, A. Quantized groups, string algebras and Galois theory for algebras. In {\em Operator algebras and applications.}, {\em 2} \textit{London Math. Soc. Lecture Note Ser., 136}; Cambridge Univ. Press: Cambridge,1988;pp. 119-172.

\bibitem{Ocneanu2k1}
Ocneanu, A. Operator algebras, topology and subgroups of quantum symmetry~-- construction of subgroups of quantum groups.  In {\em Taniguchi Conference on Mathematics Nara '98}, {\em Adv. Stud. Pure Math.},{\bf 2001}, {\em 31}, {\em Math. Soc. Japan}: Tokyo;pp. 235--263.

\bibitem{Ostrik2k8}
Ostrik, V.   Module categories over representations of ${\rm SL}_q(2)$ in the non-simple case. {\em Geom. Funct. Anal.} {\bf 2008}, {\em 17}, 2005--2017. %%\href{http://arxiv.org/abs/math.QA/0509530}{math.QA/0509530}.

\bibitem{Patera}
Patera, J. Orbit functions of compact semisimple, however, as special functions. Iin {\em Proceedings of Fifth International Conference ``Symmetry in Nonlinear Mathematical Physics''} (June 23--29, 2003: Kyiev); Editors: A.G.~Nikitin; V.M.~Boyko; R.O.~Popovych; and I.A.~Yehorchenko. {\em Proceedings of the Institute of Mathematics, Kyiev} {\bf 2004}, {\em 50}, Part~1, 1152--1160.

\bibitem{Paterson99}
Paterson, A.L.T.  Groupoids, inverse semigroups, and their operator algebras. {\em Progress in Mathematics}, {\em 170}; Birkh\"auser Boston, Inc.: Boston, MA, 1999.

\bibitem{Paterson2k3a}
Paterson, A.L.T. The Fourier--Stieltjes and Fourier algebras for locally compact groupoids.\\
 In {\em Trends in Banach Spaces and Operator Theory, (Memphis, TN, 2001}, \\emph{Contemp. Math.} {\bf 2003}, {\em 321}, 223--237.
%% \href{http://arxiv.org/abs/math.OA/0310138}{math.OA/0310138}.

\bibitem{Paterson2k3b}
Paterson, A.L.T.  Graph inverse semigroups, groupoids and their $C^*$-algebras.  {\em J. Operator Theory} {\bf 2002}, {\em 48}, 645--662. %%\href{http://arxiv.org/abs/math.OA/0304355}{math.OA/0304355}.

\bibitem{Penrose1}
Penrose, R. The role of aestetics in pure and applied mathematical research. \emph{Bull. Inst. Math. Appl.} {\bf 1974}, \emph{10}, 266--271.

\bibitem{Penrose71}
Penrose, R. {\em Applications of Negative Dimensional Tensors}; In Welsh, Editor. {\em Combinatorial Mathematics and its Applications}; Academic Press: London, 1971.

\bibitem{Perelman}
Perelman, G. The entropy formula for the Ricci flow and its geometric applications.\\ 
%% \href{http://arxiv.org/abs/math.DG/0211159}{math.DG/0211159}.

\bibitem{Plymen-Robinson94}
Plymen, R.J.; Robinson, P.L. Spinors in Hilbert space. \emph{Cambridge Tracts in Mathematics}, Vol.~{\bf 114}; Cambridge University Press: Cambrige, 1994.

\bibitem{Podles95}
Podles, P.   Symmetries of quantum spaces. Subgroups and quotient spaces of quantum $SU (2)$ and $SO (3)$ groups. {\em Commun. Math. Phys.} {\bf 1995}, {\em 170}, 1-20.

\bibitem{Poi96}
Poincar{\'e}, H.  {\em {\OE}uvres. Tome {\em VI}}; Les Grands Classiques Gauthier-Villars. \'Editions; Jacques Gabay: \\ Sceaux ,1996. {\em G\'eom\'etrie. Analysis situs (topologie)};
reprint of the 1953 edition.

\bibitem{Popescu68}
Popescu, N.  {\em The theory of Abelian categories with applications to rings and modules.}  {\em London Mathematical Society Monographs}, no.~3; Academic Press: London-- New York, 1973.

\bibitem{Porter98}
Porter, T.  Topological quantum field theories from homotopy $n$--types. (English summary) 
{\em J. London Math. Soc. (2) } {\bf 1998}, {\em 58 }, no. 3, 723--732. 

\bibitem{Porter-T2k8}
Porter, T.; and Turaev, V.
\newblock Formal Homotopy Quantum Field Theories {I}: Formal   maps and crossed {$\cal C$-}algebras.
\newblock {\em Journal Homotopy and Related Structures} ) {\bf 2008}, {\em 3}~(1), 113--159.

\bibitem{Prigogine80}
Prigogine, I. {\em From being to becoming time and complexity in the physical sciences.}; W. H. Freeman and Co.: San Francisco, 1980.

\bibitem{Pronk95}
Pronk,  D. {\em Groupoid representations for sheaves on orbifolds.}  PhD Thesis: University of Utrecht, 1995.

\bibitem{Radin}
Radin, C.  Symmetries of Quasicrystals. \emph{Journal of Statistical Physics} {\bf 1999}, {\bf 95}, Nos 5/6, 827--833.

\bibitem{Ramsay82}
Ramsay, A. Topologies on measured groupoids. {\em J. Funct. Anal.}{\bf 1982}, {\em 47}, 314--343.

\bibitem{Ramsay-Walter97}
Ramsay, A.; Walter, M.E. Fourier--Stieltjes algebras of locally compact groupoids.  {\em J. Funct. Anal.},{\bf 1997},{\em 148}, 314--367. %%\href{http://arxiv.org/abs/math.OA/9602219}{math.OA/9602219}.

\bibitem{Ran-Wen2k6}
Ran, Y.; Wen, X.-G. Detecting topological order through a continuous quantum phase transition. {\em Phys. Rev Lett.}, {\bf 2006}, {\em 96}, 026802, 4~pages.
%% \href{http://arxiv.org/abs/cond-mat/0509155}{cond-mat/0509155}.

\bibitem{Raptis-Zapatrin2k}
Raptis I.; Zapatrin, R.R. Quantization of discretized spacetimes and the correspondence principle. {\em Internat. J. Theoret. Phys.} {\bf 2000}, {\em 39}, 1--13.
%%\href{http://arxiv.org/abs/gr-qc/9904079}{gr-qc/9904079}.

\bibitem{Reshetikhin-T91}
Reshetikhin, N.;Tureaev, V. Invariants of 3-Manifolds via Link Polynomials. {\em Inventiones Math.} {\bf 1991}, {\em 103}, 547-597.

\bibitem{Rehren97}
Rehren, H.-K. Weak $C^*$-Hopf symmetry. In {\em Proceedings of Quantum Group Symposium at Group 21} ; Goslar, 1996; Heron Press: Sofia, 1997;pp. 62--69. %%\href{http://arxiv.org/abs/q-alg/9611007}{q-alg/9611007}.

\bibitem{Regge61}
Regge, T.  General relativity without coordinates. {\em Nuovo Cimento (10)} {\bf 1961}, {\em 19}, 558--571.

\bibitem{Renault80}
Renault, J.  A groupoid approach to $C^*$-algebras. {\em Lecture Notes in Mathematics}, Vol.~{\em 793}; Springer: Berlin,1980.

\bibitem{Renault87}
Renault, J.  Repr\'esentations des produits crois\'es d'alg\`ebres de groupo\"{\i}des. {\em J. Operator Theory} {\bf 1987}, {\em 18}, 67--97.

\bibitem{Renault97}
Renault, J. The Fourier algebra of a measured groupoid and its multiplier. {\em J. Funct. Anal.} {\bf 1997}, {\em 145}, 455--490.

\bibitem{Ribeiro-Wen2k5}
Ribeiro, T.C.; Wen, X.-G.  New mean f\/ield theory of the $tt't''J$ model applied to the high-$T_c$ superconductors. {\em Phys. Rev. Lett.} {\bf 2005}, {\em 95}, 057001, 4~pages. 
%%\href{http://arxiv.org/abs/cond-mat/0410750}{cond-mat/0410750}.

\bibitem{Rieffel2k2}
Rieffel, M.A. Group $C^*$-algebras as compact quantum metric spaces. \emph{Doc. Math.}  {\bf 2002}, {\em 7}, 605--651.
 %%\href{http://arxiv.org/abs/math.OA/0205195}{math.OA/0205195}.

\bibitem{Rieffel74}
Rieffel, M.A.  Induced representations of $C^*$-algebras. {\em Adv. Math.}{\bf 1974}, {\em 13}, 176--257.

\bibitem{Roberts2k4}
Roberts, J. E.  More lectures on algebraic quantum field theory. In {\em Noncommutative Geometry};pp.263--342; Editors A. Connes et al., \textit{Springer lect. Notes in Math.} \emph{1831}; Springer: Berlin, 2004.

\bibitem{Roberts95}
Roberts, J.E.  Skein theory and Turaev--Viro invariants. {\em Topology} {\bf 1995}, {\em 34},771--787.

\bibitem{Roberts97}
Roberts, J.E. Refined state-sum invariants of 3-and 4-manifolds. In {\em Geometric Topology}: Athens, GA, 1993, {\em AMS/IP Stud. Adv. Math.}, {\em 2.1}, {\em Amer. Math. Soc.}; Providence, RI, 1997;pp. 217--234.

\bibitem{Rovelli98}
Rovelli, C. Loop quantum gravity. {\em Living Rev. Relativ.} {\bf 1998}, {\em 1}, 68~pages. 
%%\href{http://arxiv.org/abs/gr-qc/9710008}{gr-qc/9710008}.

\bibitem{Sato-Wakui2k1}
Sato, N.; Wakui, M.  $(2 + 1)$-dimensional topological quantum field theory with a Verlinde basis and Turaev--Viro--Ocneanu invariants of 3-manifolds.  In {\em Invariants of Knots and 3-Manifolds}; Kyoto, 2001. {\em Geom. Topol. Monogr.}, {\em 4}; {\em Geom. Topol. Publ.}: Coventry, 2002;pp. 281--294.  %%\href{http://arxiv.org/abs/math.QA/0210368}{math.QA/0210368}.

\bibitem{Schreiber2k10}
Schreiber, U. Invited lecture in {\em Workshop and School on Higher Gauge Theory, TQFT and Quantum Gravity, 2011}; http://sites.google.com/site/hgtqgr/ ;  {\em Conference on Higher Gauge Theory, Quantum Gravity, and Topological Field Theory, December 18, 2010};\\
 http://sbseminar.wordpress.com/2010/12/18/conference-on-higher-gauge-theory-quantum-gravity-TFT

\bibitem{Schwartz45}
Schwartz, L. G\'en\'eralisation de la notion de fonction, de d\'erivation, de transformation de Fourier et applications math\'ematiques et physiques. {\em Ann. Univ. Grenoble. Sect. Sci. Math. Phys. (N.S.)} {\bf 1945}, {\em 21}, 57--74.

\bibitem{Schwartz52}
Schwartz, L. {\em Th\'eorie des distributions.} {\em Publications de l'Institut de Math\'ematique de l'Universit\'e de Strasbourg}, no.~IX--X; Hermann: Paris, 1966.

\bibitem{Schechtman}
Schechtman, D.; Blech, L.; Gratias, D.; Cahn, J.W. Metallic phase with long-range orientational order and no translational symmetry. \emph{\em Phys. Rev. Lett.} {\bf 1984}, \emph{53}, 1951--1953.

\bibitem{Seda76}
Seda, A.K. Haar measures for groupoids. {\em Proc. Roy. Irish Acad. Sect. A} {\bf 1976}, {\em 76}, no.~5, 25--36.

\bibitem{Seda82}
Seda, A.K. Banach bundles of continuous functions and an integral representation theorem. {\em Trans. Amer. Math. Soc.} {\bf 1982}, {\em 270}, 327--332.

\bibitem{Seda86}
Seda, A.K. On the Continuity of Haar measures on topological groupoids. \emph{Proc. Amer. Math. Soc.}{\bf 1986}, {\em 96}, 115--120.

\bibitem{Segal47a}
Segal, I.E. Irreducible representations of operator algebras. \emph{Bull. Amer. Math. Soc.}   {\bf 1947}, {\em 53}, 73--88.

\bibitem{Segal47b}
Segal, I.E. Postulates for general quantum mechanics. \emph{Ann. of Math. (2)} {\bf 1947}, {\em 4}, 930--948.

\bibitem{Seifert1}
Seifert, H.  Konstruction drei dimensionaler geschlossener Raume. {\em Berichte S{\"a}chs. {A}kad. {L}eipzig, Math.-Phys. Kl.}  {\bf 1931}, {\em 83}, 26--66.

\bibitem{Serre65}
Serre, J. P. {\em Lie Algebras and Lie Groups: 1964 Lectures given at Harvard University}. Lecture Notes in Mathematics, {\em 1500}; Springer: Berlin, 1965.

\bibitem{Sheu1}
Sheu, A. J. L. Compact quantum groups and groupoid C*-algebras. \emph{J. Funct. Anal.} {\bf 1997}, {\em 144, no. 2}, 371--393.

\bibitem{Sheu2}
Sheu, A. J. L. The structure of quantum spheres. {\em J. Funct. Anal.}  {\bf 2001}, {\em 129, no. 11}, 3307--3311.

\bibitem{Sklyanin83}
Sklyanin, E.K. Some algebraic structures connected with the Yang--Baxter equation. {\em Funktsional. Anal. i Prilozhen.} {\bf 1997}, {\em 6}; {\bf 16} (1982), no.~4, 27--34.

\bibitem{Sklyanin84}
Sklyanin, E.K. Some algebraic structures connected with the Yang--Baxter equation. Representations of quantum algebras. {\em Funct. Anal. Appl.}  {\bf 1984}, {\em 17}, 273--284.

\bibitem{Sklyanin83en}
Sklyanin, E. K. Some Algebraic Structures Connected with the Yang--Baxter equation. {\em Funct. Anal. Appl.}  {\bf 1983}, {\em 16}, 263--270.

\bibitem{Street1}
Street, R. The quantum double and related constructions. {\em J. Pure Appl. Algebra} {\bf 1998}, {\em 132, no. 2}, 195--206.

\bibitem{Stern-Halperin2k6}
Stern, A.; Halperin, B.I. Proposed experiments to probe the non-Abelian $\nu=5/2$ quantum Hall state. {\em Phys. Rev. Lett.}  {\bf 2006}, {\em 96}, 016802, 4~pages. 

%%\href{http://arxiv.org/abs/cond-mat/0508447}{cond-mat/0508447}.

\bibitem{Stradling78}
Stradling, R.A. Quantum transport. {\em Phys. Bulletin}  {\bf 1978}, {\em 29}, 559--562.

\bibitem{Szlachanyi2k4}
Szlach\'anyi, K. The double algebraic view of f\/inite quantum groupoids. {\em J. Algebra} {\bf 2004}, {\em280}, 249--294.

\bibitem{Sweedler96}
Sweedler, M.E.  Hopf algebras. {\em Mathematics Lecture Note Series}; W.A. Benjamin, Inc.: New York, 1969.

\bibitem{Tn2k6}
T\u{a}nas\u{a},  A.  Extension of the Poincar\'e symmetry and its field theoretical interpretation. {\em SIGMA}  {\bf 2006}, {\em 2}, 056, 23~pages. %%\href{http://arxiv.org/abs/hep-th/0510268}{hep-th/0510268}.

\bibitem{Taylor88}
Taylor, J. Quotients of groupoids by the action of a group. \emph{Math. Proc. Cambridge Philos. Soc.} (1988), {\bf 103}, 239--249.

\bibitem{Tonks93}
Tonks, A.P. {\em Theory and applications of crossed complexes}. PhD Thesis; University of Wales: Bangor, 1993.

\bibitem{Tsui-Allen2k7}
Tsui, D.C.; Allen, S.J. Jr.  Mott--Anderson localization in the two-dimensional band tail of Si inversion layers. {\em Phys. Rev. Lett.}  {\bf 1974}, {\em 32}, 1200--1203.

\bibitem{Turaev--Viro92}
Turaev, V.G.; Viro, O.Ya. State sum invariants of 3-manifolds and quantum $6j$-symbols. \emph{Topology} (1992), {\bf 31}, 865--902.

\bibitem{van Kampen33}
van Kampen, E.H. On the connection between the fundamental groups of some related spaces. \emph{Amer. J. Math.} (1933), {\bf 55}, 261--267.

\bibitem{Varilly97}
V\'arilly, J.C. An introduction to noncommutative geometry. {\em EMS Series of Lectures in Mathematics}; European Mathematical Society (EMS): Z\"{u}rich, 2006.

\bibitem{Weinberg96}
Weinberg, S. {\em The quantum theory of fields., Vol. I. Foundations and Vol. II. Modern applications.}; Cambridge University Press: Cambridge, 1996-1997.

\bibitem{Weinberg2005}
Weinberg, S. {\em The quantum theory of fields. Vol.~III. Supersymmetry}; Cambridge University Press: Cambridge, 2005.

\bibitem{Weinstein96}
Weinstein, A.  Groupoids: unifying internal and external symmetry. A tour through some examples.{\em Notices Amer. Math. Soc.} {\bf 1996}, {\em 43}, 744--752. 
%%\href{http://arxiv.org/abs/math.RT/9602220}{math.RT/9602220}.

\bibitem{Wen91}
Wen, X.-G.  Non-Abelian statistics in the fractional quantum Hall states. {\em Phys. Rev. Lett.} {\bf 1991}, {\em 66}, 802--805.

\bibitem{Wen99}
Wen, X.-G.  Projective construction of non-Abelian quantum Hall liquids. {\em Phys. Rev. B} {\bf 1999}, {\em 60}, 8827, 4~pages. %%\href{http://arxiv.org/abs/cond-mat/9811111}{cond-mat/9811111}.

\bibitem{Wen2k3}
Wen, X.-G. Quantum order from string-net condensations and origin of light and massless fermions.\\ {\em Phys. Rev. D} {\bf 2003}, {\em 68}, 024501, 25~pages. %%\href{http://arxiv.org/abs/hep-th/0302201}{hep-th/0302201}.

\bibitem{Wen2k4}
Wen, X.-G.{\em Quantum field theory of many--body systems -- from the origin of sound to an origin of\\  light and electrons}; Oxford University Press: Oxford, 2004.

\bibitem{Wess-Bagger83}
Wess, J.; Bagger, J. Supersymmetry and supergravity. {\em Princeton Series in Physics}; Princeton University Press: Princeton, N.J., 1983.

\bibitem{WJ1}
Westman, J.J. Harmonic analysis on groupoids. {\em Pacific J. Math.} {\bf 1968}, {\em 27}, 621--632.

\bibitem{WJ2}
Westman, J.J. {\em Groupoid theory in algebra, topology and analysis}; University of California at Irvine, 1971.

\bibitem{Wickramasekara-Bohm2k2}
Wickramasekara, S.; Bohm, A.  Symmetry representations in the rigged Hilbert space formulation of quantum mechanics.  {\em J. Phys. A: Math. Gen.} {\bf 2002}, {\em 35}, 807--829. 

%%\href{http://arxiv.org/abs/math-ph/0302018}{math-ph/0302018}.

\bibitem{Wightman56}
Wightman, A.S. Quantum field theory in terms of vacuum expectation values. {\em Phys. Rev.} {\bf 1956}, {\em 101}, 860--866.

\bibitem{Wightman76}
Wightman, A.S., Hilbert's sixth problem: mathematical treatment of the axioms of physics.  In {\em Mathematical Developments Arising from Hilbert Problems : Proc. Sympos. Pure Math.}, Northern Illinois Univ., De Kalb, Ill., 1974. {\em Amer. Math. Soc.}: Providence, R.I., 1976;pp. 147--240.

\bibitem{Wightman-Garding64}
Wightman, A.S.; G\"arding,  L.  Fields as operator-valued distributions in relativistic quantum theory. {\em Ark. Fys.} {\bf 1964}, {\em 28}, 129--184.

\bibitem{WEP31}
 Wigner, E. P. \emph {Gruppentheorie}; Friedrich Vieweg und Sohn: Braunschweig, Germany, 1931;pp. 251-254. {\em Group Theory}; Academic Press Inc.: New York, 1959;pp. 233-236.

\bibitem{WEP39}
Wigner, E. P.  On unitary representations of the inhomogeneous Lorentz group. {\em Annals of Mathematics} {\bf 1939}, {\em 40} (1), 149--204. doi:10.2307/1968551.

\bibitem{Witten89}
Witten, E. Quantum field theory and the Jones polynomial. {\em Comm. Math. Phys.} {\bf 1989}, {\em 121}, 351--355.

\bibitem{Witten98}
Witten, E. Anti de Sitter space and holography. {\em Adv. Theor. Math. Phys.} {\bf 1998},{\em 2}, 253--291. 
%%\mbox{\href{http://arxiv.org/abs/hep-th/9802150}{hep-th/9802150}}.

\bibitem{Wood97}
Wood, E.E. Reconstruction Theorem for Groupoids and Principal Fiber Bundles. {\em Intl. J. Theor. Physics} {\em 1997} {\em 36} (5), 1253-1267.  DOI: 10.1007/BF02435815. 

\bibitem{Woronowicz1}
Woronowicz, S.L. Twisted $SU(2)$ group. An example of a non-commutative differential calculus. \emph{Publ. Res. Inst. Math. Sci. } (1987), {\bf 23},
117--181.

\bibitem{Woronowicz98}
Woronowicz, S. L. Compact quantum groups. In \emph{Quantum Symmetries.}  Les Houches Summer School--1995, Session LXIV; Editors: A. Connes; K. Gawedzki; J. Zinn-Justin; Elsevier Science: Amsterdam,1998;pp. 845--884.

\bibitem{Xu97}
Xu, P. Quantum groupoids and deformation quantization. \emph{C. R. Acad. Sci. Paris S\'{e}r. I Math.} (1998), {\bf 326}, 289--294.  %%\href{http://arxiv.org/abs/q-alg/9708020}{q-alg/9708020}.

\bibitem{Yang-Mills54}
Yang, C.N.; Mills, R.L. Conservation of isotopic spin and isotopic gauge invariance. {\em Phys. Rev.} {\bf 1954}, {\em 96}, 191--195.

\bibitem{Yang62}
Yang, C.N. Concept of Off-Diagonal Long-Range Order and the Quantum Phases of Liquid He and of Superconductors.{\em Rev. Mod. Phys.} {\bf 1962}, {\em 34}, 694--704.

\bibitem{Yetter93}
Yetter, D.N. TQFTs from homotopy 2-types. \textit{J. Knot Theory Ramifications} {\bf 1993}, {\em 2}, 113--123.

\bibitem{Ypma1}
Ypma, F.  K-theoretic gap labelling for quasicrystals, {\em Contemp. Math.}{\bf 2007}, {\em 434}, 247--255; \emph{Amer. Math. Soc.}: Providence, RI,.

\bibitem{Ypma2}
Ypma, F. {\em Quasicrystals, C*-algebras and K-theory}. Msc. Thesis. {\bf 2004}: University of Amsterdam.

\bibitem{Zhang91}
Zhang, R.B. Invariants of the quantum supergroup $U_q(gl(m/1))$. {\em J. Phys. A: Math. Gen.} {\bf 1991}, {\em 24}, L1327--L1332.

\bibitem{Zhang-Gould99}
Zhang, Y.-Z.; Gould,  M.D. Quasi-Hopf superalgebras and elliptic quantum supergroups, \emph{J. Math. Phys.} {\bf 1999}, {\em 40}, 5264--5282,
%%\href{http://arxiv.org/abs/math.QA/9809156}{math.QA/9809156}.


\end{thebibliography}

%%%%%
%%%%%
\end{document}

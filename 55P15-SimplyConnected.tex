\documentclass[12pt]{article}
\usepackage{pmmeta}
\pmcanonicalname{SimplyConnected}
\pmcreated{2013-03-22 11:59:33}
\pmmodified{2013-03-22 11:59:33}
\pmowner{CWoo}{3771}
\pmmodifier{CWoo}{3771}
\pmtitle{simply connected}
\pmrecord{9}{30911}
\pmprivacy{1}
\pmauthor{CWoo}{3771}
\pmtype{Definition}
\pmcomment{trigger rebuild}
\pmclassification{msc}{55P15}
\pmrelated{SemilocallySimplyConnected}

\endmetadata

\usepackage{amssymb}
\usepackage{amsmath}
\usepackage{amsfonts}
\usepackage{graphicx}
%%%\usepackage{xypic}
\begin{document}
A topological space is said to be \emph{simply connected} if it is path connected and the fundamental group of the space is trivial (i.e. the one element group).  What this means, basically, is that every path on the space can be shrunk to a point.  This is equivalent to saying that every path is contractible.  A simply connected space can be visualized as a space with no ``holes''.

Some basic examples of a simply connected space are the unit disc in $\mathbb{R}^2$, $S^2$ or the Riemann sphere.  Non-examples of a simply connected space are the circle, the annulus, and a punctured plane (a plane with a point removed).  In each of the non-examples, any closed curve around the ``hole'' is a path that can not be shrunk to a point.
%%%%%
%%%%%
%%%%%
\end{document}

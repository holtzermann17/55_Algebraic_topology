\documentclass[12pt]{article}
\usepackage{pmmeta}
\pmcanonicalname{Monodromy}
\pmcreated{2013-03-22 13:26:20}
\pmmodified{2013-03-22 13:26:20}
\pmowner{mathcam}{2727}
\pmmodifier{mathcam}{2727}
\pmtitle{monodromy}
\pmrecord{8}{34000}
\pmprivacy{1}
\pmauthor{mathcam}{2727}
\pmtype{Definition}
\pmcomment{trigger rebuild}
\pmclassification{msc}{55R05}
\pmrelated{MonodromyGroup}
\pmdefines{monodromy}
\pmdefines{monodromy action}
\pmdefines{monodromy homomorphism}

\usepackage{amsmath}
%\usepackage[all,poly,knot,dvips]{xy}
%\usepackage{pstricks,pst-poly,pst-node,pstcol}


\usepackage{amssymb,latexsym}

\usepackage{amsthm,latexsym}
\usepackage{eucal,latexsym}

% THEOREM Environments --------------------------------------------------

\newtheorem{thm}{Theorem}
 \newtheorem*{mainthm}{Main~Theorem}
 \newtheorem{cor}[thm]{Corollary}
 \newtheorem{lem}[thm]{Lemma}
 \newtheorem{prop}[thm]{Proposition}
 \newtheorem{claim}[thm]{Claim}
 \theoremstyle{definition}
 \newtheorem{defn}[thm]{Definition}
 \theoremstyle{remark}
 \newtheorem{rem}[thm]{Remark}
 \numberwithin{equation}{subsection}


%---------------------  Greek letters, etc ------------------------- 

\newcommand{\CA}{\mathcal{A}}
\newcommand{\CC}{\mathcal{C}}
\newcommand{\CM}{\mathcal{M}}
\newcommand{\CP}{\mathcal{P}}
\newcommand{\CS}{\mathcal{S}}
\newcommand{\BC}{\mathbb{C}}
\newcommand{\BN}{\mathbb{N}}
\newcommand{\BR}{\mathbb{R}}
\newcommand{\BZ}{\mathbb{Z}}
\newcommand{\FF}{\mathfrak{F}}
\newcommand{\FL}{\mathfrak{L}}
\newcommand{\FM}{\mathfrak{M}}
\newcommand{\Ga}{\alpha}
\newcommand{\Gb}{\beta}
\newcommand{\Gg}{\gamma}
\newcommand{\GG}{\Gamma}
\newcommand{\Gd}{\delta}
\newcommand{\GD}{\Delta}
\newcommand{\Ge}{\varepsilon}
\newcommand{\Gz}{\zeta}
\newcommand{\Gh}{\eta}
\newcommand{\Gq}{\theta}
\newcommand{\GQ}{\Theta}
\newcommand{\Gi}{\iota}
\newcommand{\Gk}{\kappa}
\newcommand{\Gl}{\lambda}
\newcommand{\GL}{\Lamda}
\newcommand{\Gm}{\mu}
\newcommand{\Gn}{\nu}
\newcommand{\Gx}{\xi}
\newcommand{\GX}{\Xi}
\newcommand{\Gp}{\pi}
\newcommand{\GP}{\Pi}
\newcommand{\Gr}{\rho}
\newcommand{\Gs}{\sigma}
\newcommand{\GS}{\Sigma}
\newcommand{\Gt}{\tau}
\newcommand{\Gu}{\upsilon}
\newcommand{\GU}{\Upsilon}
\newcommand{\Gf}{\varphi}
\newcommand{\GF}{\Phi}
\newcommand{\Gc}{\chi}
\newcommand{\Gy}{\psi}
\newcommand{\GY}{\Psi}
\newcommand{\Gw}{\omega}
\newcommand{\GW}{\Omega}
\newcommand{\Gee}{\epsilon}
\newcommand{\Gpp}{\varpi}
\newcommand{\Grr}{\varrho}
\newcommand{\Gff}{\phi}
\newcommand{\Gss}{\varsigma}

\def\co{\colon\thinspace}
\begin{document}
Let $(X,*)$ be a connected and locally connected based space and $p\co E \to X$ a covering map. We will denote $p^{-1}(*)$, the fiber over the basepoint, by $F$, and the fundamental
group $\pi_1(X,*)$ by $\pi$. Given a loop
$\Gg\co I \to X$ with $\Gg(0)=\Gg(1)=*$ and a point $e\in F$ there exists a
unique $\tilde\Gg\co I\to E,$ with $ \tilde\Gg(0)=e$ such that $p\circ\tilde\Gg=\Gg$,
that is, a lifting of $\Gg$ starting at $e$. Clearly, the endpoint
$\tilde\Gg(1)$ is also a point of the fiber, which we will denote by
$e\cdot\Gg$.

\begin{thm}
  With  notation as above we have:
  \begin{enumerate}
  \item If $\Gg_1$ and $\Gg_2$ are homotopic relative $\partial I$ then
$$\forall e\in F \quad e\cdot\Gg_1=e\cdot\Gg_2.$$
\item The map 
$$ F\times\pi\to F,\quad (e,\Gg)\mapsto e\cdot\Gg$$
defines a right action of $\pi$ on $F$.
\item The stabilizer of a point $e$ is the image of the fundamental group
  $\pi_1(E,e)$ under the map induced by $p$:
$$ \operatorname{Stab}(x) = p_{*}\left(\pi_1(E,e)\right)\,.$$
 
  \end{enumerate}
\end{thm}
\begin{proof}\begin{enumerate}
 \item Let $e\in F$,  $\Gg_1,\Gg_2\co I\to X$ two loops  homotopic relative
 $\partial I$ and  $\tilde\Gg_1,\tilde\Gg_2\co I\to E$ their liftings
 starting at $e$. Then there is a homotopy $H\co I\times I \to X$ with the
 following properties:
 \begin{itemize}
 \item $H(\bullet,0)=\Gg_1$,
 \item  $H(\bullet,1)=\Gg_2$,
 \item $H(0,t)=H(1,t)=*,\quad \forall t\in I$.
  \end{itemize}
 According to the lifting theorem $H$ lifts to a homotopy $\tilde H\co
 I\times I\to E$ with $H(0,0)=e$. Notice that  $\tilde
 H(\bullet,0)=\tilde\Gg_1$ (respectively $\tilde H(\bullet,1)=\tilde\Gg_2$)
 since they both are liftings of $\Gg_1$ (respectively $\Gg_2$) starting at
 $e$. Also notice that that $\tilde H(1,\bullet)$ is a path that lies entirely in
 the fiber (since it lifts the constant path $*$). Since the fiber is
 discrete this means that  $\tilde H(1,\bullet)$ is a constant path. In particular
 $\tilde H(1,0)=\tilde H(1,1)$ or equivalently $\tilde \Gg_1(1)=\tilde
 \Gg_2(1)$.
\item By (1) the map is well defined. To prove that it is an action notice
  that firstly the constant path $*$ lifts to constant paths and therefore
$$\forall e\in F,\quad e\cdot 1=e\,.$$
 Secondly the concatenation of two paths lifts to the concatenation of their
 liftings (as is easily verified by projecting). In other words, the lifting
 of $\Gg_1\Gg_2$ that starts at $e$ is the concatenation of $\tilde \Gg_1$, 
  the lifting of
 $\Gg_1$ that starts at $e$, and $\tilde \Gg_2$  the lifting of $\Gg_2$ that
 starts in $\Gg_1(1)$. Therefore 
  $$e\cdot (\Gg_1\Gg_2)=(e\cdot \Gg_1)\cdot \Gg_2\,.$$   
\item This is a tautology: $\Gg$ fixes $e$ if and only if its lifting
  starting at $e$ is a loop.  
\end{enumerate}
\end{proof}
 
\begin{defn}
The action described in the above theorem is called the \emph{monodromy
  action} and the corresponding homomorphism
 $$\Gr\co \Gp\to {\rm Sym}(F) $$
 is called the \emph{monodromy} of $p$. 
\end{defn}
%%%%%
%%%%%
\end{document}

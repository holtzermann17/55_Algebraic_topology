\documentclass[12pt]{article}
\usepackage{pmmeta}
\pmcanonicalname{Fibration}
\pmcreated{2013-03-22 15:37:57}
\pmmodified{2013-03-22 15:37:57}
\pmowner{whm22}{2009}
\pmmodifier{whm22}{2009}
\pmtitle{fibration}
\pmrecord{5}{37560}
\pmprivacy{1}
\pmauthor{whm22}{2009}
\pmtype{Definition}
\pmcomment{trigger rebuild}
\pmclassification{msc}{55R65}
\pmrelated{fibremap}
\pmrelated{FibreBundle}
\pmrelated{LocallyTrivialBundle}
\pmrelated{LongExactSequenceLocallyTrivialBundle}
\pmrelated{homotopyliftingproperty}
\pmrelated{cofibration}
\pmdefines{fibration}

\endmetadata

% this is the default PlanetMath preamble.  as your knowledge
% of TeX increases, you will probably want to edit this, but
% it should be fine as is for beginners.

% almost certainly you want these
\usepackage{amssymb}
\usepackage{amsmath}
\usepackage{amsfonts}

% used for TeXing text within eps files
%\usepackage{psfrag}
% need this for including graphics (\includegraphics)
%\usepackage{graphicx}
% for neatly defining theorems and propositions
%\usepackage{amsthm}
% making logically defined graphics
%%%\usepackage{xypic}

% there are many more packages, add them here as you need them

% define commands here
\begin{document}
A fibration is a map satisfying the homotopy lifting property.  This is easily seen to be equivalent to the following:

A map $f:X \to Y$ is a fibration if and only if there is a continuous function which given a path, $\phi$, in $Y$ and a point, $x$, lying above $\phi(0)$, returns a lift of $\phi$, starting at $x$.

Let $D^2$ denote the set of complex numbers with modulus less than or equal to 1. An example of a fibration is the map $g: D^2 \to [-1,1]$ sending a complex number $z$ to $re(z)$. 

Note that if we restrict $g$ to the boundary of $D^2$, we do not get a fibration.  Although we may still lift any path to begin at a prescribed point, we cannot make this assignment continuously.

Another class of fibrations are found in fibre bundles.
%%%%%
%%%%%
\end{document}

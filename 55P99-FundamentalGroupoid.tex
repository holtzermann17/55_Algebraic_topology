\documentclass[12pt]{article}
\usepackage{pmmeta}
\pmcanonicalname{FundamentalGroupoid}
\pmcreated{2013-03-22 13:24:00}
\pmmodified{2013-03-22 13:24:00}
\pmowner{CWoo}{3771}
\pmmodifier{CWoo}{3771}
\pmtitle{fundamental groupoid}
\pmrecord{9}{33941}
\pmprivacy{1}
\pmauthor{CWoo}{3771}
\pmtype{Definition}
\pmcomment{trigger rebuild}
\pmclassification{msc}{55P99}
\pmrelated{FundamentalGroupoidFunctor}
\pmrelated{FundamentalGroupoid2}
\pmrelated{HomotopyDoubleGroupoidOfAHausdorffSpace}
\pmrelated{QuantumFundamentalGroupoids}
\pmrelated{HomotopyCategory}
\pmrelated{GroupoidCategory}

%\documentclass{amsart}
\usepackage{amsmath}
\usepackage[all,poly,knot,dvips]{xy}
%\usepackage{pstricks,pst-poly,pst-node,pstcol}


\usepackage{amssymb,latexsym}

\usepackage{amsthm,latexsym}
\usepackage{eucal,latexsym}

% THEOREM Environments --------------------------------------------------

\newtheorem{thm}{Theorem}
 \newtheorem*{mainthm}{Main~Theorem}
 \newtheorem{cor}[thm]{Corollary}
 \newtheorem{lem}[thm]{Lemma}
 \newtheorem{prop}[thm]{Proposition}
 \newtheorem{claim}[thm]{Claim}
 \theoremstyle{definition}
 \newtheorem{defn}[thm]{Definition}
 \theoremstyle{remark}
 \newtheorem{rem}[thm]{Remark}
 \numberwithin{equation}{subsection}


%---------------------  Greek letters, etc ------------------------- 

\newcommand{\CA}{\mathcal{A}}
\newcommand{\CC}{\mathcal{C}}
\newcommand{\CM}{\mathcal{M}}
\newcommand{\CP}{\mathcal{P}}
\newcommand{\CS}{\mathcal{S}}
\newcommand{\BC}{\mathbb{C}}
\newcommand{\BN}{\mathbb{N}}
\newcommand{\BR}{\mathbb{R}}
\newcommand{\BZ}{\mathbb{Z}}
\newcommand{\FF}{\mathfrak{F}}
\newcommand{\FL}{\mathfrak{L}}
\newcommand{\FM}{\mathfrak{M}}
\newcommand{\Ga}{\alpha}
\newcommand{\Gb}{\beta}
\newcommand{\Gg}{\gamma}
\newcommand{\GG}{\Gamma}
\newcommand{\Gd}{\delta}
\newcommand{\GD}{\Delta}
\newcommand{\Ge}{\varepsilon}
\newcommand{\Gz}{\zeta}
\newcommand{\Gh}{\eta}
\newcommand{\Gq}{\theta}
\newcommand{\GQ}{\Theta}
\newcommand{\Gi}{\iota}
\newcommand{\Gk}{\kappa}
\newcommand{\Gl}{\lambda}
\newcommand{\GL}{\Lamda}
\newcommand{\Gm}{\mu}
\newcommand{\Gn}{\nu}
\newcommand{\Gx}{\xi}
\newcommand{\GX}{\Xi}
\newcommand{\Gp}{\pi}
\newcommand{\GP}{\Pi}
\newcommand{\Gr}{\rho}
\newcommand{\Gs}{\sigma}
\newcommand{\GS}{\Sigma}
\newcommand{\Gt}{\tau}
\newcommand{\Gu}{\upsilon}
\newcommand{\GU}{\Upsilon}
\newcommand{\Gf}{\varphi}
\newcommand{\GF}{\Phi}
\newcommand{\Gc}{\chi}
\newcommand{\Gy}{\psi}
\newcommand{\GY}{\Psi}
\newcommand{\Gw}{\omega}
\newcommand{\GW}{\Omega}
\newcommand{\Gee}{\epsilon}
\newcommand{\Gpp}{\varpi}
\newcommand{\Grr}{\varrho}
\newcommand{\Gff}{\phi}
\newcommand{\Gss}{\varsigma}

\def\co{\colon\thinspace}
\begin{document}
\begin{defn} Given a topological space $X$ the fundamental groupoid
 $\Pi_1(X)$ of $X$ is defined as
follows:
\begin{itemize}
\item The objects of $\Pi_1(X)$ are the points of $X$
$$\mathrm{Obj}(\Pi_1(X))=X\,,$$
\item morphisms are homotopy classes of paths ``rel endpoints'' that is
$$\mathrm{Hom}_{\Pi_1(X)}(x,y)=\mathrm{Paths}(x,y)/\sim\,,$$
where, $\sim$ denotes homotopy rel endpoints, and,
\item composition of morphisms is defined via concatenation of paths.
\end{itemize}
\end{defn}

It is easily checked that the above defined category is indeed a groupoid
with the inverse of (a morphism represented by) a path being (the homotopy
class of) the ``reverse'' path.
Notice that for $x \in X$, the group of automorphisms of $x$ is the
fundamental group of $X$ with basepoint $x$,
$$\mathrm{Hom}_{\Pi_1(X)}(x,x)=\pi_1(X,x)\,.$$

\begin{defn}
  Let $f\co X\to Y$ be a continuous function between two topological spaces.
Then there is an induced functor 
$$\Pi_1(f)\co \Pi_1(X)\to\Pi_1(Y)$$
defined as follows
\begin{itemize}
\item on objects $\Pi_1(f)$ is just $f$,
\item on morphisms $\Pi_1(f)$ is given by ``composing with $f$'', that is 
if $\alpha\co I\to$~$ X$ is a path representing the morphism 
$[\alpha]\co x\to y$ then a representative of
 $\Pi_1(f)([\alpha])\co f(x)\to f(y)$
is determined by the following commutative diagram 
 $$\xymatrix{{I}\ar[d]_{\alpha}\ar@{-->}[dr]^{\Pi_1(f)(\alpha)}\\
 {X}\ar[r]_f&{Y} }$$
\end{itemize}
\end{defn}

It is straightforward to check that the above indeed defines a
functor. Therefore $\Pi_1$ can (and should) be regarded as a functor from
the category of topological spaces to the category of groupoids. This functor 
is not really homotopy invariant but it is ``homotopy invariant up to
homotopy'' in the sense that the following holds.
\begin{thm}
  A homotopy between two continuous maps induces a natural transformation
  between the corresponding functors.
\end{thm}

A reader who understands the meaning of the statement should be able to give 
an explicit construction and supply the proof without much trouble.
%%%%%
%%%%%
\end{document}

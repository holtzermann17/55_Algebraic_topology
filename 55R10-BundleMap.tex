\documentclass[12pt]{article}
\usepackage{pmmeta}
\pmcanonicalname{BundleMap}
\pmcreated{2013-03-22 13:07:24}
\pmmodified{2013-03-22 13:07:24}
\pmowner{RevBobo}{4}
\pmmodifier{RevBobo}{4}
\pmtitle{bundle map}
\pmrecord{5}{33557}
\pmprivacy{1}
\pmauthor{RevBobo}{4}
\pmtype{Definition}
\pmcomment{trigger rebuild}
\pmclassification{msc}{55R10}
\pmdefines{bundle morphism}

\endmetadata

% this is the default PlanetMath preamble.  as your knowledge
% of TeX increases, you will probably want to edit this, but
% it should be fine as is for beginners.

% almost certainly you want these
\usepackage{amssymb}
\usepackage{amsmath}
\usepackage{amsfonts}
%%\usepackage{xypic}

% used for TeXing text within eps files
%\usepackage{psfrag}
% need this for including graphics (\includegraphics)
%\usepackage{graphicx}
% for neatly defining theorems and propositions
%\usepackage{amsthm}
% making logically defined graphics
%%%\usepackage{xypic} 

% there are many more packages, add them here as you need them

% define commands here
\begin{document}
Let $E_1 \overset{\pi_1}\to B_1$ and $E_2 \overset{\pi_2}\to B_2$ be fiber bundles for which there is a continuous map $f:B_1 \to B_2$ of base spaces. A \emph{bundle map} (or \emph{bundle morphism}) is  a commutative square
$$
\xymatrix{
E_1 \ar[r]^{\hat{f}} \ar[d]_{\pi_1} & E_2 \ar[d]^{\pi_2} \\
B_1 \ar[r]^{f} & B_2
}
$$
such that the induced map $E_1 \to f^{-1}E_2$ is a homeomorphism (here $f^{-1}E_2$ denotes the pullback of $f$ along the bundle projection $\pi_2$).
%%%%%
%%%%%
\end{document}

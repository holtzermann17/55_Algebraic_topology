\documentclass[12pt]{article}
\usepackage{pmmeta}
\pmcanonicalname{HomotopyInvariance}
\pmcreated{2013-03-22 14:24:51}
\pmmodified{2013-03-22 14:24:51}
\pmowner{pbruin}{1001}
\pmmodifier{pbruin}{1001}
\pmtitle{homotopy invariance}
\pmrecord{4}{35919}
\pmprivacy{1}
\pmauthor{pbruin}{1001}
\pmtype{Definition}
\pmcomment{trigger rebuild}
\pmclassification{msc}{55Pxx}
\pmrelated{HomotopyEquivalence}
\pmdefines{homotopy invariant}

\endmetadata

% this is the default PlanetMath preamble.  as your knowledge
% of TeX increases, you will probably want to edit this, but
% it should be fine as is for beginners.

% almost certainly you want these
\usepackage{amssymb}
\usepackage{amsmath}
\usepackage{amsfonts}

% used for TeXing text within eps files
%\usepackage{psfrag}
% need this for including graphics (\includegraphics)
%\usepackage{graphicx}
% for neatly defining theorems and propositions
%\usepackage{amsthm}
% making logically defined graphics
%%%\usepackage{xypic}

% there are many more packages, add them here as you need them

% define commands here
\begin{document}
Let $\cal F$ be a functor from the category of topological spaces to some category $\cal C$.  Then $\cal F$ is called {\em homotopy invariant} if for any two homotopic maps $f,g\colon X\to Y$ between topological spaces $X$ and $Y$ the morphisms ${\cal F}f$ and ${\cal F}g$ in $\cal C$ induced by $\cal F$ are identical.

Suppose $\cal F$ is a homotopy invariant functor, and $X$ and $Y$ are homotopy equivalent topological spaces.  Then there are continuous maps $f\colon X\to Y$ and $g\colon Y\to X$ such that $g\circ f\simeq{\rm id}_X$ and $f\circ g\simeq{\rm id}_Y$ (i.e. $g\circ f$ and $f\circ g$ are homotopic to the identity maps on $X$ and $Y$, respectively).  Assume that $\cal F$ is a covariant functor.  Then the homotopy invariance of $\cal F$ implies
$$
{\cal F}g\circ{\cal F}f={\cal F}(g\circ f)={\rm id}_{{\cal F}X}
$$
and
$$
{\cal F}f\circ{\cal F}g={\cal F}(f\circ g)={\rm id}_{{\cal F}Y}.
$$
From this we see that ${\cal F}X$ and ${\cal F}Y$ are isomorphic in $\cal C$.  (The same argument clearly holds if $\cal F$ is contravariant instead of covariant.)

An important example of a homotopy invariant functor is the fundamental group $\pi_1$; here $\cal C$ is the category of groups.
%%%%%
%%%%%
\end{document}

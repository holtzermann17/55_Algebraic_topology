\documentclass[12pt]{article}
\usepackage{pmmeta}
\pmcanonicalname{LiftingTheorem}
\pmcreated{2013-03-22 13:24:58}
\pmmodified{2013-03-22 13:24:58}
\pmowner{Dr_Absentius}{537}
\pmmodifier{Dr_Absentius}{537}
\pmtitle{lifting theorem}
\pmrecord{9}{33963}
\pmprivacy{1}
\pmauthor{Dr_Absentius}{537}
\pmtype{Theorem}
\pmcomment{trigger rebuild}
\pmclassification{msc}{55R05}

%\documentclass{amsart}
\usepackage{amsmath}
\usepackage[all,poly,knot,dvips]{xy}
%\usepackage{pstricks,pst-poly,pst-node,pstcol}


\usepackage{amssymb,latexsym}

\usepackage{amsthm,latexsym}
\usepackage{eucal,latexsym}

% THEOREM Environments --------------------------------------------------

\newtheorem{thm}{Theorem}
 \newtheorem*{mainthm}{Main~Theorem}
 \newtheorem{cor}[thm]{Corollary}
 \newtheorem{lem}[thm]{Lemma}
 \newtheorem{prop}[thm]{Proposition}
 \newtheorem{claim}[thm]{Claim}
 \theoremstyle{definition}
 \newtheorem{defn}[thm]{Definition}
 \theoremstyle{remark}
 \newtheorem{rem}[thm]{Remark}
 \numberwithin{equation}{subsection}


%---------------------  Greek letters, etc ------------------------- 

\newcommand{\CA}{\mathcal{A}}
\newcommand{\CC}{\mathcal{C}}
\newcommand{\CM}{\mathcal{M}}
\newcommand{\CP}{\mathcal{P}}
\newcommand{\CS}{\mathcal{S}}
\newcommand{\BC}{\mathbb{C}}
\newcommand{\BN}{\mathbb{N}}
\newcommand{\BR}{\mathbb{R}}
\newcommand{\BZ}{\mathbb{Z}}
\newcommand{\FF}{\mathfrak{F}}
\newcommand{\FL}{\mathfrak{L}}
\newcommand{\FM}{\mathfrak{M}}
\newcommand{\Ga}{\alpha}
\newcommand{\Gb}{\beta}
\newcommand{\Gg}{\gamma}
\newcommand{\GG}{\Gamma}
\newcommand{\Gd}{\delta}
\newcommand{\GD}{\Delta}
\newcommand{\Ge}{\varepsilon}
\newcommand{\Gz}{\zeta}
\newcommand{\Gh}{\eta}
\newcommand{\Gq}{\theta}
\newcommand{\GQ}{\Theta}
\newcommand{\Gi}{\iota}
\newcommand{\Gk}{\kappa}
\newcommand{\Gl}{\lambda}
\newcommand{\GL}{\Lamda}
\newcommand{\Gm}{\mu}
\newcommand{\Gn}{\nu}
\newcommand{\Gx}{\xi}
\newcommand{\GX}{\Xi}
\newcommand{\Gp}{\pi}
\newcommand{\GP}{\Pi}
\newcommand{\Gr}{\rho}
\newcommand{\Gs}{\sigma}
\newcommand{\GS}{\Sigma}
\newcommand{\Gt}{\tau}
\newcommand{\Gu}{\upsilon}
\newcommand{\GU}{\Upsilon}
\newcommand{\Gf}{\varphi}
\newcommand{\GF}{\Phi}
\newcommand{\Gc}{\chi}
\newcommand{\Gy}{\psi}
\newcommand{\GY}{\Psi}
\newcommand{\Gw}{\omega}
\newcommand{\GW}{\Omega}
\newcommand{\Gee}{\epsilon}
\newcommand{\Gpp}{\varpi}
\newcommand{\Grr}{\varrho}
\newcommand{\Gff}{\phi}
\newcommand{\Gss}{\varsigma}

\def\co{\colon\thinspace}
\begin{document}
Let $p\co E\to B$ be a covering map and $f\co X\to B$ be a (continuous)
map where $X$,  $B$ and $E$ are path connected and \PMlinkname{locally path connected}{LocallyConnected}.  
Also let $x\in X$  and $e\in E$ be points such that $f(x)=p(e)$.
 Then $f$ lifts to a map $\tilde f\co X\to E$ with $\tilde f(x)=e$ if and only if
$\pi_1(f)$ maps  $\pi_1(X,x)$ inside the image
$\pi_1(p)\left(\pi_1(E,e)\right)$, where $\pi_1$ denotes the fundamental
group functor. Furthermore $\tilde f$ is unique (provided it exists of course).

The following diagrams might be useful: To check
$$\xymatrix{
&{(E,e)}\ar[d]^{p}\\
{(X,x)}\ar[r]_{f}\ar@{-->}[ur]^{?\tilde f}&{(B,b)} }
$$
one only needs to check
$$\xymatrix{
&&{\pi_1(E,e)}\ar[d]^{\pi_1(p)}\\
&&{\pi_1(p)\left(\pi_1(E,e)\right)}\ar[d]^{\subset}\\
{\pi_1(X,x)}\ar[r]^-{\pi_1(f)}&{\pi_1(f)\left(\pi_1(X,x)\right)}\ar@{-->}[ur]^{?}\ar[r]_{\quad\subset}& {\pi_1(B,b)} }
$$

\begin{cor}
  Every map from a simply connected space $X$ lifts. In particular:
  \begin{enumerate}
  \item a path $\Gg\co I\to B$ lifts, 
  \item a homotopy of paths $H\co I\times I\to B$ lifts, and
  \item a map $\Gs\co S^n\to B$, lifts if $n\geq 2$. 
  \end{enumerate}
\end{cor}

Note that (3) is not true for $n=1$ because the circle is not simply
connected.  So although by (1) every closed path in $B$ lifts to a path in
$E$  it does not
necessarily lift to a closed path.
%%%%%
%%%%%
\end{document}

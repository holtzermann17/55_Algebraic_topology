\documentclass[12pt]{article}
\usepackage{pmmeta}
\pmcanonicalname{DeckTransformation}
\pmcreated{2013-03-22 13:26:49}
\pmmodified{2013-03-22 13:26:49}
\pmowner{mathcam}{2727}
\pmmodifier{mathcam}{2727}
\pmtitle{deck transformation}
\pmrecord{17}{34010}
\pmprivacy{1}
\pmauthor{mathcam}{2727}
\pmtype{Definition}
\pmcomment{trigger rebuild}
\pmclassification{msc}{55R05}
\pmsynonym{covering transformation}{DeckTransformation}
\pmrelated{PoperlyDiscontinuousAction}
\pmrelated{ClassificationOfCoveringSpaces}
\pmdefines{deck transformation}
\pmdefines{covering transformation}
\pmdefines{equivalence}
\pmdefines{self equivalence}
\pmdefines{self-equivalence.}
\pmdefines{Galois group of a cover}

\endmetadata

%\documentclass{amsart}
\usepackage{amsmath}
\usepackage[all,poly,knot,dvips]{xy}
%\usepackage{pstricks,pst-poly,pst-node,pstcol}


\usepackage{amssymb,latexsym}

\usepackage{amsthm,latexsym}
\usepackage{eucal,latexsym}

% THEOREM Environments --------------------------------------------------

\newtheorem{thm}{Theorem}
 \newtheorem*{mainthm}{Main~Theorem}
 \newtheorem{cor}[thm]{Corollary}
 \newtheorem{lem}[thm]{Lemma}
 \newtheorem{prop}[thm]{Proposition}
 \newtheorem{claim}[thm]{Claim}
 \theoremstyle{definition}
 \newtheorem{defn}[thm]{Definition}
 \theoremstyle{remark}
 \newtheorem{rem}[thm]{Remark}
 \numberwithin{equation}{subsection}


%---------------------  Greek letters, etc ------------------------- 

\newcommand{\CA}{\mathcal{A}}
\newcommand{\CC}{\mathcal{C}}
\newcommand{\CM}{\mathcal{M}}
\newcommand{\CP}{\mathcal{P}}
\newcommand{\CS}{\mathcal{S}}
\newcommand{\BC}{\mathbb{C}}
\newcommand{\BN}{\mathbb{N}}
\newcommand{\BR}{\mathbb{R}}
\newcommand{\BZ}{\mathbb{Z}}
\newcommand{\FF}{\mathfrak{F}}
\newcommand{\FL}{\mathfrak{L}}
\newcommand{\FM}{\mathfrak{M}}
\newcommand{\Ga}{\alpha}
\newcommand{\Gb}{\beta}
\newcommand{\Gg}{\gamma}
\newcommand{\GG}{\Gamma}
\newcommand{\Gd}{\delta}
\newcommand{\GD}{\Delta}
\newcommand{\Ge}{\varepsilon}
\newcommand{\Gz}{\zeta}
\newcommand{\Gh}{\eta}
\newcommand{\Gq}{\theta}
\newcommand{\GQ}{\Theta}
\newcommand{\Gi}{\iota}
\newcommand{\Gk}{\kappa}
\newcommand{\Gl}{\lambda}
\newcommand{\GL}{\Lamda}
\newcommand{\Gm}{\mu}
\newcommand{\Gn}{\nu}
\newcommand{\Gx}{\xi}
\newcommand{\GX}{\Xi}
\newcommand{\Gp}{\pi}
\newcommand{\GP}{\Pi}
\newcommand{\Gr}{\rho}
\newcommand{\Gs}{\sigma}
\newcommand{\GS}{\Sigma}
\newcommand{\Gt}{\tau}
\newcommand{\Gu}{\upsilon}
\newcommand{\GU}{\Upsilon}
\newcommand{\Gf}{\varphi}
\newcommand{\GF}{\Phi}
\newcommand{\Gc}{\chi}
\newcommand{\Gy}{\psi}
\newcommand{\GY}{\Psi}
\newcommand{\Gw}{\omega}
\newcommand{\GW}{\Omega}
\newcommand{\Gee}{\epsilon}
\newcommand{\Gpp}{\varpi}
\newcommand{\Grr}{\varrho}
\newcommand{\Gff}{\phi}
\newcommand{\Gss}{\varsigma}
\newcommand{\Au}{\operatorname{Aut}}
\def\co{\colon\thinspace}
\begin{document}
Let $p\co E\to X$ be a covering map.  A \emph{deck transformation} or
\emph{covering transformation} is a map $D\co E\to E$ such that $p\circ
D=p$, that is, such that the following diagram commutes.
$$\xymatrix{ {E}\ar[dr]_{p}\ar[rr]^{D}&&{E}\ar[dl]^{p}\\
          &{X}&  }$$

It is straightforward to check that the set of deck transformations is closed
under compositions and the operation of taking inverses. Therefore the set
of deck transformations is a subgroup of the group of homeomorphisms of $E$. 
This group  will be denoted by $\Au(p)$ and referred
to as the \emph{group of deck transformations} or as the \emph{automorphism
group} of $p$.  It is worth noting that an alternative name for the group of deck transformations is the \emph{Galois group} of the covering.  This terminology arises from an analogy with the fundamental theorem of Galois theory which gives the inclusion-reversing identification addressed in the classification of covering spaces.

In the more general context of fiber bundles deck transformations correspond
to \emph{isomorphisms over the identity} since the above diagram could be
expanded to:
  $$\xymatrix{ {E}\ar[d]_{p}\ar[r]^{D}&{E}\ar[d]^{p}\\
               {X}\ar[r]_{\text{id}}& {X}  }$$ 
An isomorphism not necessarily over the identity is called an
\emph{equivalence}. In other words an equivalence between two covering maps    
 $p\co E\to X$ and  $p'\co E'\to X'$ is a pair of maps $(\tilde f,f)$ that make the
 following diagram commute
$$\xymatrix{ {E'}\ar[d]_{p'}\ar[r]^{\tilde f}&{E}\ar[d]^{p}\\
               {X'}\ar[r]_{f}& {X}  }$$ 
 i.e. such that $p\circ\tilde f=f\circ p'$.

Deck transformations should be perceived as the symmetries of $p$ (hence the
notation $\Au(p)$), and therefore they should be expected to preserve any
concept 
that is defined in
terms of $P$. Most of what follows is an instance of this meta--principle.


\section*{Properties of deck transformations}
For this section  we assume that  the total space $E$ is connected and locally
path connected.
Notice that a deck transformation is a lifting of $p\co E\to X$ and
therefore (according to the lifting theorem) it is uniquely determined by
the image of a point. In other words:
\begin{prop}\label{prop:uniq}
  Let  $D_1,D_2\in \Au(p)$. If there is  $e\in E$ such that $D_1(e)=D_2(e)$
  then $D_1=D_2$. In particular if $D_1(e)=e$ for some $e\in E$ then
  $D_1=\text{id}$.  
\end{prop}

Another simple (or should I say double?) application of the lifting theorem gives

\begin{prop}\label{prop:exist}
  Given $e,e' \in E$ with $p(e)=p(e')$, there is a $D\in \Au(p)$ such that $D(e)=e'$ if and
  only if $p_*\left(\pi_1(E,e)\right)=p_*\left(\pi_1(E,e')\right)$, where
  $p_*$  denotes $\pi_1(p)$. 
\end{prop}


\begin{prop}
  Deck transformations commute with the monodromy action. That is if $*\in
  X$, $e\in p^{-1}(*)$, $\Gg\in \pi_1(X,*)$ and $D\in \Au(p)$ then 
$$D(x\cdot\Gg)=D(x)\cdot\Gg$$
where $\cdot$ denotes the monodromy action.
\end{prop}
\begin{proof}
  If $\tilde \Gg$ is a lifting of $\Gg$ starting at $e$, then
  $D\circ\tilde\Gg$  is a lifting of $\Gg$ staring at $D(e)$.
\end{proof}

We simplify notation by using $\pi_e$ to denote the fundamental group
$\pi_1(E,e)$ for $e\in E$.

\begin{thm} For all  $e\in E$ 
  $$\Au(p)\cong N\left(p_*(\pi_e)\right)/p_*(\pi_e)$$
where,  $ N(p_*\pi_e)$ denotes the normalizer of  $p_*\pi_e$ inside
$\pi_1\left(X,p(e)\right)$. 
\end{thm}

\begin{proof}
  Denote $N(p_*\pi_e)$ by $N$.  Note that if  $\Gg\in N$ then
  $p_*(\pi_{e\cdot\Gg})=p_*(\pi_e)$. Indeed, recall that
  $p_*(\pi_e)$ is the stabilizer of $e$ under the momodromy
  action and therefore we have  
 $$ p*(\pi_{e\cdot\Gg})=
 \operatorname{Stab}(e\cdot\Gg)=\Gg\operatorname{Stab}(e)\Gg^{-1}=\Gg p_*
 (\pi_e)\Gg^{-1}= p_* (\pi_e)$$
where, the last equality follows from the definition of normalizer.
One can then
define a map  
$$\Gf\co N\to \Au(p)$$
as follows: For $\Gg\in N$ let $\Gf(\Gg)$ be the deck transformation that
maps $e$ to $e\cdot\Gg$. Notice that Proposition~\ref{prop:exist} ensures
the existence of such a deck transformation while
Proposition~\ref{prop:uniq} guarantees its uniqueness. Now
\begin{itemize}
\item $\Gf$ is a homomorphism.\newline
Indeed $\Gf(\Gg_1\Gg_2)$ and  $\Gf(\Gg_1)\circ\Gf(\Gg_2)$ are deck
transformations that map $e$ to $e\cdot(\Gg_1\Gg_2)$.
\item $\Gf$ is onto.\newline
Indeed given $D\in \Au(p)$ since $E$ is path connected one can find a path
$\alpha$ in $E$ connecting $e$ and $D(e)$. Then $p\circ\alpha$ is a loop in
$X$ and $D=\Gf(p\circ\alpha)$.
\item $\ker(\Gf)=p_*(\pi_e)$. \newline
 Obvious.

Therefore the theorem follows by the first isomorphism theorem.
\end{itemize}
   
\end{proof}

\begin{cor}
  If $p$ is regular covering then
$$\Au(p)\cong \pi_1(X,*)/p_*\left(\pi_1(E,e)\right).$$
\end{cor}

\begin{cor}
  If $p$ is the universal cover then 
$$\Au(p)\cong \pi_1(X,*)\,.$$
\end{cor}
%%%%%
%%%%%
\end{document}

\documentclass[12pt]{article}
\usepackage{pmmeta}
\pmcanonicalname{pathConnectnessAsAHomotopyInvariant}
\pmcreated{2013-03-22 18:02:15}
\pmmodified{2013-03-22 18:02:15}
\pmowner{joking}{16130}
\pmmodifier{joking}{16130}
\pmtitle{(path) connectness as a homotopy invariant}
\pmrecord{8}{40558}
\pmprivacy{1}
\pmauthor{joking}{16130}
\pmtype{Theorem}
\pmcomment{trigger rebuild}
\pmclassification{msc}{55P10}
\pmrelated{Homotopy}
\pmrelated{homotopyequivalence}
\pmrelated{path}
\pmrelated{connectedspace}

\endmetadata

% this is the default PlanetMath preamble.  as your knowledge
% of TeX increases, you will probably want to edit this, but
% it should be fine as is for beginners.

% almost certainly you want these
\usepackage{amssymb}
\usepackage{amsmath}
\usepackage{amsfonts}

% used for TeXing text within eps files
%\usepackage{psfrag}
% need this for including graphics (\includegraphics)
%\usepackage{graphicx}
% for neatly defining theorems and propositions
%\usepackage{amsthm}
% making logically defined graphics
%%%\usepackage{xypic}

% there are many more packages, add them here as you need them

% define commands here

\begin{document}
\textbf{Theorem.} Let $X$ and $Y$ be arbitrary topological spaces with $Y$ (path) connected. If there are maps $f:X\rightarrow Y$ and $g:Y\rightarrow X$ such that $g\circ f:X\rightarrow X$ is homotopic to the identity map, then $X$ is (path) connected.\\ \\
\textbf{\textit{Proof:}} Let $f:X\rightarrow Y$ and $g:Y\rightarrow X$ be maps satisfying theorem's assumption. Furthermore let $X=\bigcup X_i$ be a decomposition of $X$ into (path) connected components. Since $Y$ is (path) connected, then $g(Y)\subseteq X_i$ for some $i$. Thus $(g\circ f)(X)\subseteq X_i$. Now let $H:I\times X\rightarrow X$ be the homotopy from $g\circ f$ to the identity map. Let $\alpha_{x}:I\rightarrow X$ be a path defined by the formula: $\alpha_{x}(t)=H(t,x)$. Since for all $x\in X$ we have $\alpha_{x}(0)\in X_i$ and $I$ is path connected, then $\alpha_{x}(I)\subseteq X_i$. Therefore $H(I\times X)\subseteq X_i$, but $H(\{1\}\times X)=X$ which implies that $X_i=X$, so $X$ is (path) connected. $\square$ \\ \\
Straightforward application of this theorem is following:\\ \\
\textbf{Corollary.} Let $X$ and $Y$ be homotopy equivalent spaces. Then $X$ is (path) connected if and only if $Y$ is (path) connected.
%%%%%
%%%%%
\end{document}

\documentclass[12pt]{article}
\usepackage{pmmeta}
\pmcanonicalname{4SurfaceBundles}
\pmcreated{2013-03-22 16:01:40}
\pmmodified{2013-03-22 16:01:40}
\pmowner{juanman}{12619}
\pmmodifier{juanman}{12619}
\pmtitle{4 surface bundles}
\pmrecord{12}{38070}
\pmprivacy{1}
\pmauthor{juanman}{12619}
\pmtype{Feature}
\pmcomment{trigger rebuild}
\pmclassification{msc}{55R10}
\pmrelated{SurfaceBundleOverTheCircle}

\endmetadata

% this is the default PlanetMath preamble.  as your knowledge
% of TeX increases, you will probably want to edit this, but
% it should be fine as is for beginners.

% almost certainly you want these
\usepackage{amssymb}
\usepackage{amsmath}
\usepackage{amsfonts}

% used for TeXing text within eps files
%\usepackage{psfrag}
% need this for including graphics (\includegraphics)
%\usepackage{graphicx}
% for neatly defining theorems and propositions
%\usepackage{amsthm}
% making logically defined graphics
%%%\usepackage{xypic}

% there are many more packages, add them here as you need them

% define commands here

\begin{document}
\centerline{\bf Four Kleinbottle bundles $K\subset M\to S^1$.}

There are four because the extended mapping class group  for the genus two, non orientable surface $K$ the Klein bottle, is ${\mathbb{Z}}_2\oplus{\mathbb{Z}}_2$.

This group is generated by a Dehn-twist $\tau$ about the unique two-sided curve in $K$  and by the y-homeomorphism, both representing two isotopy classes of order two. 

These bundles are\\

\begin{itemize}
\item $K\times S^1$, the trivial Cartesian product
\item $K\times_{\tau}S^1$,
\item $K\times_yS^1=K \stackrel{\sim}\times I^O \cup_{(0,1)} M\ddot{o}\times S^1 $, 

\item $K\times_{y\tau}S^1$.
\end{itemize}
 
Where $K\stackrel{\sim}\times I^O$ is the orientable twisted $I$-bundle over $K$, among the three $I$-bundles over $K$.The symbol 
$\cup_{(0,1)}$ is used to indicate that, the meridian in $\partial(M\ddot{o}\times S^1)$ is attached to the meridian of $\partial(K\stackrel{\sim}\times I ^O)$, both 2-tori. $M\ddot{o}$ is the M\"obius band.
 
Now, since those monodromies are periodic then they are also homeomorphic respectively to the Seifert fiber spaces
\begin{itemize}
\item $(NnI,2|0)=K\times S^1$,
\item $(NnI,2|1)=(K\times S^1\setminus{\rm int} W)\cup_{(1,1)}W$,
\item $(NnII,2|0)=K\times_yS^1=K \stackrel{\sim}\times I^O \cup_{(0,1)} M\ddot{o}\times S^1$ and
\item $(NnII,2|1)=(K\times_y S^1\setminus{\rm int} W)\cup_{(1,1)}W$
\end{itemize}

Where $W$ is a solid torus in the space and $\cup_{(1,1)}$ is the Dehn surgery: meridian of $\partial W$ to the longitude of $\partial(K\times S^1\setminus{\rm int} W)$.

The non trivial homeomorphisms were given by Per Orlik and Frank Raymond, in 1969.






%%%%%
%%%%%
\end{document}

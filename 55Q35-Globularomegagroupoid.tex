\documentclass[12pt]{article}
\usepackage{pmmeta}
\pmcanonicalname{Globularomegagroupoid}
\pmcreated{2013-03-22 19:21:02}
\pmmodified{2013-03-22 19:21:02}
\pmowner{bci1}{20947}
\pmmodifier{bci1}{20947}
\pmtitle{globular $\omega$-groupoid}
\pmrecord{42}{42300}
\pmprivacy{1}
\pmauthor{bci1}{20947}
\pmtype{Definition}
\pmcomment{trigger rebuild}
\pmclassification{msc}{55Q35}
\pmclassification{msc}{55Q05}
\pmclassification{msc}{20L05}
\pmclassification{msc}{18D05}
\pmclassification{msc}{18-00}
%\pmkeywords{n-category with invertible morphisms}
%\pmkeywords{HDA}
%\pmkeywords{NAAT}
%\pmkeywords{NA-QAT}
%\pmkeywords{n-groupoids}
%\pmkeywords{2-category}
%\pmkeywords{double groupoid}
%\pmkeywords{multiple groupoid}
\pmdefines{filtered space}
\pmdefines{$G^n$}
\pmdefines{$n$-globe}
\pmdefines{fundamental globular $\omega$-groupoid of a filtered topological space}

\endmetadata

% this is the default PlanetPhysics preamble. 

% almost certainly you want these
\usepackage{amsmath, amssymb, amsfonts, amsthm, amscd, latexsym, enumerate}
\usepackage{xypic, xspace}
\usepackage[mathscr]{eucal}
\usepackage[dvips]{graphicx}
\usepackage[curve]{xy}
% define commands here
\theoremstyle{plain}
\newtheorem{lemma}{Lemma}[section]
\newtheorem{proposition}{Proposition}[section]
\newtheorem{theorem}{Theorem}[section]
\newtheorem{corollary}{Corollary}[section]
\theoremstyle{definition}
\newtheorem{definition}{Definition}[section]
\newtheorem{example}{Example}[section]
%\theoremstyle{remark}
\newtheorem{remark}{Remark}[section]
\newtheorem*{notation}{Notation}
\newtheorem*{claim}{Claim}
\renewcommand{\thefootnote}{\ensuremath{\fnsymbol{footnote}}}
\numberwithin{equation}{section}
\newcommand{\Ad}{{\rm Ad}}
\newcommand{\Aut}{{\rm Aut}}
\newcommand{\Cl}{{\rm Cl}}
\newcommand{\Co}{{\rm Co}}
\newcommand{\DES}{{\rm DES}}
\newcommand{\Diff}{{\rm Diff}}
\newcommand{\Dom}{{\rm Dom}}
\newcommand{\Hol}{{\rm Hol}}
\newcommand{\Mon}{{\rm Mon}}
\newcommand{\Hom}{{\rm Hom}}
\newcommand{\Ker}{{\rm Ker}}
\newcommand{\Ind}{{\rm Ind}}
\newcommand{\IM}{{\rm Im}}
\newcommand{\Is}{{\rm Is}}
\newcommand{\ID}{{\rm id}}
\newcommand{\grpL}{{\rm GL}}
\newcommand{\Iso}{{\rm Iso}}
\newcommand{\rO}{{\rm O}}
\newcommand{\Sem}{{\rm Sem}}
\newcommand{\SL}{{\rm Sl}}
\newcommand{\St}{{\rm St}}
\newcommand{\Sym}{{\rm Sym}}
\newcommand{\Symb}{{\rm Symb}}
\newcommand{\SU}{{\rm SU}}
\newcommand{\Tor}{{\rm Tor}}
\newcommand{\U}{{\rm U}}
\newcommand{\A}{\mathcal A}
\newcommand{\Ce}{\mathcal C}
\newcommand{\D}{\mathcal D}
\newcommand{\E}{\mathcal E}
\newcommand{\F}{\mathcal F}
%\newcommand{\grp}{\mathcal G}
\renewcommand{\H}{\mathcal H}
\renewcommand{\cL}{\mathcal L}
\newcommand{\Q}{\mathcal Q}
\newcommand{\R}{\mathcal R}
\newcommand{\cS}{\mathcal S}
\newcommand{\cU}{\mathcal U}
\newcommand{\W}{\mathcal W}
\newcommand{\bA}{\mathbb{A}}
\newcommand{\bB}{\mathbb{B}}
\newcommand{\bC}{\mathbb{C}}
\newcommand{\bD}{\mathbb{D}}
\newcommand{\bE}{\mathbb{E}}
\newcommand{\bF}{\mathbb{F}}
\newcommand{\bG}{\mathbb{G}}
\newcommand{\bK}{\mathbb{K}}
\newcommand{\bM}{\mathbb{M}}
\newcommand{\bN}{\mathbb{N}}
\newcommand{\bO}{\mathbb{O}}
\newcommand{\bP}{\mathbb{P}}
\newcommand{\bR}{\mathbb{R}}
\newcommand{\bV}{\mathbb{V}}
\newcommand{\bZ}{\mathbb{Z}}
\newcommand{\bfE}{\mathbf{E}}
\newcommand{\bfX}{\mathbf{X}}
\newcommand{\bfY}{\mathbf{Y}}
\newcommand{\bfZ}{\mathbf{Z}}
\renewcommand{\O}{\Omega}
\renewcommand{\o}{\omega}
\newcommand{\vp}{\varphi}
\newcommand{\vep}{\varepsilon}
\newcommand{\diag}{{\rm diag}}
\newcommand{\grp}{{\mathsf{G}}}
\newcommand{\dgrp}{{\mathsf{D}}}
\newcommand{\desp}{{\mathsf{D}^{\rm{es}}}}
\newcommand{\grpeod}{{\rm Geod}}
%\newcommand{\grpeod}{{\rm geod}}
\newcommand{\hgr}{{\mathsf{H}}}
\newcommand{\mgr}{{\mathsf{M}}}
\newcommand{\ob}{{\rm Ob}}
\newcommand{\obg}{{\rm Ob(\mathsf{G)}}}
\newcommand{\obgp}{{\rm Ob(\mathsf{G}')}}
\newcommand{\obh}{{\rm Ob(\mathsf{H})}}
\newcommand{\Osmooth}{{\Omega^{\infty}(X,*)}}
\newcommand{\grphomotop}{{\rho_2^{\square}}}
\newcommand{\grpcalp}{{\mathsf{G}(\mathcal P)}}
\newcommand{\rf}{{R_{\mathcal F}}}
\newcommand{\grplob}{{\rm glob}}
\newcommand{\loc}{{\rm loc}}
\newcommand{\TOP}{{\rm TOP}}
\newcommand{\wti}{\widetilde}
\newcommand{\what}{\widehat}
\renewcommand{\a}{\alpha}
\newcommand{\be}{\beta}
\newcommand{\grpa}{\grpamma}
%\newcommand{\grpa}{\grpamma}
\newcommand{\de}{\delta}
\newcommand{\del}{\partial}
\newcommand{\ka}{\kappa}
\newcommand{\si}{\sigma}
\newcommand{\ta}{\tau}
\newcommand{\lra}{{\longrightarrow}}
\newcommand{\ra}{{\rightarrow}}
\newcommand{\rat}{{\rightarrowtail}}
\newcommand{\ovset}[1]{\overset {#1}{\ra}}
\newcommand{\ovsetl}[1]{\overset {#1}{\lra}}
\newcommand{\hr}{{\hookrightarrow}}
\newcommand{\<}{{\langle}}
\def\baselinestretch{1.1}
\hyphenation{prod-ucts}
\newcommand{\sqdiagram}[9]{$$ \diagram #1 \rto^{#2} \dto_{#4}&
#3 \dto^{#5} \\ #6 \rto_{#7} & #8 \enddiagram
\eqno{\mbox{#9}}$$ }
\def\C{C^{\ast}}
\newcommand{\labto}[1]{\stackrel{#1}{\longrightarrow}}
\newcommand{\quadr}[4]
{\begin{pmatrix} & #1& \\[-1.1ex] #2 & & #3\\[-1.1ex]& #4&
\end{pmatrix}}
\def\D{\mathsf{D}}

\begin{document}
\begin{definition}
An $\omega$-groupoid has a distinct meaning from that of $\omega$-category, although certain authors restrict its definition to the latter by adding the restriction of invertible morphisms, and thus also assimilate the $\omega$-groupoid with the $\infty$-groupoid.  Ronald Brown and Higgins showed in 1981 that $\infty$-groupoids and crossed complexes are {\em equivalent}. Subsequently,in 1987, these authors introduced the tensor products and homotopies for $\omega$-groupoids and crossed complexes. ``{\em It is because the geometry of convex sets is so much more complicated in dimensions $>1$ than in dimension $1$ that new complications emerge for the theories of higher order group theory and of higher homotopy groupoids.''}

However, in order to introduce a precise and useful definition of globular $\omega$-groupoids one needs to define first the $n$-globe $G^n$ which is the subspace of an Euclidean $n$-space $R^n$ of points $x$ such that that their norm 
$||x|| \leq 1$, but with the cell structure for $n \geq 1$ specified in Section {\bf 1} of R. Brown (2007). Also, one needs to consider a {\em filtered space} that is defined as a compactly generated space $X_{\infty}$ and a sequence of subspaces $X_*$. Then, the $n$-globe $G^n$ has a skeletal filtration giving a filtered space ${G^n}_*$.

Thus, a {\em fundamental globular {\em $\omega$-groupoid} of a filtered (topological) space} is defined by using an $n$-globe with its skeletal filtration (R. Brown, 2007 available from: arXiv:math/0702677v1 [math.AT]). This is analogous to the fundamental cubical omega--groupoid of Ronald Brown and Philip Higgins (1981a-c) that relates the construction to the fundamental crossed complex of a filtered space. Thus, as shown in R. Brown (2007: http://arxiv.org/abs/math/0702677), the crossed complex associated to the free globular omega-groupoid on one element of dimension $n$ is the fundamental crossed complex of the $n$-globe.

{\bf more to come... entry in progress}
\end{definition}

\begin{remark}

 An important reason for studying $n$--categories, and especially 
$n$-groupoids, is to use them as coefficient objects for non-Abelian Cohomology theories. Thus, some double groupoids defined over Hausdorff spaces that are non-Abelian (or non-commutative) are relevant to non-Abelian Algebraic Topology (NAAT) and \PMlinkexternal{NAQAT (or NA-QAT)}{http://planetphysics.org/?op=getobj&from=lec&id=61}.

 Furthermore, whereas the definition of an $n$-groupoid is a straightforward generalization of a 2-groupoid, the notion of a \emph{multiple groupoid} is not at all an obvious generalization or extension of the concept of double groupoid. 
\end{remark}

\begin{thebibliography}{9}
\bibitem{BH81a} 
Brown, R. and Higgins, P.J. (1981). The algebra of cubes. J. Pure Appl. Alg. 21 : 233--260.
\bibitem{BH81b} 
Brown, R. and Higgins, P. J. Colimit theorems for relative homotopy groups. J.Pure Appl. Algebra 22 (1) (1981) 11--41.
\bibitem{BH81c} Brown, R. and Higgins, P. J. The equivalence of $\infty$-groupoids and crossed complexes. Cahiers Topologie G$\'e$om. Diff$\'{e}$rentielle 22 (4) (1981) 371--386.

\bibitem{BHR2}
Brown, R., Higgins, P. J. and R. Sivera,: 2011. {\em ``Non-Abelian Algebraic Topology''}, EMS Publication.\\
http://www.bangor.ac.uk/~mas010/nonab-a-t.html ; \\
http://www.bangor.ac.uk/~mas010/nonab-t/partI010604.pdf

\bibitem{BJk4}
Brown, R. and G. Janelidze: 2004. Galois theory and a new homotopy double groupoid of a map of spaces, \emph{Applied Categorical Structures} \textbf{12}: 63-80.

\end{thebibliography}

%%%%%
%%%%%
\end{document}

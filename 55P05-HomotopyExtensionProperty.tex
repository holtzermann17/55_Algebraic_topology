\documentclass[12pt]{article}
\usepackage{pmmeta}
\pmcanonicalname{HomotopyExtensionProperty}
\pmcreated{2013-03-22 12:13:36}
\pmmodified{2013-03-22 12:13:36}
\pmowner{RevBobo}{4}
\pmmodifier{RevBobo}{4}
\pmtitle{homotopy extension property}
\pmrecord{15}{31600}
\pmprivacy{1}
\pmauthor{RevBobo}{4}
\pmtype{Definition}
\pmcomment{trigger rebuild}
\pmclassification{msc}{55P05}
\pmrelated{Cofibration}
\pmrelated{HomotopyLiftingProperty}

% this is the default PlanetMath preamble.  as your knowledge
% of TeX increases, you will probably want to edit this, but
% it should be fine as is for beginners.

% almost certainly you want these
\usepackage{amssymb}
\usepackage{amsmath}
\usepackage{amsfonts}

% used for TeXing text within eps files
%\usepackage{psfrag}
% need this for including graphics (\includegraphics)
%\usepackage{graphicx}
% for neatly defining theorems and propositions
%\usepackage{amsthm}
% making logically defined graphics
\usepackage[all]{xypic} 

% there are many more packages, add them here as you need them

% define commands here
\begin{document}
Let $X$ be a topological space and $A$ a subspace of $X$. Suppose there is a continuous map $f:X \to Y$ and a homotopy of maps $F: A \times I \to Y$. The inclusion map $i: A \to X$ is said to have the \emph{homotopy extension property} if there exists a continuous map $F^{'}$ such that the following diagram commutes:
$$
\SelectTips{eu}{}
\xymatrix{
& A \ar[d]^{i} \ar[r]^{i_{0}} & A \times I \ar[d]^{F} \ar@/^1.5pc/[dd]^{i \times id_{I}} \\
& X \ar[r]^{f} \ar[dr]^{i_{0}} & Y \\
& &  X \times I \ar[u]_{F^{'}} 
}
$$
Here, $i_0=(x,0)$ for all $ x \in X$.
%%%%%
%%%%%
%%%%%
\end{document}

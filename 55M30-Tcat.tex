\documentclass[12pt]{article}
\usepackage{pmmeta}
\pmcanonicalname{Tcat}
\pmcreated{2013-03-22 15:54:54}
\pmmodified{2013-03-22 15:54:54}
\pmowner{juanman}{12619}
\pmmodifier{juanman}{12619}
\pmtitle{t-cat}
\pmrecord{7}{37919}
\pmprivacy{1}
\pmauthor{juanman}{12619}
\pmtype{Definition}
\pmcomment{trigger rebuild}
\pmclassification{msc}{55M30}
\pmrelated{RoundComplexity}

\endmetadata

% this is the default PlanetMath preamble.  as your knowledge
% of TeX increases, you will probably want to edit this, but
% it should be fine as is for beginners.

% almost certainly you want these
\usepackage{amssymb}
\usepackage{amsmath}
\usepackage{amsfonts}

% used for TeXing text within eps files
%\usepackage{psfrag}
% need this for including graphics (\includegraphics)
%\usepackage{graphicx}
% for neatly defining theorems and propositions
%\usepackage{amsthm}
% making logically defined graphics
%%%\usepackage{xypic}

% there are many more packages, add them here as you need them

% define commands here

\begin{document}
The {\bf t-cat} of a topological space $X$ is the minimal number of open sets that cover $X$ such that each open set in the cover has the homotopy type of the unit circle $S^1$.  This means that for each open set $U$, the inclusion $U\stackrel{i}\hookrightarrow X$ is homotopic to some factorization $U\stackrel{a}\to S^1\stackrel{b}\to X$, i.e.
$$i\simeq b\circ a.$$
 
When $X$ is manifold, this is related to the round complexity of $X$.

\begin{thebibliography}{9}
\bibitem{dsgk} D. Siersma, G. Khimshiasvili, {\it On minimal round functions}, Preprint 1118, Department of Mathematics, Utrecht University, 1999, pp. 18.
\end{thebibliography}

%%%%%
%%%%%
\end{document}

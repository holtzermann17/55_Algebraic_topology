\documentclass[12pt]{article}
\usepackage{pmmeta}
\pmcanonicalname{TheTopologistsSineCurveHasTheFixedPointProperty}
\pmcreated{2013-03-22 16:59:37}
\pmmodified{2013-03-22 16:59:37}
\pmowner{Mathprof}{13753}
\pmmodifier{Mathprof}{13753}
\pmtitle{the topologist's sine curve has the fixed point property}
\pmrecord{6}{39274}
\pmprivacy{1}
\pmauthor{Mathprof}{13753}
\pmtype{Proof}
\pmcomment{trigger rebuild}
\pmclassification{msc}{55M20}
\pmclassification{msc}{54H25}
\pmclassification{msc}{47H10}

\endmetadata

% this is the default PlanetMath preamble.  as your knowledge
% of TeX increases, you will probably want to edit this, but
% it should be fine as is for beginners.

% almost certainly you want these
\usepackage{amssymb}
\usepackage{amsmath}
\usepackage{amsfonts}

% used for TeXing text within eps files
%\usepackage{psfrag}
% need this for including graphics (\includegraphics)
%\usepackage{graphicx}
% for neatly defining theorems and propositions
%\usepackage{amsthm}
% making logically defined graphics
%%%\usepackage{xypic}

% there are many more packages, add them here as you need them

% define commands here

\begin{document}
The typical example of a connected space that is not path connected (the topologist's sine curve) has the fixed point property.

Let $X_1 = \{0\} \times [-1,1]$ and $X_2 = \{(x,\sin(1/x)) : 0<x \le 1\}$, and $X = X_1 \cup X_2$.

If $f:X \to X$ is a continuous map, then since $X_1$ and $X_2$ are both path connected, the image of each one of them must be entirely contained in another of them.

If $f(X_1)\subset X_1$, then $f$ has a fixed point because the interval has the fixed point property.
If $f(X_2) \subset X_1$, then $f(X) = f(cl(X_2)) \subset cl(f(X_2)) \subset X_1$, and in particular 
$f(X_1) \subset X_1 $and again $f$ has a fixed point.

So the only case that remains is that $f(X)\subset X_2$. And since $X$ is compact, its projection to the first coordinate is also compact so that it must be an interval $[a,b]$ with $a>0$. Thus $f(X)$ is contained in 
$S = \{(x, \sin (1/x)) : x\in [a,b]\}$. But $S$ is homeomorphic to a closed interval, so that it has the fixed point property, and 
the restriction of $f$ to $S$ is a continuous map $S \to S$, so that it has a fixed point.

This proof is due to Koro.

%%%%%
%%%%%
\end{document}

\documentclass[12pt]{article}
\usepackage{pmmeta}
\pmcanonicalname{SpernersLemma}
\pmcreated{2013-03-22 13:44:33}
\pmmodified{2013-03-22 13:44:33}
\pmowner{mathcam}{2727}
\pmmodifier{mathcam}{2727}
\pmtitle{Sperner's lemma}
\pmrecord{12}{34436}
\pmprivacy{1}
\pmauthor{mathcam}{2727}
\pmtype{Theorem}
\pmcomment{trigger rebuild}
\pmclassification{msc}{55M20}

\endmetadata

\usepackage{amssymb}
\usepackage{amsmath}
\usepackage{amsfonts}
\usepackage{graphicx}
\newcommand{\Z}{\mathbb{Z}}
\begin{document}
\PMlinkescapeword{connected} \PMlinkescapeword{opposite}
\PMlinkescapeword{contour integral}
\PMlinkescapeword{contour integrals}
\PMlinkescapeword{combinations} \PMlinkescapeword{colors}
\PMlinkescapeword{size} \PMlinkescapeword{adjacent}
\PMlinkescapeword{equivalent} \PMlinkescapeword{contains}
Let $ABC$ be a triangle, and let $S$ be the set of vertices of some
triangulation $T$ of $ABC$. Let $f$ be a mapping
of $S$ into a three-element set, say $\{1,2,3\}=T$ (indicated by
red/green/blue respectively in the figure), such that:
\begin{itemize}
\item
any point $P$ of $S$, if it is on the side $AB$, satisfies
$f(P)\in\{1,2\}$
\item
similarly if $P$ is on the side $BC$, then $f(P)\in\{2,3\}$
\item
if $P$ is on the side $CA$, then $f(P)\in\{3,1\}$
\end{itemize}
(It follows that $f(A)=1, f(B)=2, f(C)=3$.)
Then some (triangular) simplex of $T$, say $UVW$, satisfies
$$f(U)=1 \qquad f(V)=2 \qquad f(W)=3\;.$$

\begin{center}
\includegraphics{sperner}
\end{center}

We will informally sketch a proof of a stronger statement:
Let $M$ (resp. $N$) be the number of simplexes satisfying (1) and whose
vertices have the same orientation as $ABC$ (resp. the
opposite orientation). Then $M-N=1$ (whence $M>0$).

The proof is in the style of well-known proofs of, for example, Stokes's
theorem in the plane, or Cauchy's theorems about a holomorphic function.

Define an antisymmetric function $d:T\times T\to\mathbb{Z}$ by
$$d(1,1)=d(2,2)=d(3,3)=0$$
$$d(1,2)=d(2,3)=d(3,1)=1$$
$$d(2,1)=d(3,2)=d(1,3)=-1\;.$$

Let's define a ``circuit'' of size $n$ as an injective mapping $z$ of the
cyclic group $\Z/n\Z$ into $V$ such that $z(n)$ is adjacent to $z(n+1)$
for all $n$ in the group.

Any circuit $z$ has what we will call a contour integral $Iz$, namely
$$Iz=\sum_n d(z(n),z(n+1))\;.$$
Let us say that two vertices $P$ and $Q$ are equivalent if $f(P)=f(Q)$.

There are four steps:

1) Contour integrals are added when their corresponding circuits are juxtaposed.

2) A circuit of size 3, hitting the vertices of a simplex $PQR$, has contour integral
\begin{itemize}
\item 0 if any two of $P$, $Q$, $R$ are equivalent, else
\item +3 if they are inequivalent and have the same orientation as $ABC$, else
\item -3
\end{itemize}

3) If $y$ is a circuit which travels around the perimeter of the whole
triangle $ABC$, and with same orientation as $ABC$, then $Iy=3$.

4) Combining the above results, we get
$$3=\sum Iw=3M - 3N$$
where the sum contains one summand for each simplex $PQR$.

\noindent
\textbf{Remarks: }
In the figure, $M=2$ and $N=1$: there are two ``red-green-blue'' simplexes and one blue-green-red. 

With the same hypotheses as in Sperner's lemma, there is such a simplex $UVW$
which is connected (along edges of the triangulation) to
the side $AB$ (resp. $BC$,$CA$)
by a set of vertices $v$ for which $f(v)\in\{1,2\}$ (resp. $\{2,3\}$,
$\{3,1\})$. The figure illustrates that result: one of the red-green-blue
simplexes is connected to the red-green side by a red-green ``curve'',
and to the other two sides likewise.

The original use of Sperner's lemma was in a proof
of Brouwer's fixed point theorem in two dimensions.
%%%%%
%%%%%
\end{document}

\documentclass[12pt]{article}
\usepackage{pmmeta}
\pmcanonicalname{RoundComplexity}
\pmcreated{2013-03-22 15:54:52}
\pmmodified{2013-03-22 15:54:52}
\pmowner{juanman}{12619}
\pmmodifier{juanman}{12619}
\pmtitle{round complexity}
\pmrecord{11}{37918}
\pmprivacy{1}
\pmauthor{juanman}{12619}
\pmtype{Definition}
\pmcomment{trigger rebuild}
\pmclassification{msc}{55M30}
%\pmkeywords{t-cat}
\pmrelated{LusternikSchnirelmannCategory}
\pmrelated{TCat}

% this is the default PlanetMath preamble.  as your knowledge
% of TeX increases, you will probably want to edit this, but
% it should be fine as is for beginners.

\newcommand{\paren}[1]{\left(\begin{array}{c} #1 \end{array}\right) }

% almost certainly you want these
\usepackage{amssymb}
\usepackage{amsmath}
\usepackage{amsfonts}

% used for TeXing text within eps files
%\usepackage{psfrag}
% need this for including graphics (\includegraphics)
%\usepackage{graphicx}
% for neatly defining theorems and propositions
%\usepackage{amsthm}
% making logically defined graphics
%%%\usepackage{xypic}

% there are many more packages, add them here as you need them

% define commands here

\begin{document}
Mimicking the Lusternik-Schnirelmann category invariant for a smooth manifold $M$ we can ask about the minimal number of critical loops of smooth scalar maps $M\to\mathbb{R}$ which are round functions, that is functions whose critical points are aligned in  a disjoint union of closed curves (a link). 

This number is called the {\bf round complexity} of $M$ and it is symbolized as ${\rm roc}(M)$

Then 
$${\rm roc}(M)=\min\#\{\mbox{critical loops of\quad}f\quad|\quad f\colon M \to\mathbb{R}\quad \mbox{is round function}\}$$

This concept is related to the invariant called t-cat.

Theorem 1: {\it The round complexity for the 2-torus and the Klein bottle is two; all the other closed surfaces have a round complexity of three.} 

Theorem 2: {\it For each closed manifold, $t-cat\le roc$}


{\bf Bibliography}

D. Siersma, G. Khimshiasvili, On minimal round functions, Preprint 1118, Department of Mathematics, Utrecht University, 1999, pp. 18. 



%%%%%
%%%%%
\end{document}

\documentclass[12pt]{article}
\usepackage{pmmeta}
\pmcanonicalname{CupProduct}
\pmcreated{2013-03-22 15:37:42}
\pmmodified{2013-03-22 15:37:42}
\pmowner{whm22}{2009}
\pmmodifier{whm22}{2009}
\pmtitle{cup product}
\pmrecord{7}{37554}
\pmprivacy{1}
\pmauthor{whm22}{2009}
\pmtype{Definition}
\pmcomment{trigger rebuild}
\pmclassification{msc}{55N45}
%\pmkeywords{homology}
%\pmkeywords{homological algebra}

\endmetadata

% this is the default PlanetMath preamble.  as your knowledge
% of TeX increases, you will probably want to edit this, but
% it should be fine as is for beginners.

% almost certainly you want these
\usepackage{amssymb}
\usepackage{amsmath}
\usepackage{amsfonts}

% used for TeXing text within eps files
%\usepackage{psfrag}
% need this for including graphics (\includegraphics)
%\usepackage{graphicx}
% for neatly defining theorems and propositions
%\usepackage{amsthm}
% making logically defined graphics
%%%\usepackage{xypic}

% there are many more packages, add them here as you need them

% define commands here
\begin{document}
Let $X$ be a topological space and $R$ be a commutative ring.  The diagonal map $\Delta: X \to X \times X$ induces a chain map between singular cochain complexes:
$$
 \Delta^*: C^*(X \times X;\, R) \to C^*(X; \, R)  
$$. 

Let $h:C^*(X ;\, R) \otimes C^*(X ;\, R) \to C^*(X \times X;\, R )$ 

denote the chain homotopy equivalence associated with the Kunneth \PMlinkescapetext{Formula}.    

Given $\alpha \in C^p (X ;\, R)$ and $\beta \in C^q(X ;\, R)$ we define

$
\alpha \smile \beta = \Delta^* h(\alpha \otimes \beta)
$.

As $\Delta^*$ and $h$ are chain maps, $\smile$ induces a well defined product on cohomology groups, known as the cup product.  Hence the direct sum of the cohomology groups of $X$ has the structure of a ring.  This is called the cohomology ring of $X$. 

%%%%%
%%%%%
\end{document}

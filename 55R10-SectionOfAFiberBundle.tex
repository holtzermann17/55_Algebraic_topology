\documentclass[12pt]{article}
\usepackage{pmmeta}
\pmcanonicalname{SectionOfAFiberBundle}
\pmcreated{2013-03-22 13:26:43}
\pmmodified{2013-03-22 13:26:43}
\pmowner{antonio}{1116}
\pmmodifier{antonio}{1116}
\pmtitle{section of a fiber bundle}
\pmrecord{10}{34008}
\pmprivacy{1}
\pmauthor{antonio}{1116}
\pmtype{Definition}
\pmcomment{trigger rebuild}
\pmclassification{msc}{55R10}
\pmsynonym{section}{SectionOfAFiberBundle}
\pmsynonym{cross section}{SectionOfAFiberBundle}
\pmsynonym{cross-section}{SectionOfAFiberBundle}
\pmrelated{FiberBundle}
\pmdefines{smooth section}
\pmdefines{global section}
\pmdefines{local section}
\pmdefines{zero section}

% used for TeXing text within eps files
%\usepackage{psfrag}
% need this for including graphics (\includegraphics)
%\usepackage{graphicx}
% for neatly defining theorems and propositions
%\usepackage{amsthm}
% making logically defined graphics
%%%\usepackage{xypic}

\usepackage{theorem}
\usepackage{amsmath}
\usepackage{amsfonts}
\usepackage{amssymb}
\newcommand{\limv}[2]{\lim\limits_{#1\rightarrow #2}}
\newcommand{\eb}{\mathbf{e}} % Standard basis
\newcommand{\comp}{\circ} % Function composition
\newcommand{\reals}{{\mathbb R}} % The reals
\newcommand{\integs}{{\mathbb Z}} % The integers
\newcommand{\cpxs}{{\mathbb C}} % The "complexes" :)
\newcommand{\setc}[2]{\left\{#1:\: #2\right\}}
\newcommand{\set}[1]{{\left\{#1\right\}}}
\newcommand{\cycle}[1]{\left(#1\right)}
\newcommand{\tuple}[1]{\left(#1\right)}
\newcommand{\Partial}[2]{\frac{\partial #1}{\partial #2}}
\newcommand{\PartialSl}[2]{\partial #1/\partial #2}
\newcommand{\funcsig}[2]{#1\rightarrow #2}
\newcommand{\funcdef}[3]{#1:\funcsig{#2}{#3}}
\newcommand{\supp}{\mathop{\mathrm{Supp}}} % Support of a function
\newcommand{\sgn}{\mathop{\mathrm{sgn}}} % Sign function
\newcommand{\tr}[1]{#1^\mathrm{tr}} % Transpose of a matrix
\newcommand{\inprod}[2]{\left<#1,#2\right>} % Inner product
\newenvironment{smallbmatrix}{\left[\begin{smallmatrix}}{\end{smallmatrix}\right]}
\newcommand{\maps}[2]{\mathop{\mathrm{Maps}}\left(#1,#2\right)}
\newcommand{\intoc}[2]{\left(#1,#2\right]}
\newcommand{\intco}[2]{\left[#1,#2\right)}
\newcommand{\intoo}[2]{\left(#1,#2\right)}
\newcommand{\intcc}[2]{\left[#1,#2\right]}
\newcommand{\transv}{\pitchfork}
\newcommand{\pair}[2]{\left\langle#1,#2\right\rangle}
\newcommand{\norm}[1]{\left\|#1\right\|}
\newcommand{\sqnorm}[1]{\left\|#1\right\|^2}
\newcommand{\bdry}{\partial}
\newcommand{\inv}[1]{#1^{-1}}
\newcommand{\tensor}{\otimes}
\newcommand{\bigtensor}{\bigotimes}
\newcommand{\im}{\operatorname{im}}
\newcommand{\coker}{\operatorname{im}}
\newcommand{\map}{\operatorname{Map}}
\newcommand{\crit}{\operatorname{Crit}}
\theorembodyfont{\upshape}
\newtheorem{thm}{Theorem}
\newtheorem{dthm}[thm]{Desired Theorem}
\newtheorem{cor}[thm]{Corollary}
\newtheorem{dcor}[thm]{Desired Corollary}
\newtheorem{lem}[thm]{Lemma}
\newtheorem{prop}[thm]{Proposition}
\newtheorem{defn}{Definition}
\newtheorem{rmk}{Remark}
\newtheorem{exm}{Example}
\newcommand{\cross}{\times}
\newcommand{\del}{\nabla}
\newcommand{\homeo}{\cong}
\newcommand{\isom}{\cong}
\newcommand{\htpyeq}{\backsimeq}
\newcommand{\codim}{\operatorname{codim}}
\newcommand{\projp}{{\mathbb R}P}

% open cells (not very nice...)
\newcommand{\oce}{\smash{\overset{\circ}e}} 
\newcommand{\ocD}{\smash{\overset{\circ}D}} 

\newcommand{\susp}{\Sigma}
\newcommand{\restr}[2]{{#1}|_{#2}}

\renewcommand{\hom}{\mathop{\mathrm{Hom}}} % Homomorphisms functor
\newcommand{\rp}{\reals P} % real projective space
\newcommand{\cp}{\cpxs P} % complex projective space
\newcommand{\zmod}[1]{\integs / #1\integs} % Z/nZ
\begin{document}
\PMlinkescapeword{restricted}
\PMlinkescapeword{projection}
\PMlinkescapeword{equivalent}
\PMlinkescapeword{normal}

Let $\funcdef{p}{E}{B}$ be a fiber bundle, denoted by $\xi.$ 

A {\em section\/} of $\xi$ 
is a continuous map $\funcdef{s}{B}{E}$ such that the composition $p\comp s$ equals the identity. 
That is, for every $b\in B,$ $s(b)$ is an element of the fiber over $b.$
More generally, given a topological subspace $A$ of $B,$ a section of $\xi$ over $A$ is a section of the restricted bundle 
$\funcdef{\restr{p}{A}}{\inv{p}(A)}{A}.$ 

The set of sections of $\xi$ over $A$ is often denoted by $\Gamma(A;\xi),$ or
by $\Gamma(\xi)$ for sections defined on all of $B.$ Elements of $\Gamma(\xi)$ are sometimes
called {\em global sections,\/} in contrast with the {\em local sections\/} $\Gamma(U;\xi)$ defined on an open set $U.$


\begin{rmk}
If $E$ and $B$ have, for example, smooth structures, one can talk about smooth 
sections of the bundle. According to the context, the notation $\Gamma(\xi)$ often
denotes smooth sections, or some other set of suitably restricted sections.
\end{rmk}

\begin{exm}
If $\xi$ is a trivial fiber bundle with fiber $F,$ so that $E=F\cross B$ and
$p$ is projection to $B,$ then sections of $\xi$ are in a natural bijective correspondence with continuous functions $\funcsig{B}{F}.$
\end{exm}


\begin{exm}
If $B$ is a smooth manifold and $E=TB$ its tangent bundle, a (smooth) section of this bundle is precisely a (smooth) tangent vector field. 

In fact, any tensor field on a smooth manifold $M$ is a section of
an appropriate vector bundle. For instance, a contravariant $k$-tensor field is a section of the bundle $TM^{\otimes k}$ obtained by repeated  tensor product from the tangent bundle, and similarly for covariant and mixed tensor fields.
\end{exm}

\begin{exm}
If $B$ is a smooth manifold which is smoothly embedded in a Riemannian manifold 
$M,$ we can let the fiber over $b\in B$ be the orthogonal complement in $T_b M$ of the tangent space $T_b B$ of $B$ at $b$. These choices of fiber turn out to 
make up a vector bundle $\nu(B)$ over $B,$ called the {\em 
\PMlinkescapetext{normal bundle}\/} of $B$. A section of $\nu(B)$ is a normal 
vector field on $B.$
\end{exm}

\begin{exm}
If $\xi$ is a vector bundle, the {\em zero section\/} is defined simply by
$s(b)=0,$ the zero vector on the fiber. 

It is interesting to ask if a  vector bundle admits a section which is 
nowhere zero. The answer is yes, for example, in the case of a trivial vector 
bundle, but in general it depends on the topology of the spaces involved.
A well-known case of this question is the {\em hairy ball theorem,} which
says that there are no nonvanishing tangent vector fields on the sphere.
\end{exm}

\begin{exm}
If $\xi$ is a \PMlinkname{principal}{PrincipalBundle} $G$-\PMlinkname{bundle}{PrincipalBundle}, the existence of {\em any\/} section is 
equivalent to the bundle being trivial.
\end{exm}

\begin{rmk}
The correspondence taking an open set $U$ in $B$ to $\Gamma(U;\xi)$ is an example
of a sheaf on $B.$
\end{rmk}
%%%%%
%%%%%
\end{document}

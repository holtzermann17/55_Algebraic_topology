\documentclass[12pt]{article}
\usepackage{pmmeta}
\pmcanonicalname{QuantumFieldTheoriesQFT}
\pmcreated{2013-03-22 18:10:52}
\pmmodified{2013-03-22 18:10:52}
\pmowner{bci1}{20947}
\pmmodifier{bci1}{20947}
\pmtitle{quantum field theories (QFT)}
\pmrecord{35}{40753}
\pmprivacy{1}
\pmauthor{bci1}{20947}
\pmtype{Topic}
\pmcomment{trigger rebuild}
\pmclassification{msc}{55U99}
\pmclassification{msc}{81T80}
\pmclassification{msc}{81T75}
\pmclassification{msc}{81T70}
\pmclassification{msc}{81T60}
\pmclassification{msc}{81T40}
\pmclassification{msc}{81T25}
\pmclassification{msc}{81T18}
\pmclassification{msc}{81T13}
\pmclassification{msc}{81T10}
\pmclassification{msc}{81T05}
\pmsynonym{quantum theories}{QuantumFieldTheoriesQFT}
%\pmkeywords{quantum fields}
%\pmkeywords{quantum theories}
%\pmkeywords{QED}
%\pmkeywords{QCD}
%\pmkeywords{Yang-Mills theories}
%\pmkeywords{non-Abelian gauge theories}
\pmrelated{QEDInTheoreticalAndMathematicalPhysics}
\pmrelated{QuantumChromodynamicsQCD}
\pmrelated{Algebroids}
\pmrelated{Distribution4}
\pmrelated{AlgebraicQuantumFieldTheoriesAQFT}
\pmrelated{Quantization}
\pmrelated{QuantumChromodynamicsQCD}
\pmdefines{quantum interactions of all kinds}
\pmdefines{minus gravitational ones}

% this is the default PlanetMath preamble.  as your knowledge
% of TeX increases, you will probably want to edit this, but
% it should be fine as is for beginners.

% almost certainly you want these
\usepackage{amssymb}
\usepackage{amsmath}
\usepackage{amsfonts}

% used for TeXing text within eps files
%\usepackage{psfrag}
% need this for including graphics (\includegraphics)
%\usepackage{graphicx}
% for neatly defining theorems and propositions
%\usepackage{amsthm}
% making logically defined graphics
%%%\usepackage{xypic}

% there are many more packages, add them here as you need them

% define commands here

\begin{document}
This topic links the general framework of quantum field theories to group symmetries
and other relevant mathematical concepts utilized to represent quantum fields
and their fundamental properties.  

\subsection{Fundamental, mathematical concepts in quantum field theory }

 \emph{Quantum field theory (QFT)} is the general framework for describing the physics of relativistic quantum systems, such as, notably, accelerated elementary particles. 

 \emph{Quantum electrodynamics (QED)}, and \PMlinkname{QCD or quantum chromodynamics}{QCDorQuantumChromodynamics} are only two distinct theories among several quantum field theories, as their fundamental representations correspond, respectively, to very different-- $U(1)$ and $SU(3)$-- group symmetries. This obviates the need for `more fundamental' , or extended quantum symmetries, such as those afforded by either larger groups such as 
$U(1) \times SU(2) \times SU(3)$ or spontaneously broken, special symmetries of a less restrictive kind present in `quantum groupoids' as for example in weak Hopf algebra representations, or in locally compact groupoid, $G_{lc}$ unitary representations, and so on, to the higher dimensional (quantum) symmetries of quantum double groupoids, quantum double algebroids, quantum categories,quantum supercategories and/or quantum (supersymmetry) superalgebras (or graded `Lie' algebras); see, for example, their full development in a recent QFT textbook \cite{Weinberg2003} that lead to superalgebroids in quantum gravity or QCD. 


\begin{thebibliography}{9}

\bibitem{AABB70}
A. Abragam and B. Bleaney.: {\em Electron Paramagnetic Resonance of Transition Ions.}
Clarendon Press: Oxford, (1970).

\bibitem{AS}
E. M. Alfsen and F. W. Schultz: \emph{Geometry of State Spaces of
Operator Algebras}, Birkh\"auser, Boston--Basel--Berlin (2003).

\bibitem{Y} 
D.N. Yetter., TQFT's from homotopy 2-types. \textit{J. Knot Theor}. \textbf{2}: 113--123(1993).

\bibitem{Weinberg2003}
S. Weinberg.:  \emph{The Quantum Theory of Fields}. Cambridge, New York and Madrid:
Cambridge University Press, Vols. 1 to 3, (1995--2000).

\bibitem{Weinstein}
A. Weinstein : Groupoids: unifying internal and external symmetry,
\emph{Notices of the Amer. Math. Soc.} \textbf{43} (7): 744--752 (1996).

\bibitem{WB}
J. Wess and J. Bagger: \emph{Supersymmetry and Supergravity},
Princeton University Press, (1983).

\bibitem{WJ1}
J. Westman: Harmonic analysis on groupoids, \textit{Pacific J. Math.} \textbf{27}: 621-632. (1968).

\bibitem{WJ1}
J. Westman: Groupoid theory in algebra, topology and analysis., \textit{University of California at Irvine} (1971).

\bibitem{Wickra}
S. Wickramasekara and A. Bohm: Symmetry representations in the rigged Hilbert space formulation of quantum mechanics, \emph{J. Phys. A} \textbf{35}(3): 807-829 (2002).

\bibitem{Wightman1}
Wightman, A. S., 1956, Quantum Field Theory in Terms of Vacuum Expectation Values, Physical Review, \textbf{101}: 860--866.

\bibitem{Wightman--Garding3}
Wightman, A.S. and Garding, L., 1964, Fields as Operator--Valued Distributions in Relativistic Quantum Theory, Arkiv f\"ur Fysik, 28: 129--184.

\bibitem{Woronowicz1}
S. L. Woronowicz : Twisted {\em SU(2)} group : An example of a non--commutative differential calculus, RIMS, Kyoto University \textbf{23} (1987), 613--665.

\end{thebibliography}
%%%%%
%%%%%
\end{document}

\documentclass[12pt]{article}
\usepackage{pmmeta}
\pmcanonicalname{SimplicialApproximation}
\pmcreated{2013-03-22 16:54:24}
\pmmodified{2013-03-22 16:54:24}
\pmowner{Mathprof}{13753}
\pmmodifier{Mathprof}{13753}
\pmtitle{simplicial approximation}
\pmrecord{6}{39166}
\pmprivacy{1}
\pmauthor{Mathprof}{13753}
\pmtype{Definition}
\pmcomment{trigger rebuild}
\pmclassification{msc}{55U10}

\endmetadata

% this is the default PlanetMath preamble.  as your knowledge
% of TeX increases, you will probably want to edit this, but
% it should be fine as is for beginners.

% almost certainly you want these
\usepackage{amssymb}
\usepackage{amsmath}
\usepackage{amsfonts}

% used for TeXing text within eps files
%\usepackage{psfrag}
% need this for including graphics (\includegraphics)
%\usepackage{graphicx}
% for neatly defining theorems and propositions
%\usepackage{amsthm}
% making logically defined graphics
%%%\usepackage{xypic}

% there are many more packages, add them here as you need them

% define commands here

\begin{document}
Let $K$ and $L$ be simplicial complexes and $f: |K| \to |L|$ be a continuous function.
A simplicial mapping $g: |K| \to |L|$ which is homotopic to $f$ is called
a \emph{simplicial approximation} of $f$.

For example, suppose that $L$ is the closure of an $n$-simplex and $a_0$ is a vertex of $L$. Let $f$ be a continuous map of $|K|$ to $|L|$ where $K$ 
is some simplicial complex. Then the map $g$ that sends all of $K$ to $a_0$ is
a simplicial approximation of $f$. 
%%%%%
%%%%%
\end{document}

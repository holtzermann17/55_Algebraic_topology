\documentclass[12pt]{article}
\usepackage{pmmeta}
\pmcanonicalname{EilenbergMacLaneSpace}
\pmcreated{2013-03-22 13:25:42}
\pmmodified{2013-03-22 13:25:42}
\pmowner{antonio}{1116}
\pmmodifier{antonio}{1116}
\pmtitle{Eilenberg-MacLane space}
\pmrecord{6}{33986}
\pmprivacy{1}
\pmauthor{antonio}{1116}
\pmtype{Definition}
\pmcomment{trigger rebuild}
\pmclassification{msc}{55P20}
\pmsynonym{Eilenberg-Mac Lane space}{EilenbergMacLaneSpace}
%\pmkeywords{cohomology}
%\pmkeywords{CW complex}
\pmrelated{NaturalTransformation}
\pmrelated{LoopSpace}
\pmrelated{HomotopyGroups}
\pmrelated{RepresentableFunctor}
\pmrelated{FundamentalGroupoid2}
\pmrelated{CohomologyGroupTheorem}
\pmrelated{ProofOfCohomologyGroupTheorem}
\pmrelated{OmegaSpectrum}

% used for TeXing text within eps files
%\usepackage{psfrag}
% need this for including graphics (\includegraphics)
%\usepackage{graphicx}
% for neatly defining theorems and propositions
%\usepackage{amsthm}
% making logically defined graphics
%%%\usepackage{xypic}

\usepackage{amsthm}
\usepackage{amsmath}
\usepackage{amsfonts}
\usepackage{amssymb}

\newcommand{\limv}[2]{\lim\limits_{#1\rightarrow #2}}
\newcommand{\eb}{\mathbf{e}} % Standard basis
\newcommand{\comp}{\circ} % Function composition
\newcommand{\reals}{{\mathbb R}} % The reals
\newcommand{\integs}{{\mathbb Z}} % The integers
\newcommand{\cpxs}{{\mathbb C}} % The "complexes" :)
\newcommand{\setc}[2]{\left\{#1:\: #2\right\}}
\newcommand{\set}[1]{{\left\{#1\right\}}}
\newcommand{\cycle}[1]{\left(#1\right)}
\newcommand{\tuple}[1]{\left(#1\right)}
\newcommand{\Partial}[2]{\frac{\partial #1}{\partial #2}}
\newcommand{\PartialSl}[2]{\partial #1/\partial #2}
\newcommand{\funcsig}[2]{#1\rightarrow #2}
\newcommand{\funcdef}[3]{#1:\funcsig{#2}{#3}}
\newcommand{\supp}{\mathop{\mathrm{Supp}}} % Support of a function
\newcommand{\sgn}{\mathop{\mathrm{sgn}}} % Sign function
\newcommand{\tr}[1]{#1^\mathrm{tr}} % Transpose of a matrix
\newcommand{\inprod}[2]{\left<#1,#2\right>} % Inner product
\newenvironment{smallbmatrix}{\left[\begin{smallmatrix}}{\end{smallmatrix}\right]}
\newcommand{\maps}[2]{\mathop{\mathrm{Maps}}\left(#1,#2\right)}
\newcommand{\intoc}[2]{\left(#1,#2\right]}
\newcommand{\intco}[2]{\left[#1,#2\right)}
\newcommand{\intoo}[2]{\left(#1,#2\right)}
\newcommand{\intcc}[2]{\left[#1,#2\right]}
\newcommand{\transv}{\pitchfork}
\newcommand{\pair}[2]{\left\langle#1,#2\right\rangle}
\newcommand{\norm}[1]{\left\|#1\right\|}
\newcommand{\sqnorm}[1]{\left\|#1\right\|^2}
\newcommand{\bdry}{\partial}
\newcommand{\inv}[1]{#1^{-1}}
\newcommand{\tensor}{\otimes}
\newcommand{\bigtensor}{\bigotimes}
\newcommand{\im}{\operatorname{im}}
\newcommand{\coker}{\operatorname{im}}
\newcommand{\map}{\operatorname{Map}}
\newcommand{\crit}{\operatorname{Crit}}
\newtheorem{thm}{Theorem}
\newtheorem{dthm}[thm]{Desired Theorem}
\newtheorem{cor}[thm]{Corollary}
\newtheorem{dcor}[thm]{Desired Corollary}
\newtheorem{lem}[thm]{Lemma}
\newtheorem{prop}[thm]{Proposition}
\newtheorem{defn}{Definition}
\newtheorem{rmk}{Remark}
\newtheorem{exm}{Example}
\newcommand{\cross}{\times}
\newcommand{\del}{\nabla}
\newcommand{\homeo}{\cong}
\newcommand{\isom}{\cong}
\newcommand{\htpyeq}{\backsimeq}
\newcommand{\codim}{\operatorname{codim}}
\newcommand{\projp}{{\mathbb R}P}

% open cells (not very nice...)
\newcommand{\oce}{\smash{\overset{\circ}e}} 
\newcommand{\ocD}{\smash{\overset{\circ}D}} 

\newcommand{\susp}{\Sigma}
\newcommand{\restr}[2]{{#1}|_{#2}}

\renewcommand{\hom}{\mathop{\mathrm{Hom}}} % Homomorphisms functor
\begin{document}
\PMlinkescapeword{type}

Let $\pi$ be a discrete group. A based topological space $X$ is called an {\em Eilenberg-MacLane space\/} of type $K(\pi,n)$, where $n\ge 1,$  if all the homotopy groups $\pi_k(X)$ are trivial except for $\pi_n(X),$ which is isomorphic to $\pi.$ Clearly, for such a space to exist when $n\ge 2,$ $\pi$ must be abelian.

Given any group $\pi,$ with $\pi$ abelian if $n\ge 2,$ there exists an Eilenberg-MacLane space of type $K(\pi,n).$ Moreover, this space can be constructed as a CW complex. It turns out that any two Eilenberg-MacLane spaces of type $K(\pi,n)$ are weakly homotopy equivalent. The Whitehead theorem then implies that there is a unique $K(\pi,n)$ space up to homotopy equivalence in the category of topological spaces of the homotopy type of a CW complex. We will henceforth restrict ourselves to this category. With a slight abuse of notation, we refer to any such space as $K(\pi,n).$

An important property of $K(\pi,n)$ is that, for $\pi$ abelian, 
there is a natural isomorphism
\[
H^n(X;\pi) \isom [X,K(\pi,n)]
\]
of contravariant set-valued functors, where $[X,K(\pi,n)]$ is the set of homotopy classes of based maps from $X$ to $K(\pi,n).$ Thus one says that the $K(\pi,n)$ are {\em representing spaces\/} for cohomology with coefficients in $\pi.$

\begin{rmk}
Even when the group $\pi$ is nonabelian, it can be seen that the set 
$[X,K(\pi,1)]$ is naturally isomorphic to $\hom(\pi_1(X),\pi)/\pi;$ that is,
to conjugacy classes of homomorphisms from $\pi_1(X)$ to $\pi.$ In fact, this
is a way to define $H^1(X;\pi)$ when $\pi$ is nonabelian.
\end{rmk}

\begin{rmk}
Though the above description does not include the case $n=0,$ it is natural to define a $K(\pi,0)$ to be any space homotopy equivalent to $\pi.$ The above statement about cohomology then becomes true for the reduced zeroth cohomology functor.
\end{rmk}


% Might add this later...
%
%Using the fact that $\pi_k(\Omega X)\isom\pi_{k+1}(X),$ it is clear that %$\Omega K(\pi,n)$ is a $K(\pi,n).$ It follows  that there is a homotopy %equivalence $\funcdef{f_n}{\Omega K(\pi,n)}{K(\pi,n-1)}$, making the sequence %$\seq{K(\pi,n),f_n}$ an Omega-spectrum.
%%%%%
%%%%%
\end{document}

\documentclass[12pt]{article}
\usepackage{pmmeta}
\pmcanonicalname{FiberBundle}
\pmcreated{2013-03-22 13:07:06}
\pmmodified{2013-03-22 13:07:06}
\pmowner{bwebste}{988}
\pmmodifier{bwebste}{988}
\pmtitle{fiber bundle}
\pmrecord{10}{33551}
\pmprivacy{1}
\pmauthor{bwebste}{988}
\pmtype{Definition}
\pmcomment{trigger rebuild}
\pmclassification{msc}{55R10}
\pmsynonym{fibre bundle}{FiberBundle}
\pmrelated{ReductionOfStructureGroup}
\pmrelated{SectionOfAFiberBundle}
\pmrelated{Fibration}
\pmrelated{Fibration2}
\pmrelated{HomotopyLiftingProperty}
\pmrelated{SurfaceBundleOverTheCircle}
\pmdefines{trivial bundle}
\pmdefines{local trivializations}
\pmdefines{structure group}
\pmdefines{cocycle condition}
\pmdefines{local trivialization}

% this is the default PlanetMath preamble.  as your knowledge
% of TeX increases, you will probably want to edit this, but
% it should be fine as is for beginners.

% almost certainly you want these
\usepackage{amssymb}
\usepackage{amsmath}
\usepackage{amsfonts}
%%\usepackage{xypic}

% used for TeXing text within eps files
%\usepackage{psfrag}
% need this for including graphics (\includegraphics)
%\usepackage{graphicx}
% for neatly defining theorems and propositions
%\usepackage{amsthm}
% making logically defined graphics
%%%\usepackage{xypic} 

% there are many more packages, add them here as you need them

% define commands here
\begin{document}
Let $F$ be a topological space and $G$ be a topological group which acts on $F$ on the left. A \emph{fiber bundle} with fiber $F$ and \emph{structure group} $G$ consists of the following data:
\begin{itemize}
\item a topological space $B$ called the base space, a space $E$ called the total space and a continuous surjective map $\pi:E \to B$ called the projection of the bundle,
\item an open cover $\{U_i\}$ of $B$ along with a collection of continuous maps
$\{\phi_i: \pi^{-1}U_i \to F\}$ called \emph{local trivializations} and
\item a collection of continuous maps $\{g_{ij}: U_i \cap U_j \to G\}$ called \emph{transition functions}
\end{itemize}
which satisfy the following properties
\begin{enumerate}
\item the map $\pi^{-1}U_i \to U_i \times F$ given by $e \mapsto (\pi(e),\phi_i(e))$ is a homeomorphism for each $i$,
\item for all indices $i,j$ and $e \in \pi^{-1}(U_i \cap U_j)$, $g_{ji}(\pi(e))\cdot \phi_i(e) = \phi_j(e)$ and
\item for all indices $i,j,k$ and $b \in U_i \cap U_j \cap U_k$, $g_{ij}(b)g_{jk}(b) = g_{ik}(b)$.
\end{enumerate}

Readers familiar with \v{C}ech cohomology may recognize condition 3), it is often called the \emph{cocycle condition}. Note, this imples that $g_{ii}(b)$ is the identity in $G$ for each $b$, and $g_{ij}(b) = g_{ji}(b)^{-1}$. 

If the total space $E$ is homeomorphic to the product $B \times F$ so that the bundle projection is essentially projection onto the first factor, then $\pi : E \to B$ is called a \emph{trivial bundle}. Some examples of fiber bundles are vector bundles and covering spaces.

There is a notion of morphism of fiber bundles $E,E'$ over the same base $B$ with the same structure group $G$.  Such a morphism is a $G$-equivariant map $\xi:E\to E'$, making the following diagram commute

$$\xymatrix{E\ar[rr]^\xi\ar[dr]_\pi& &E'\ar[dl]^{\pi'}\\ &B&}.$$

Thus we have a category of fiber bundles over a fixed base with fixed structure group.
%%%%%
%%%%%
\end{document}

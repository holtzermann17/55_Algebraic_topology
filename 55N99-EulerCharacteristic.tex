\documentclass[12pt]{article}
\usepackage{pmmeta}
\pmcanonicalname{EulerCharacteristic}
\pmcreated{2013-03-22 16:12:51}
\pmmodified{2013-03-22 16:12:51}
\pmowner{Mathprof}{13753}
\pmmodifier{Mathprof}{13753}
\pmtitle{Euler  characteristic}
\pmrecord{13}{38311}
\pmprivacy{1}
\pmauthor{Mathprof}{13753}
\pmtype{Definition}
\pmcomment{trigger rebuild}
\pmclassification{msc}{55N99}

% this is the default PlanetMath preamble.  as your knowledge
% of TeX increases, you will probably want to edit this, but
% it should be fine as is for beginners.

% almost certainly you want these
\usepackage{amssymb}
\usepackage{amsmath}
\usepackage{amsfonts}

% used for TeXing text within eps files
%\usepackage{psfrag}
% need this for including graphics (\includegraphics)
%\usepackage{graphicx}
% for neatly defining theorems and propositions
%\usepackage{amsthm}
% making logically defined graphics
%%%\usepackage{xypic}

% there are many more packages, add them here as you need them

% define commands here

\begin{document}
The term \emph{Euler characteristic} is defined for several objects.

If $K$ is a finite simplicial complex of dimension $m$, let $\alpha_i$ be the number of
simplexes of dimension $i$. The \emph{Euler characteristic} of $K$
is defined to be
$$
\chi(K) = \sum_{i=0}^m (-1)^i \alpha_i .
$$

Next, if $K$ is a finite CW complex, let $\alpha_i$ be the number of i-cells
in $K$. The \emph{Euler characteristic} of $K$
is defined to be

$$
\chi(K) = \sum_{i \ge 0}(-1)^i \alpha_i .
$$

If $X$ is a finite polyhedron, with triangulation $K$, a simplicial complex,
then the \emph{Euler characteristic} of $X$ is $\chi(K)$. It can be shown
that all triangulations of $X$ have the same value for $\chi(K)$ so that
this is well-defined. 

Finally, if $C=\{C_q\}$ is a finitely generated graded group, then
the \emph{Euler characteristic} of $C$ is defined to be
$$
\chi(C) = \sum_{q \ge 0}  (-1)^q rank(C_q) .
$$
%%%%%
%%%%%
\end{document}

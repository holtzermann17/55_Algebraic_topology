\documentclass[12pt]{article}
\usepackage{pmmeta}
\pmcanonicalname{VanKampensTheorem}
\pmcreated{2013-03-22 13:24:17}
\pmmodified{2013-03-22 13:24:17}
\pmowner{RonaldBrown}{18528}
\pmmodifier{RonaldBrown}{18528}
\pmtitle{Van Kampen's theorem}
\pmrecord{9}{33947}
\pmprivacy{1}
\pmauthor{RonaldBrown}{18528}
\pmtype{Theorem}
\pmcomment{trigger rebuild}
\pmclassification{msc}{55Q05}
\pmsynonym{Seifert-Van Kampen theorem}{VanKampensTheorem}

%\documentclass{amsart}
\usepackage{amsmath}
\usepackage[all,poly,knot,dvips]{xy}
%\usepackage{pstricks,pst-poly,pst-node,pstcol}


\usepackage{amssymb,latexsym}

\usepackage{amsthm,latexsym}
\usepackage{eucal,latexsym}

% THEOREM Environments --------------------------------------------------

\newtheorem{thm}{Theorem}
 \newtheorem*{mainthm}{Main~Theorem}
 \newtheorem{cor}[thm]{Corollary}
 \newtheorem{lem}[thm]{Lemma}
 \newtheorem{prop}[thm]{Proposition}
 \newtheorem{claim}[thm]{Claim}
 \theoremstyle{definition}
 \newtheorem{defn}[thm]{Definition}
 \theoremstyle{remark}
 \newtheorem{rem}[thm]{Remark}
 \numberwithin{equation}{subsection}


%---------------------  Greek letters, etc ------------------------- 

\newcommand{\CA}{\mathcal{A}}
\newcommand{\CC}{\mathcal{C}}
\newcommand{\CM}{\mathcal{M}}
\newcommand{\CP}{\mathcal{P}}
\newcommand{\CS}{\mathcal{S}}
\newcommand{\BC}{\mathbb{C}}
\newcommand{\BN}{\mathbb{N}}
\newcommand{\BR}{\mathbb{R}}
\newcommand{\BZ}{\mathbb{Z}}
\newcommand{\FF}{\mathfrak{F}}
\newcommand{\FL}{\mathfrak{L}}
\newcommand{\FM}{\mathfrak{M}}
\newcommand{\Ga}{\alpha}
\newcommand{\Gb}{\beta}
\newcommand{\Gg}{\gamma}
\newcommand{\GG}{\Gamma}
\newcommand{\Gd}{\delta}
\newcommand{\GD}{\Delta}
\newcommand{\Ge}{\varepsilon}
\newcommand{\Gz}{\zeta}
\newcommand{\Gh}{\eta}
\newcommand{\Gq}{\theta}
\newcommand{\GQ}{\Theta}
\newcommand{\Gi}{\iota}
\newcommand{\Gk}{\kappa}
\newcommand{\Gl}{\lambda}
\newcommand{\GL}{\Lamda}
\newcommand{\Gm}{\mu}
\newcommand{\Gn}{\nu}
\newcommand{\Gx}{\xi}
\newcommand{\GX}{\Xi}
\newcommand{\Gp}{\pi}
\newcommand{\GP}{\Pi}
\newcommand{\Gr}{\rho}
\newcommand{\Gs}{\sigma}
\newcommand{\GS}{\Sigma}
\newcommand{\Gt}{\tau}
\newcommand{\Gu}{\upsilon}
\newcommand{\GU}{\Upsilon}
\newcommand{\Gf}{\varphi}
\newcommand{\GF}{\Phi}
\newcommand{\Gc}{\chi}
\newcommand{\Gy}{\psi}
\newcommand{\GY}{\Psi}
\newcommand{\Gw}{\omega}
\newcommand{\GW}{\Omega}
\newcommand{\Gee}{\epsilon}
\newcommand{\Gpp}{\varpi}
\newcommand{\Grr}{\varrho}
\newcommand{\Gff}{\phi}
\newcommand{\Gss}{\varsigma}

\def\co{\colon\thinspace}
\begin{document}
Van Kampen's theorem for fundamental groups may be stated as
follows:
\begin{thm}
 Let $X$ be a topological space which is the union of the interiors of  two path connected
 subspaces $X_1, X_2$. Suppose $X_0:=X_1\cap X_2$ is path connected. Let
 further $*\in X_0$ and $i_k\co \pi_1(X_0,*)\to\pi_1(X_k,*)$,
 $j_k\co\pi_1(X_k,*)\to\pi_1(X,*)$ be induced by the inclusions for
 $k=1,2$. Then $X$ is path connected and the natural morphism
$$\pi_1(X_1,*)\bigstar_{\pi_1(X_0,*)}\pi_1(X_2,*)\to \pi_1(X,*)\,,$$
is an isomorphism,     that is, the fundamental group of $X$ is the
free product of the
    fundamental groups of $X_1$ and $X_2$ with amalgamation  of $\pi_1(X_0,*)$.
\end{thm}

Usually the morphisms induced by inclusion in this theorem are not
themselves injective, and the more precise version of the  statement
is in terms of {pushouts} of groups.

The notion of pushout in the category of groupoids allows for a
version of the theorem  for the non path connected case, using the
fundamental groupoid $\pi_1(X,A)$  on a set $A$ of base points,
\cite{rb1}. This groupoid consists of homotopy classes rel end
points of paths in $X$ joining points of $A\cap X$. In particular,
if $X$ is a contractible space, and $A$ consists of two distinct
points of $X$, then $\pi_1(X,A)$ is easily seen to be isomorphic to
the groupoid often written $\mathcal I$ with two vertices and
exactly one morphism between any two vertices. This groupoid plays a
role in the theory of groupoids analogous to that of the group of
integers in the theory of groups.


\begin{thm}
Let the topological space $X$ be covered by the interiors of two
subspaces $X_1, X_2$ and let $A$ be a set which meets each path
component of $X_1, X_2$ and $X_0:=X_1 \cap X_2$. Then $A$ meets each
path component of $X$ and the following diagram of morphisms induced
by inclusion
$$\xymatrix{ {\pi_1(X_0,A)}\ar [r]^{\pi_1(i_1)}\ar[d]_{\pi_1(i_2)}
&\pi_1(X_1,A)\ar[d]^{\pi_1(j_1)} \\
{\pi_1(X_2,A)}\ar [r]_{\pi_1(j_2)}&  {\pi_1(X,A)} }
$$
is a  pushout diagram in the category of groupoids.
\end{thm}

The interpretation of this theorem as a calculational tool for
fundamental groups needs some development of `combinatorial groupoid
theory', \cite{rb,higgins}. This theorem implies the calculation of
the fundamental group of the circle as the group of integers, since
the group of integers is obtained from the groupoid $\mathcal I$ by
identifying, in the category of groupoids,  its two vertices.

There is a version of the last theorem when $X$ is covered by the
union of the interiors of a family $\{U_\lambda : \lambda \in
\Lambda\}$ of subsets, \cite{brs}.  The conclusion is that if  $A$
meets each path component of all 1,2,3-fold intersections of the
sets $U_\lambda$, then A meets all path components of $X$ and the
diagram
$$ \bigsqcup_{(\lambda,\mu) \in \Lambda^2} \pi_1(U_\lambda \cap U_\mu, A) \rightrightarrows \bigsqcup_{\lambda \in \Lambda} \pi_1(U_\lambda, A)\rightarrow \pi_1(X,A) $$
of morphisms induced by inclusions is a coequaliser in the category
of groupoids.

\begin{thebibliography}{8}
\bibitem{rb1} R. Brown, ``Groupoids and Van Kampen's theorem'', {\em Proc. London
Math. Soc.} (3)  17 (1967) 385-401.

\bibitem{rb} R. Brown, {\em Topology and Groupoids}, Booksurge PLC (2006).
\bibitem{brs} R. Brown and A. Razak, ``A van Kampen theorem for unions of
non-connected  spaces'', {\em Archiv. Math.} 42 (1984) 85-88.
\bibitem{higgins} P.J. Higgins, {\em Categories and Groupoids}, van Nostrand, 1971,
Reprints of   Theory and Applications of Categories,  No. 7 (2005)
pp 1-195.

\end{thebibliography}
%%%%%
%%%%%
\end{document}

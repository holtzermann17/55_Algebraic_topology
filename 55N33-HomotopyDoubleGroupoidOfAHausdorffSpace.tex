\documentclass[12pt]{article}
\usepackage{pmmeta}
\pmcanonicalname{HomotopyDoubleGroupoidOfAHausdorffSpace}
\pmcreated{2013-03-22 18:14:56}
\pmmodified{2013-03-22 18:14:56}
\pmowner{bci1}{20947}
\pmmodifier{bci1}{20947}
\pmtitle{homotopy double groupoid of a Hausdorff space}
\pmrecord{76}{40843}
\pmprivacy{1}
\pmauthor{bci1}{20947}
\pmtype{Definition}
\pmcomment{trigger rebuild}
\pmclassification{msc}{55N33}
\pmclassification{msc}{55N20}
\pmclassification{msc}{22A22}
\pmclassification{msc}{18D05}
\pmclassification{msc}{55U40}
\pmsynonym{homotopy double groupoid of a compact space}{HomotopyDoubleGroupoidOfAHausdorffSpace}
%\pmkeywords{groupoid}
%\pmkeywords{double groupoid}
%\pmkeywords{homotopy}
%\pmkeywords{homotopy group}
%\pmkeywords{homotopy double groupoid}
%\pmkeywords{Hausdorff space}
%\pmkeywords{homotopy double groupoid of a Hausdorff space}
\pmrelated{HomotopyAdditionLemma}
\pmrelated{GroupoidCategory}
\pmrelated{ThinDoubleTracks}
\pmrelated{HigherDimensionalAlgebraHDA}
\pmrelated{FundamentalGroupoidFunctor}
\pmrelated{FundamentalGroupoid2}
\pmrelated{FundamentalGroupoid}
\pmrelated{HomotopyGroups}
\pmrelated{HigherDimensionalAlgebraHDA}
\pmrelated{GeneralizedVanKampenTheoremsHigherDimensional}
\pmrelated{Homot}
\pmdefines{double groupoid}
\pmdefines{homotopy double groupoid}
\pmdefines{thin equivalence}
\pmdefines{thinly equivalent}
\pmdefines{higher dimensional homotopy}
\pmdefines{double category}

% this is the default PlanetMath preamble.  as your knowledge
% of TeX increases, you will probably want to edit this, but
% it should be fine as is for beginners.

% almost certainly you want these
\usepackage{amsmath, amssymb, amsfonts, amsthm, amscd, latexsym, enumerate}
\usepackage{xypic, xspace}
\usepackage[mathscr]{eucal}
\usepackage[dvips]{graphicx}
\usepackage[curve]{xy}

\setlength{\textwidth}{6.5in}
%\setlength{\textwidth}{16cm}
\setlength{\textheight}{9.0in}
%\setlength{\textheight}{24cm}

\hoffset=-.75in     %%ps format
%\hoffset=-1.0in     %%hp format
\voffset=-.4in


\theoremstyle{plain}
\newtheorem{lemma}{Lemma}[section]
\newtheorem{proposition}{Proposition}[section]
\newtheorem{theorem}{Theorem}[section]
\newtheorem{corollary}{Corollary}[section]

\theoremstyle{definition}
\newtheorem{definition}{Definition}[section]
\newtheorem{example}{Example}[section]
%\theoremstyle{remark}
\newtheorem{remark}{Remark}[section]
\newtheorem*{notation}{Notation}
\newtheorem*{claim}{Claim}

\renewcommand{\thefootnote}{\ensuremath{\fnsymbol{footnote}}}
\numberwithin{equation}{section}

\newcommand{\Ad}{{\rm Ad}}
\newcommand{\Aut}{{\rm Aut}}
\newcommand{\Cl}{{\rm Cl}}
\newcommand{\Co}{{\rm Co}}
\newcommand{\DES}{{\rm DES}}
\newcommand{\Diff}{{\rm Diff}}
\newcommand{\Dom}{{\rm Dom}}
\newcommand{\Hol}{{\rm Hol}}
\newcommand{\Mon}{{\rm Mon}}
\newcommand{\Hom}{{\rm Hom}}
\newcommand{\Ker}{{\rm Ker}}
\newcommand{\Ind}{{\rm Ind}}
\newcommand{\IM}{{\rm Im}}
\newcommand{\Is}{{\rm Is}}
\newcommand{\ID}{{\rm id}}
\newcommand{\grpL}{{\rm GL}}
\newcommand{\Iso}{{\rm Iso}}
\newcommand{\rO}{{\rm O}}
\newcommand{\Sem}{{\rm Sem}}
\newcommand{\SL}{{\rm Sl}}
\newcommand{\St}{{\rm St}}
\newcommand{\Sym}{{\rm Sym}}
\newcommand{\Symb}{{\rm Symb}}
\newcommand{\SU}{{\rm SU}}
\newcommand{\Tor}{{\rm Tor}}
\newcommand{\U}{{\rm U}}

\newcommand{\A}{\mathcal A}
\newcommand{\Ce}{\mathcal C}
\newcommand{\D}{\mathcal D}
\newcommand{\E}{\mathcal E}
\newcommand{\F}{\mathcal F}
%\newcommand{\grp}{\mathcal G}
\renewcommand{\H}{\mathcal H}
\renewcommand{\cL}{\mathcal L}
\newcommand{\Q}{\mathcal Q}
\newcommand{\R}{\mathcal R}
\newcommand{\cS}{\mathcal S}
\newcommand{\cU}{\mathcal U}
\newcommand{\W}{\mathcal W}

\newcommand{\bA}{\mathbb{A}}
\newcommand{\bB}{\mathbb{B}}
\newcommand{\bC}{\mathbb{C}}
\newcommand{\bD}{\mathbb{D}}
\newcommand{\bE}{\mathbb{E}}
\newcommand{\bF}{\mathbb{F}}
\newcommand{\bG}{\mathbb{G}}
\newcommand{\bK}{\mathbb{K}}
\newcommand{\bM}{\mathbb{M}}
\newcommand{\bN}{\mathbb{N}}
\newcommand{\bO}{\mathbb{O}}
\newcommand{\bP}{\mathbb{P}}
\newcommand{\bR}{\mathbb{R}}
\newcommand{\bV}{\mathbb{V}}
\newcommand{\bZ}{\mathbb{Z}}

\newcommand{\bfE}{\mathbf{E}}
\newcommand{\bfX}{\mathbf{X}}
\newcommand{\bfY}{\mathbf{Y}}
\newcommand{\bfZ}{\mathbf{Z}}

\renewcommand{\O}{\Omega}
\renewcommand{\o}{\omega}
\newcommand{\vp}{\varphi}
\newcommand{\vep}{\varepsilon}

\newcommand{\diag}{{\rm diag}}
\newcommand{\grp}{{\mathsf{G}}}
\newcommand{\dgrp}{{\mathsf{D}}}
\newcommand{\desp}{{\mathsf{D}^{\rm{es}}}}
\newcommand{\grpeod}{{\rm Geod}}
%\newcommand{\grpeod}{{\rm geod}}
\newcommand{\hgr}{{\mathsf{H}}}
\newcommand{\mgr}{{\mathsf{M}}}
\newcommand{\ob}{{\rm Ob}}
\newcommand{\obg}{{\rm Ob(\mathsf{G)}}}
\newcommand{\obgp}{{\rm Ob(\mathsf{G}')}}
\newcommand{\obh}{{\rm Ob(\mathsf{H})}}
\newcommand{\Osmooth}{{\Omega^{\infty}(X,*)}}
\newcommand{\grphomotop}{{\rho_2^{\square}}}
\newcommand{\grpcalp}{{\mathsf{G}(\mathcal P)}}

\newcommand{\rf}{{R_{\mathcal F}}}
\newcommand{\grplob}{{\rm glob}}
\newcommand{\loc}{{\rm loc}}
\newcommand{\TOP}{{\rm TOP}}

\newcommand{\wti}{\widetilde}
\newcommand{\what}{\widehat}

\renewcommand{\a}{\alpha}
\newcommand{\be}{\beta}
\newcommand{\grpa}{\grpamma}
%\newcommand{\grpa}{\grpamma}
\newcommand{\de}{\delta}
\newcommand{\del}{\partial}
\newcommand{\ka}{\kappa}
\newcommand{\si}{\sigma}
\newcommand{\ta}{\tau}

\newcommand{\med}{\medbreak}
\newcommand{\medn}{\medbreak \noindent}
\newcommand{\bign}{\bigbreak \noindent}

\newcommand{\lra}{{\longrightarrow}}
\newcommand{\ra}{{\rightarrow}}
\newcommand{\rat}{{\rightarrowtail}}
\newcommand{\ovset}[1]{\overset {#1}{\ra}}
\newcommand{\ovsetl}[1]{\overset {#1}{\lra}}
\newcommand{\hr}{{\hookrightarrow}}

\newcommand{\<}{{\langle}}

%\newcommand{\>}{{\rangle}}

%\usepackage{geometry, amsmath,amssymb,latexsym,enumerate}
%%%\usepackage{xypic}

\def\baselinestretch{1.1}


\hyphenation{prod-ucts}

%\grpeometry{textwidth= 16 cm, textheight=21 cm}

\newcommand{\sqdiagram}[9]{$$ \diagram  #1  \rto^{#2} \dto_{#4}&
#3  \dto^{#5} \\ #6    \rto_{#7}  &  #8   \enddiagram
\eqno{\mbox{#9}}$$ }

\def\C{C^{\ast}}

\newcommand{\labto}[1]{\stackrel{#1}{\longrightarrow}}

%\newenvironment{proof}{\noindent {\bf Proof} }{ \hfill $\Box$
%{\mbox{}}

\newcommand{\quadr}[4]
{\begin{pmatrix} & #1& \\[-1.1ex] #2 & & #3\\[-1.1ex]& #4&
 \end{pmatrix}}
\def\D{\mathsf{D}}

\begin{document}
\textbf{Homotopy double groupoid of a Hausdorff space}

 Let $X$ be a Hausdorff space. Also consider the HDA concept of a
\PMlinkname{double groupoid}{HigherDimensionalAlgebraHDA}, 
and how it can be completely specified for a Hausdorff space, $X$. Thus, in 
ref. \cite{BHKP} Brown et al. associated to $X$ a double groupoid, $\boldsymbol{\rho}^{\square}_2 (X)$
, called the {\em homotopy double groupoid of X} which is completely defined by the data specified in Definitions 0.1 to 0.3 in this entry and related objects.

Generally, the geometry of squares and their compositions leads to a common representation of a \emph{double groupoid} in the following form:


\begin{equation}
\label{squ} \D = \vcenter{\xymatrix @=3pc {S \ar @<1ex> [r] ^{s^1} \ar @<-1ex> [r]
_{t^1} \ar @<1ex> [d]^{\, t_2}  \ar @<-1ex> [d]_{s_2} & H   \ar[l]
\ar @<1ex> [d]^{\,t}
 \ar @<-1ex> [d]_s \\
V \ar [u]  \ar @<1ex> [r] ^s \ar @<-1ex> [r] _t & M \ar [l] \ar[u]}},
\end{equation}


where $M$ is a set of `points', $H,V$ are `horizontal' and `vertical' groupoids, and $S$ is a set of
`squares' with two compositions. 

The laws for a double groupoid are also defined, more generally, for any topological space $\mathbb{T}$, and make it also describable as a groupoid internal to the category of groupoids. Further details of this general definition are provided next.

 Given two groupoids $H,V$  over a set $M$, there is a double groupoid $\Box(H,V)$ with $H,V$ as
 horizontal and vertical edge groupoids, and squares given by
 quadruples
 \bigbreak
\begin{equation}
\begin{pmatrix} & h& \\[-0.9ex] v & & v'\\[-0.9ex]& h'&
\end{pmatrix}
\end{equation}
for which we assume always that $h,h' \in H, \, v,v' \in V$ and
that the initial and final points of these edges match in $M$ as
suggested by the notation, that is for example $sh=sv, th=sv',
\ldots$, etc. The compositions are to be inherited from those of
$H,V$,
 that is:
 \bigbreak
\begin{equation}
\quadr{h}{v}{v'}{h'} \circ_1\quadr{h'}{w}{w'}{h''}
=\quadr{h}{vw}{v'w'}{h''}, \;\quadr{h}{v}{v'}{h'}
\circ_2\quadr{k}{v'}{v''}{k'}=\quadr{hk}{v}{v''}{h'k'} ~.
\end{equation}

Alternatively, the data for the above double groupoid $\D$  can be specified as a triple of groupoid structures: 
$$(D_2,D_1, \partial^{-}_{1}, \partial^{+}_{1}, +_1,\varepsilon_1), (D_2,D_1,\partial^{-}_{2}, \partial^{+}_{2}, +_2, \varepsilon_2), (D_1, D_0, \partial^{-}_{1}, \partial^{+}_{1}, + , \varepsilon),$$

where: 
$$D_0 = M ~,~ D_1= V = H ~,~ D_2 = S,$$  
$$s^1 = \partial^{-}_{2}~,~ t^1 = \partial^{+}_{2}~,~ s_2 = s= \partial^{-}_{1}$$ and 
$$t_2 = t = \partial^{+}_{1}.$$ 

Then, as a first step, consider this data for the homotopy double groupoid specified in the following definition; 
in order to specify completely such data one also needs to define the related concepts of \emph{thin equivalence} and the \emph{relation of cubically thin homotopy}, as provided in the two definitions following the homotopy double groupoid data specified above and in the (main) Definition 0.1.  
 
\begin{definition}
The data for the homotopy double groupoid, $\boldsymbol{\rho}^{\square} (X) $, 
will be denoted by :

\[
\begin{array}{c}
(\boldsymbol{\rho}^{\square}_2 (X), \boldsymbol{\rho}_1^{\square} (X) ,
\partial^{-}_{1} , \partial^{+}_{1} , +_{1} , \varepsilon _{1}) ,
\boldsymbol{\rho}^{\square}_2 (X), \boldsymbol{\rho}^{\square}_1 (X) ,
\partial^{-}_{2} , \partial^{+}_{2} , +_{2} , \varepsilon _{2})\\[3mm]
(\boldsymbol{\rho}^{\square}_1 (X) , X , \partial^{-} , \partial^{+} , + , \varepsilon).
\end{array}\]
\bigbreak

 Here $\boldsymbol{\rho}_1 (X)$ denotes the \emph{path groupoid} of $X$
from ref. \cite{HKK}  where it was defined as follows. The objects of 
$\boldsymbol{\rho}_1 (X) $ are the points of $ X $. The morphisms of 
$\boldsymbol{\rho}^\square_1 (X) $ are the equivalence classes of paths in $ X$ with respect 
to the following (thin) equivalence relation $ \sim_{T} $, defined as follows. The data for 
$\boldsymbol{\rho}^{\square}_2 (X)$ is defined last; furthermore, the symbols specified after the thin square symbol specify both the sides (or the groupoid `dimensions') of the square which are involved (i.e., 1 and 2, respectively), and also the order in which the shown operations ($\partial^{-}_{1}$, $\varepsilon _{2}$... , etc) are to be performed relative to the thin square specified for each groupoid, $\rho_1 ~ or~ \rho_2$; moreover, all such symbols are explicitly and precisely defined in the related entries of the concepts involved in this definition. These two groupoids can also be pictorially represented as the $(H,V)$ pair depicted in the large Diagram (0.1), or $\D$, shown at the top of this page.
\end{definition}

\begin{definition} \emph{Thin equivalence}

 Let $ a,a' : x \simeq y $ be paths in $ X $. Then
$ a$ is \emph{ thinly equivalent} to $ a' $, denoted $ a \sim_{T} a' $, if
there is a thin relative homotopy between $ a $ and $ a' $.

 We note that $ \sim_{T} $ is an equivalence relation, see
\cite{BHKP}. We use $ \langle a \rangle : x \simeq y $ to denote
the $ \sim_{T} $ class of a path $ a: x \simeq y $ and call $
\langle a \rangle $  the {\it semitrack} of $ a $. The groupoid
structure of $ \boldsymbol{\rho}^\square_1 (X) $ is induced by concatenation,
+, of paths. Here one makes use of the fact that if $ a: x \simeq
x', \ a' : x' \simeq x'', \ a'' : x'' \simeq x''' $ are paths then
there are canonical thin relative homotopies
\[
\begin{array}{r}
(a+a') + a'' \simeq a+ (a' +a'') : x \simeq x''' \ ({\it rescale}) \\
a+e_{x'} \simeq a:x \simeq x' ; \ e_{x} + a \simeq a: x \simeq x' \
({\it dilation}) \\
a+(-a) \simeq e_{x} : x \simeq x \ ({\it cancellation}).
\end{array}
\]

The source and target maps of  $\boldsymbol{\rho}^\square_1 (X)$ are given by
$$\partial^{-}_{1} \langle a\rangle =x,\enskip \partial^{+}_{1}
\langle a\rangle =y,$$
if $\langle a\rangle :x\simeq y$ is a semitrack. Identities and inverses
are given by
$$\varepsilon (x)=\langle e_x\rangle  \quad \mathrm{ resp.} -\langle a\rangle
=\langle -a \rangle.$$
\end{definition}

At the next step, in order to construct the groupoid $\boldsymbol{\rho}^{\square}_2 (X)$
data in Definition 0.1, R. Brown et al. defined as follows a 
\emph{relation of cubically thin homotopy} on the set $R^{\square}_2(X)$ of squares.

\begin{definition}  
\emph{Cubically thin homotopy}

 Let $u,u'$ be squares in $X$ with common vertices.
\begin{enumerate}
\item A {\it cubically thin homotopy} $U:u\equiv^{\square}_T u'$
between $u$ and $u'$ is a cube $U\in R^{\square}_3(X)$ such that

(i) $U$ is a homotopy between $u$ and $u',$

\begin{center}
i.e. $\partial^{-}_1 (U)=u,\enskip \partial^{+}_1 (U)=u',$\end{center}
(ii) $U$ is rel. vertices of $I^2,$
\begin{center}
i.e. $\partial^{-}_2\partial^{-}_2 (U),\enskip\partial^{-}_2
\partial^{+}_2 (U),\enskip
\partial^{+}_2\partial^{-}_2 (U),\enskip\partial^{+}_2
\partial^{+}_2 (U)$ are
constant,\end{center}
(iii) the faces $ \partial^{\alpha}_{i} (U) $ are thin for $ \alpha =
\pm 1, \ i = 1,2 $.


\item The square $u$ is {\it cubically} $T$-{\it equivalent} to
$u',$ denoted $u\equiv^{\square}_T u'$ if there is a cubically
thin homotopy between $u$ and $u'.$
\end{enumerate}

\end{definition}

{\bf Remark}
By removing from the above double groupoid construction the condition that all morphisms
must be invertible one obtains the prototype of a \emph{double category}.  


\begin{thebibliography}{9}

\bibitem{HKK}
K.A. Hardie, K.H. Kamps and R.W. Kieboom., A homotopy 2-groupoid of a Hausdorff 
\emph{Applied Categorical Structures}, \textbf{8} (2000): 209-234.

\bibitem{BHKP}
R. Brown, K.A. Hardie, K.H. Kamps  and T. Porter., 
\PMlinkexternal{A homotopy double groupoid of a Hausdorff space}{http://www.tac.mta.ca/tac/volumes/10/2/10-02.pdf} ,
{\it Theory and Applications of Categories} \textbf{10},(2002): 71-93.

\end{thebibliography}

%%%%%
%%%%%
\end{document}
